\documentclass[11pt]{article}
\usepackage{amsmath,amssymb,amsthm,enumerate,nicefrac,fancyhdr,hyperref,graphicx,adjustbox}
\hypersetup{colorlinks=true,urlcolor=blue,citecolor=blue,linkcolor=blue}
\usepackage[left=2.6cm, right=2.6cm, top=1.5cm, includehead, includefoot]{geometry}
\usepackage[dvipsnames]{xcolor}
\usepackage[d]{esvect}

%% commands
%% useful macros [add to them as needed]
% sets
\newcommand{\C}{{\mathbb{C}}} 
\newcommand{\N}{{\mathbb{N}}}
\newcommand{\Q}{{\mathbb{Q}}}
\newcommand{\R}{{\mathbb{R}}}
\newcommand{\Z}{{\mathbb{Z}}}
\newcommand{\F}{{\mathbb{F}}}

% bases
\newcommand{\mA}{\mathcal{A}}
\newcommand{\mB}{\mathcal{B}}
\newcommand{\mC}{\mathcal{C}}
\newcommand{\mD}{\mathcal{D}}
\newcommand{\mE}{\mathcal{E}}
\newcommand{\mL}{\mathcal{L}}
\newcommand{\mM}{\mathcal{M}}
\newcommand{\mO}{\mathcal{O}}
\newcommand{\mP}{\mathcal{P}}
\newcommand{\mS}{\mathcal{S}}
\newcommand{\mT}{\mathcal{T}}

% linear algebra
\newcommand{\diag}{\operatorname{diag}}
\newcommand{\adj}{\operatorname{adj}}
\newcommand{\rank}{\operatorname{rank}}
\newcommand{\spn}{\operatorname{Span}}
\newcommand{\proj}{\operatorname{proj}}
\newcommand{\prp}{\operatorname{perp}}
\newcommand{\refl}{\operatorname{refl}}
\newcommand{\tr}{\operatorname{tr}}
\newcommand{\nul}{\operatorname{Null}}
\newcommand{\nully}{\operatorname{nullity}}
\newcommand{\range}{\operatorname{Range}}
\renewcommand{\ker}{\operatorname{Ker}}
\newcommand{\col}{\operatorname{Col}}
\newcommand{\row}{\operatorname{Row}}
\newcommand{\cof}{\operatorname{cof}}
\newcommand{\Num}{\operatorname{Num}}
\newcommand{\Id}{\operatorname{Id}}
\newcommand{\ipb}{\langle \thinspace, \rangle}
\newcommand{\ip}[2]{\left\langle #1, #2\right\rangle} % inner products
\newcommand{\M}[2]{M_{#1\times #2}(\F)}
\newcommand{\RREF}{\operatorname{RREF}}
\newcommand{\cv}[1]{\begin{bmatrix} #1 \end{bmatrix}}
\newenvironment{amatrix}[1]{\left[\begin{array}{@{}*{\numexpr#1-1}{c}|c@{}}}{\end{array}\right]}
\newcommand{\am}[2]{\begin{amatrix}{#1} #2 \end{amatrix}}

% vectors
\newcommand{\vzero}{\vv{0}}
\newcommand{\va}{\vv{a}}
\newcommand{\vb}{\vv{b}}
\newcommand{\vc}{\vv{c}}
\newcommand{\vd}{\vv{d}}
\newcommand{\ve}{\vv{e}}
\newcommand{\vf}{\vv{f}}
\newcommand{\vg}{\vv{g}}
\newcommand{\vh}{\vv{h}}
\newcommand{\vl}{\vv{\ell}}
\newcommand{\vm}{\vv{m}}
\newcommand{\vn}{\vv{n}}
\newcommand{\vp}{\vv{p}}
\newcommand{\vq}{\vv{q}}
\newcommand{\vr}{\vv{r}}
\newcommand{\vs}{\vv{s}}
\newcommand{\vt}{\vv{t}}
\newcommand{\vu}{\vv{u}}
\newcommand{\vvv}{{\vv{v}}}
\newcommand{\vw}{\vv{w}}
\newcommand{\vx}{\vv{x}}
\newcommand{\vy}{\vv{y}}
\newcommand{\vz}{\vv{z}}

% display
\newcommand{\ds}{\displaystyle}
\newcommand{\qand}{\quad\text{and}}
\newcommand{\qandq}{\quad\text{and}\quad}
\newcommand{\hint}{\textbf{Hint: }}

% misc
\newcommand{\area}{\operatorname{area}}
\newcommand{\vol}{\operatorname{vol}}
\newcommand{\red}[1]{{\color{red} #1}}
\newcommand{\rc}{\red{\checkmark}}

\title{STAT 231 Notes}
\author{Thomas Liu}
\begin{document}
\maketitle
\tableofcontents

\newpage 

\section{Lecture 1}
\subsection{Empirical Study}
Empirical study is one in which knowledge is gained by observation or by experiment to 
\begin{itemize}
  \item help further knowledge 
  \item improve systems 
  \item determine public policy
\end{itemize}
Empirical studies deal with populations and provesses which are collections of individual units
\subsection{Data Collection}
\subsubsection{Variate}
a characteristic of a unit 
\begin{itemize}
  \item continuous
  \item discrete
  \item categorical
  \item ordinal
  \item complex
\end{itemize}
\subsubsection{Attribute}
a function of the variates over the population or process 
\subsubsection{Types of Empirical Studies}
\begin{itemize}
  \item sample surveys
  \begin{itemize}
    \item select a representative sample of units from the population
    \item determine the variates of interest for each unit in the sample 
  \end{itemize}
  \item observational studies 
  \begin{itemize}
    \item data are collected about a population or process without any attempts to change the value of one or more variates for the sampled units 
    \item subtle distinction from sample survey 
  \end{itemize}
  \item experimental studies 
  \begin{itemize}
    \item the experimenter intervenes and changes or sets the values of one or more variates for the units in the sample
  \end{itemize}
\end{itemize}
\subsection{Measures of Location}
\begin{itemize}
  \item sample mean: $\overline{y} = \frac{1}{n}\sum_{i=1}^{n}y_i$
  \item sample median: $\hat{m}$ (middle value)
  \item sample mode: the value of $y$ which appears in the sample with the highest frequency
\end{itemize}
\subsection{Variability}
Measures of variability or dispersion:
\begin{itemize}
  \item sample variance: $s^2 = \frac{1}{n-1}\sum_{i=1}^{n}(y_i-\overline{y})^2$ \\
  take square root for the sample standard deviation
  \item range: $y_n-y_1$ where $y_n= max(y)$ and $y_1= min(y)$
\end{itemize}
\subsection{Measures of Shape}
\begin{itemize}
  \item sample skewness: measures ``lack of symmetry'' in the data, can be positive or negative \[g_1 = \ds\frac{\frac{1}{n}\ds\sum_{i=1}^{n}(y_i-\overline{y})^3}{\Big[\frac{1}{n}\ds\sum_{i=1}^{n}(y_i-\overline{y})^2\Big]^{3/2}}\]
  \begin{itemize}
    \item positive skew: more data on the left
    \item negative skew: more data on the right 
  \end{itemize}
  \item sample kurtosis: measures the heaviness of the tails of the data, always positive \[g_2 = \ds\frac{\frac{1}{n}\ds\sum_{i=1}^{n}(y_i-\overline{y})^4}{\Big[\frac{1}{n}\ds\sum_{i=1}^{n}(y_i-\overline{y})^2\Big]^{2}}\]
\end{itemize}
\subsection{Sample Quantiles}
\begin{itemize}
  \item define $p$th quantile (or $100p$th percentile) as a value st approximately a fraction $p$ of the $y$ values fall less than $q(p)$
  \item order dataset $\{y_1,\cdots,y_n\}$ from smallest to largest: $\{y_{(1)},\cdots,y_{(n)}\}$
  \item find $p$th sample quantile
  \begin{itemize}
    \item let $k=(n+1)p$, where $n$ is the sample size 
    \item if $k$ is integer, $1\leq k\leq n$, then $q(p)=y_{(k)}$
    \item otherwise, $q(p)$ is the average of $y_{(j)}$ and $y_{(j+1)}$ where $j$ and $j+1$ are the two closest integers that $k$ falls between 
  \end{itemize}
  \item interquantile range (IQR): $q(0.75)-q(0.25)$
\end{itemize}
\subsection{Sample Correlation for Continuous Bivariate Data}
For a sample of data $\{(x_1,y_1),\cdots,(x_n,y_n)\}$, the sample correlation is defined as \[r=\dfrac{S_{xy}}{\sqrt{S_{xx}S_{yy}}}\]
where 
\begin{itemize}
  \item $S_{xy}=\sum_{i=1}^{n}(x_i-\overline{x})(y_i-\overline{y})$
  \item $S_{xx}=\sum_{i=1}^{n}(x_i-\overline{x})^2$
  \item $S_{yy}=\sum_{i=1}^{n}(y_i-\overline{y})^2$
  \item $r$ takes on values between $-1$ and $1$
  \item measure of strength of linear relationship between $x$ and $y$
\end{itemize}

\section{Lecture 2}
\subsection{Histograms}
A histogram is a way of reresenting frequencies in a dataset $\{y_1,\cdots,y_n\}$ using rectangles \\
Partition the range of $y$ into $k$ non-overlapping intervals $l_j=[a_{j-1}, a_j)$, $j=1,\cdots,k$ 
\subsubsection{Types of Histograms}
Two ways of determining the heights of the rectangles: 
\begin{itemize}
  \item standard frequency histogram: intervals are of equal length, the height is the frequency $f_j$ or relative frequency $f_j/n$
  \item relative frequency histogram: to adjust for intervals being of different lengths, set the height to 
  \[\frac{f_j/n}{a_j-a_{j-1}}\]
\end{itemize}
\subsection{Bar Graphs and Pie Charts}
Bar graphs and pie charts are useful ways of visualizing frequencies for categorical (non-numberic) data 
\subsection{Empirical Cumulative Distribution Functions}
Suppose the dataset are from an unknown cumulative distribution function $F(y)=P(Y\leq y)$, then the empirical cumulative distribution function (ecdf):
\[\hat{F}(y) = \frac{\text{Number of values in the set }\{y_1,\cdots,y_n\}\leq y}{n}\]
\begin{figure}[tbhp]
	\begin{center}
		\includegraphics[width=0.5\textwidth]{ECDF for right-skewed data.png}
	\end{center}
	\caption{ECDF for right-skewed data}
	\label{figcaption}
\end{figure}
\begin{figure}[tbhp]
	\begin{center}
		\includegraphics[width=0.5\textwidth]{ECDF for left-skewed data.png}
	\end{center}
	\caption{ECDF for left-skewed data}
	\label{figcaption}
\end{figure}
\subsection{Boxplots}
Boxplots are a useful way of visualizing data with few numbers of groups or small smaple sizes
\begin{itemize}
  \item The line inside the box is the sample median $(q(0.5))$
  \item The top edge of the box is the upper quantile $(q(0.75))$
  \item The lower edge of the box is the lower quantile $(q(0.25))$
  \item The lower line is placed at the smallest observed data value that is larger than the value $q(0.25)-1.5\times IQR$ where $IQR = q(0.75) - q(0.25)$ is the interquantile range 
  \item The upper line is placed at the largest observed data value that is smaller than the value $q(0.75)+1.5\times IQR$
  \item Values beyond the whiskers are called outliers
\end{itemize}
\subsection{Scatterplots}
Scatterplots can be used to visualize the relationship between two variates \\
THe magnitude of the sample correlation $r$ reflects the strength of a linear relationship between the two variates 
\subsection{Run Charts}
A run chart is useful for depicting changes in a variate over time
\subsection{Statistical Models and Uses}
A probability-based model that describes a provess or the selection of units and measurement of variates for a population 
\begin{itemize}
  \item random variables can describe variation in variate values 
  \item questions are often formulated in terms of model parameters (e.g. the proportion of Canadians who drink coffee every morning as of January 8, 2024)
  \item can draw parallels between sample-based summaries of $\{y_1,\cdots,y_n\}$ and properties of the corresponding probability model for $Y$
\end{itemize}

\section{Lecture 3}
\subsection{Random Variables}
\begin{figure}[tbhp]
	\begin{center}
		\includegraphics[width=0.9\textwidth]{Random Variables.png}
	\end{center}
	\caption{Properties of discrete versus continuous random variables}
	\label{figcaption}
\end{figure}
\newpage 
\subsection{Point Estimates}
The value of a function of observed data $\{y_1,\cdots,y_n\}$ and other known quatities such as sample size $n$
\begin{itemize}
  \item notation: $\hat{\theta}$ is an estimate of $\theta$
  \item depends on the sample of data at hand 
  \item a point estimate is also a statistic because it does not contain any unknown quantities 
\end{itemize}
\subsection{Likelihood Function for Discrete Distributions}
\begin{itemize}
  \item Notation: let discret (vector) random variable $Y$ represent potential data to estimate $\theta$ and let $y$ represent the data that are actually observed 
  \item The likelihood function for $\theta$ is defined as \[L(\theta) = L(\theta; y) = P(Y=y; \theta)\text{\ for $\theta\in\Omega$}\]
\end{itemize}
\subsection{Maximum Likelihood Estimates}
Maximum likelihood estimate of $\theta$: the value of $\theta$ that maximizes $L(\theta)$ for the given data $y$
\begin{itemize}
  \item not unique 
  \item depends on the sample $y$
\end{itemize}
\subsection{Relative Likelihood Function}
The relative likelihood function is defined as \[R(\theta) = \frac{L(\theta)}{L(\hat{\theta})}, \theta\in\Omega\]
\begin{itemize}
  \item $0\leq R(\theta)\leq 1$ for all $\theta$
  \item the log likelihood function is defined as \[\log L(\theta) = \ln L(\theta) = l(\theta)\]
  \item easier working with the log likelihood when trying to calculate the MLE, we can use $\dfrac{dl(\theta)}{d\theta}$ rather than $\dfrac{dL(\theta)}{d\theta}$
\end{itemize}

\section{Lecture 4}
\subsection{Likelihood Functions for Continuous Distribution}
$\ds P(Y=y)=\prod_{i=1}^{n}\triangle f(y_i;\theta)$
\begin{itemize}
  \item $\triangle$ is a very small interval 
  \item $\triangle^n$ is even smaller and can be ignored in our likelihood and MLE calculations
\end{itemize}
\begin{figure}[tbhp]
	\begin{center}
		\includegraphics[width=0.8\textwidth]{Max Likelihood Method.png}
	\end{center}
	\caption{Summary of Maximum Likelihood Method for Named Distributions}
	\label{figcaption}
\end{figure}
\newpage 
\subsection{Multinomial Distributions}
The multinomial distribution is used to model $n$ independent trials where each trial has one of $k$ possible outcomes (outcomes $1,\cdots,k$) \\
The discrete random variables $Y_1,\cdots, Y_n$ have joint probability function 
\[P(Y_1=y_1,\cdots, Y_n=y_n; \theta) = f(y_1,\cdots,y_n;\theta) = \dfrac{n!}{y_1!\cdots y_k!}\theta_1^{y_1}\cdots\theta_k^{y_k}\]
We write $(Y_1,\cdots,Y_n)\sim\text{Multinomial}(n;\theta)$
\subsection{Likelihood Function for the Multinomial Distribution}
The multinomial distribution 
\[L(\theta) = \dfrac{n!}{y_1!\cdots y_k!}\theta_1^{y_1}\cdots\theta_k^{y_k}=\dfrac{n!}{y_1!\cdots y_k!}\prod_{i=1}^k\theta^{y_i}\]
Also 
\[\sum_{i=1}^{k}y_i = n\]
It can be shown that 
\[\hat{\theta_i} = \frac{y_i}{n}, i= 1,\cdots,k\]

\section{Lecture 5}
\subsection{Invariance Property of Maximum Likelihood Estimates}
If $\hat{\theta} = (\hat{\theta_1},\cdots,\hat{\theta_k})$ is the MLE of $\theta = (\theta_1,\cdots,\theta_k)$, then $g(\hat{\theta})$ is the MLE of $g(\theta)$

\section{Lecture 6}
Chapter $2$ review

\section{Lecture 7}
\subsection{Issue with Parameter Estimation}
The likelihood function is based on the probability of the observed sample of data \
\begin{itemize}
  \item Parameter estimation is data dependent 
  \item assume that the variate of interest is measured without error for a random sample of units 
\end{itemize}
\subsection{Point Estimate}
A point estimate $\hat{\theta}$ of $\theta$ is a function of the observed sample data $\{y_1,\cdots y_n\}$. ex:
\[\hat{\theta} = g(y_1,\cdots,y_n) = \frac{1}{n}\sum_{i=1}^{n}y_i\]
\subsection{Estimator}
An estimator $\tilde{\theta}$ is a function of random variables, i.e. $g(Y_1,\cdots,Y_n)$. Tells how to use data to obtain a numerical estimate $\hat{\theta} = g(y_1,\cdots,y_n)$
\subsection{Interval Estimator, Confidence Interval}
Suppose the interval estimator $[L(Y),U(Y)]$ has the property that 
\[P(\theta\in[L(Y),U(Y)]) = P(L(Y)\leq\theta\leq U(Y)) = p\]
The interval estimate $[L(Y),U(Y)]$ constructed for $\theta$ based on observed data $\{y_1,\cdots,y_n\}$ is called a $100p\%$ confidence interval for $\theta$

\section{Lecture 8}
\subsection{Pivotal Quantity}
A pivotal quantity $Q=Q(Y;\theta)$ is a function of data $Y$ and parameter $\theta$ such that $Q$ is a random variable with known distribution \\
Suppose we can rearrange the inequality \[a\leq Q(Y;\theta)\leq b\]a
as \[L(Y)\leq\theta\leq U(Y)\]
Then 
\begin{align*}
  p &= P(a\leq Q(Y;\theta)\leq b) \\  
    &= P(L(Y)\leq\theta\leq U(Y)) \\
    &= P(\theta\in[L(Y),U(Y)])
\end{align*} 

\section{Lecture 9}
\subsection{Likelihood Interval}
Define a $100p\%$ likelihood interval for $\theta$ as set \[\theta:R(\theta)>p\]
\begin{itemize}
  \item likelihood intervals can be determined approximately by plotting $R(\theta)$
  \item more accurate solution: $R(\theta)-p=0$
  \item likelihood intervals take on the form \[L(y),U(y)\]
  \item $L(y),U(y)$ are based on observed data 
\end{itemize}
\subsection{Log Relative Likelihood Functions}
\[r(\theta)=\log(R(\theta))=l(\theta)-l(\hat{\theta})\]

\section{Lecture 10}
\subsection{Gamma Function}
\[\Gamma(\alpha) = \int_{0}^{\infty}y^{\alpha-1}e^{-y}dy\quad\alpha>0\]
\begin{itemize}
  \item $\Gamma(\alpha)=(\alpha-1)\Gamma(\alpha-1)$
  \item $\Gamma(\alpha)=(\alpha-1)!$ for $\alpha=1,2,\cdots$
  \item $\Gamma(\frac{1}{2})=\sqrt{\pi}$
\end{itemize}
\subsection{The $\chi^2$ (Chi-squared) Distribution}
The $\chi^2(k)$ distribution is a continuous family of distribution on $(0,\infty)$ with probability density function of the form 
\[f(x;k) = \dfrac{1}{2^{k/2}\Gamma(k/2)}x^{(k/2)-1}e^{-x/2}\quad\quad x>0\]
where $k\in\{1,2,\cdots\}$ \\
For values $k\geq30$, the pdf resembles that of a $N(k,2k)$ pdf \\
If $X\sim\chi^2(k)$ then 
\[E(X)=k\text{ and } Var(X)=2k\]
\subsection{Theorem}
Let $W_1,W_2,\cdots,W_n$ be independent random variables with $W_i\sim\chi^2(k_2)$.
Then \[S=\sum_{i=1}^{n}W_i\sim\chi^2(\sum_{i=1}^{n}k_i)\]
\subsection{Theorem}
If $Z\sim G(0,1)$ the the distribution of $W=Z^2$ is $\chi^2(1)$
\subsection{Corollary}
If $Z_1,Z_2,\cdots,Z_n$ are mutually independent $G(0,1)$ random variables and $S = \ds\sum_{i=1}^{n}Z_i^2$, then $S\sim\chi^2(n)$
\subsection*{Useful Results}
\begin{itemize}
  \item if $W\sim\chi^2(1)$ then $P(W\geq w) = 2[1-P(Z\leq\sqrt{w})]$ where $Z\sim G(0,1)$
  \item if $W\sim\chi^2(2)$ then $W\sim Exponential(2)$ and $P(W\geq w) = e^{-w/2}$
\end{itemize}

\section{Lecture 11}
\subsection{Student's t Distribution}
Student's $t$ distribution (or more simply the $t$ distribution) has probability density function 
\[f(t;k) = c_k(1+\dfrac{t^2}{k})^{-(k+1)/2}\quad\text{ for $t\in\R$ and $k=1,2,\cdots$}\]
where constant $c_k$ is 
\[c_k = \dfrac{\Gamma(\frac{k+1}{2})}{\sqrt{k \pi}\Gamma(\frac{k}{2})}\]
\subsection{Theorem}
Suppose $Z\sim G(0,1)$ and $U\sim\chi^2(k)$ independently. Let 
\[T = \frac{Z}{\sqrt{U/k}}\]
Then $T$ has a Student's $t$ distribution with $k$ degrees of freedom

\section{Lecture 12}
\subsection{Likelihood Ratio Statistic}
Let random variable $\wedge(\theta)$
\[\wedge(\theta) = -2\log\Big[\dfrac{L(\theta)}{L(\tilde{\theta})}\Big]\]
where $\tilde{\theta}$ is the maximum likelihood estimator
\subsection{Theorem}
A $100p\%$ likelihood interval is an approximate $100q\%$ confidence interval where $q=2P(Z\leq\sqrt{-2\log p})-1$ and $Z\sim N(0,1)$
\subsection{Theorem}
If $a$ ia a value such that $p=2P(Z\leq a)-1$ where $Z\sim N(0,1)$, then the likelihood interval $\{\theta:R(\theta)\geq e^{-a^2/2}\}$ is an approximate $100p\%$ confidence interval 
\subsection{Theorem}
Suppose $Y_1,\cdots,Y_n$ is a random sample from the $G(\mu,\sigma)$ distribution with sample mean $\overline{Y}$ and sample variance $S^2$. Then 
\[T = \frac{\overline{Y}-\mu}{S/\sqrt{n}}\sim t(n-1)\]
\subsection{Theorem}
Suppose $Y_1,\cdots,Y_n$ is a random sample from the $G(\mu,\sigma)$ distribution with sample variance $S^2$
\[U = \dfrac{(n-1)S^2}{\sigma^2} = \frac{1}{\sigma^2}\sum_{i=1}^{n}(Y_i-\overline{Y})^2 = \sum_{i=1}^{n}(\frac{Y_i-\overline{Y}}{\sigma})^2\sim \chi^2(n-1)\]

\section{Lecture 13}

\section{Lecture 14}
\subsection{Null Hypothesis}
The default hypothesis is often referred to as the ``null'' hypothesis and is denoted by $H_0$ \\
There is an alternative hypothesis $H_A$, not always specified, usually $H_A$ is that $H_0$ is not true
\subsection{Test Statistic}
A test statistic or discrepancy measure $D$ is a function of the data $Y$ that is constructed to measure the degree of ``agreement'' between the data $Y$ and the null hypothesis $H_0$
\subsection{$P$-Value or Observed Significance}
Suppose we use the test statistic $D=D(Y)$ to test the hypothesis $H_0$. Suppose also that $d=D(y)$ be the corresponding is the observed value of $D$. The $p$-value or observed significance level of the test of hypothesis $H_0$ using test statistic $D$ is 
\[p-value = P(D\geq d;H_0)\]
\begin{center}
\begin{tabular}{c|c}
  p-value &interpretation \\ \hline
  p-value$ >0.10$ &no evidence against $H_0$ \\ \hline 
  $0.05 < $p-value$ \leq0.10$ &weak evidence against $H_0$ \\ \hline 
  $0.01 < $p-value$ \leq0.05$ &evidence against $H_0$ \\ \hline 
  $0.001 < $p-value$ \leq0.01$ &strong evidence against $H_0$ \\ \hline 
  p-value$ \leq 0.001$ &very strong evidence against $H_0$
\end{tabular}
\end{center}

\section{Lecture 15}
\subsection{Relationship between Hypothesis Testing and Interval Estimation}
The $p$-value for testing $H_0:\mu=\mu_0$ is greater than or equal to $0.05$ iff the value $\mu=\mu_0$ is an element of a $95\%$ confidence interval for $\mu$
\subsection{Find $p$-value for Likelihood Ratio Statistic}
First find observed value of $\wedge(\theta_0)$, denote as 
\[\lambda(\theta_0) = -2\log\Big[\frac{L(\theta_0)}{L(\hat{\theta})}\Big] = -2\log R(\theta_0)\]
where $R(\theta_0)$ is the relative likelihood function evaluated at $\theta=\theta_0$ \\
The approx $p$-value is 
\begin{align*}
  p-value &\approx P[W\geq\lambda(\theta_0)] &W\sim\chi^2(1) \\
          &= P(|Z|\geq\sqrt{\lambda(\theta_0)}) &Z\sim G(0,1) \\
          &= 2[1-P(Z\leq \sqrt{\lambda(\theta_0)})]
\end{align*}

\section{Lecture 16}
\subsection{Gaussian Response / Linear Regression}
A Gaussian response model is one for which the distribution of the response variate $Y$, given the associated verctor of covariates $x=(x_1,x_2,\cdots,x_k)$ for an individual unit 
\[Y\sim G(\mu(x), \sigma(x))\]
If observations are made on $n$ randomly selected units we write 
\[Y_i\sim G(\mu(x_i), \sigma(x_i))\ \ \text{ for $i=1,2,\cdots,n$ independently}\]
In most cases assume $\sigma(x_i)=\sigma$ is constant \\
Difference in Gaussian response models is the choice of function $\mu(x)$, and covariates \\
Often assume $\mu(x_i)$ is linear function 
\[Y_i\sim G(\mu(x_i), \sigma)\ \text{for $i=1,2,\cdots,n$ independently}\]
with \[\mu(x_i) = \beta_0+\sum_{j=1}^{k}\beta_jx_{ij}\]
These models are also referred to as linear regression models, $\beta_j$ are regression coefficients

\section{Lecture 17}
\subsection{Simple Linear Regression}
Consider case which there is a single covariate $x$ \\
Model with independent $Y_i$'s such that \[Y_i\sim G(\mu(x_i),\sigma)\text{ where }\mu(x_i)=\alpha+\beta(x_i)\]
The likelihood function for $(\alpha,\beta,\sigma)$
\[L(\alpha,\beta,\sigma) = \sigma^{-n}\exp\Big[-\frac{1}{2\sigma^2}\sum_{i=1}^{n}(y_i-\alpha-\beta x_i)^2\Big]\ \ \ \alpha,\beta\in\R, \sigma>0\]
Solve to get 
\[\hat{\beta} = \frac{S_{xy}}{S_{xx}}\]
\[\hat{\alpha} = \overline{y}-\hat{\beta}\overline{x}\]
\[\sigma^2 = \frac{1}{n}(S_{yy}-\hat{\beta}S_{xy})\]
where 
\[S_{xx}=\sum_{i=1}^{n}(x_i-\overline{x})^2, S_{yy} = \sum_{i=1}^{n}(y_i-\overline{y})^2, S_{xy} = \sum_{i=1}^{n}(x_i-\overline{x})(y_i-\overline{y})\]
\subsection{Least Squares Estimation}
To find a line of ``best fit'' which minimizes the sum of squares of distance between observed point and fitted line $y=\alpha+\beta x$, we want to find value $\alpha$ and $\beta$ that minimize 
\[g(\alpha,\beta) = \sum_{i=1}^{n}[y_i-(\alpha+\beta x_i)]^2\]
Or to find 
\[\frac{\partial g}{\partial \alpha} = \sum_{i=1}^{n}(y_i-\alpha-\beta x_i) = 0\]
\[\frac{\partial g}{\partial \beta} = \sum_{i=1}^{n}(y_i-\alpha-\beta x_i)x_i = 0\]
The line $y=\hat{\alpha} + \hat{\beta}x$ is often called fitted regression line for $y$ on $x$, or fitted line 
\subsection{Distribution of the estimator $\tilde{\beta}$}
\[\tilde{\beta}\sim G(\beta, \frac{\sigma}{\sqrt{S_{xx}}})\]
\subsection{Confidence Intervals for $\beta$ and Test of Hypothesis}
We have \[\frac{\tilde{\beta}-\beta}{\sigma/\sqrt{S_{xx}}}\sim G(0,1)\]
holds \[\frac{(n-2)S_e^2}{\sigma^2}\sim\chi^2(n-1)\]
Then \[\frac{\tilde{\beta}-\beta}{\sigma/\sqrt{S_{xx}}}\sim t(n-2)\]
$100p\%$ confidence interval for $\beta$
\begin{align*}
  p &= P(-a\leq T\leq a) &T\sim t(n-2) \\
    &= P(\hat{\beta}-aS_e/\sqrt{S_{xx}}\leq\beta\leq\hat{\beta}+aS_e/\sqrt{S_{xx}})
\end{align*}
or \[\hat{\beta}\pm aS_e/\sqrt{S_{xx}}\]
To test hypothesis $H_0:\beta=0$, use test statistic \[\frac{|\tilde{\beta}-0|}{S_e/\sqrt{S_{xx}}}\]
$p$-value given 
\begin{align*}
  p-value &= P(|T|\geq \frac{|\tilde{\beta}-0|}{s_e/\sqrt{S_{xx}}}) \\
          &= 2\Big[1 - P(T\leq \frac{|\tilde{\beta}-0|}{s_e/\sqrt{S_{xx}}})\Big] &T\sim t(n-2)
\end{align*}
$100p\%$ confidence interval for $\sigma^2$ 
\[\frac{(n-2)s_e^2}{b}, \frac{(n-2)s_e^2}{b}\]
\subsection{Confidence Interval for Mean Response $\mu(x) = \alpha+\beta x$}
\[E[\tilde{\mu}(x)] = \mu(x)\]
\[Var[\tilde{\mu}(x)] = \sigma^2[\frac{1}{n}+\frac{(x-\overline{x})^2}{S_{xx}}]\]
We have \[\tilde{\mu}(x)\sim G(\mu(x), \sigma\sqrt{\frac{1}{n}+\frac{(x-\overline{x})^2}{S_{xx}}})\]
Then $100p\%$ confidence interval for $\mu(x)$ is 
\[\tilde{\mu}(x) - as_e\sqrt{\frac{1}{n}+\frac{(x-\overline{x})^2}{S_{xx}}}, \tilde{\mu}(x) + as_e\sqrt{\frac{1}{n}+\frac{(x-\overline{x})^2}{S_{xx}}}\]
\subsection{Prediction Interval for Future Response}
\[E[Y-\tilde{\mu}(x)] = 0\]
\[Var[Y-\tilde{\mu}(x)] = \sigma^2[1+\frac{1}{n}+\frac{(x-\overline{x})^2}{S_{xx}}]\]
We have \[Y-\tilde{\mu}(x)\sim G(0, \sigma[1+\frac{1}{n}+\frac{(x-\overline{x})^2}{S_{xx}}]^{1/2})\]
Then $100p\%$ prediction interval is 
\[\tilde{\mu}(x)-as_e\sqrt{1+\frac{1}{n}+\frac{(x-\overline{x})^2}{S_{xx}}}, \tilde{\mu}(x)+as_e\sqrt{1+\frac{1}{n}+\frac{(x-\overline{x})^2}{S_{xx}}}\]

\section{Lecture 18}
\subsection{Residual Plot}
Residuals are defined as difference between observed response $y_i$ and fitted response $\mu_i=\tilde{\alpha}+\tilde{\beta}x$

\section{Lecture 19}
\subsection{Two Gaussian Populations with Common Variance}
The likelihood function for $\mu_1$, $\mu_2$, $\sigma$ is 
\[L(\mu_1,\mu_2,\sigma) = \prod_{j=1}^{2}\prod_{i=1}^{n_j}\frac{1}{\sqrt{2\pi}\sigma}exp[-\frac{1}{2\sigma^2}(y_{ji}-\mu_j)^2]\]
with 
\[\hat{\mu}_1 = \frac{1}{n_1}\sum_{i=1}^{n_1}y_{1i}=\overline{y}_1\]
\[\hat{\mu}_2 = \frac{1}{n_2}\sum_{i=1}^{n_2}y_{2i}=\overline{y}_2\]
\[\hat{\sigma}^2 = \frac{1}{n_1+n_2}[\sum_{i=1}^{n_1}(y_{1i}-\overline{y}_1)^2 + \sum_{i=1}^{n_2}(y_{2i}-\overline{y}_2)^2]\]
Estimate of variance $\sigma^2$ called pooled estimate of variance 
\[s^2_p = \frac{n_1+n_2}{n_1+n_2-2}\sigma^2\]
\subsection{Confidence intervals for $\mu_1-\mu_2$}
We have \[E(\overline{Y}_1 - \overline{Y}_2) = \mu_1 - \mu_2\]
and \[Var(\overline{Y}_1 - \overline{Y}_2) = \sigma^2(\frac{1}{n_1} + \frac{1}{n_2})\]
If $Y_{11},Y_{12},\cdots,Y_{1n_1}$ is random sample form a $G(\mu_1,\sigma)$ distribution and independently for $G(\mu_2,\sigma)$, then 
\[\frac{(\overline{Y}_1 - \overline{Y}_2) - (\mu_1-\mu_2)}{S_p\sqrt{\frac{1}{n_1}+\frac{1}{n_2}}}\sim t(n_1+n_2-2)\]
and 
\[\frac{(n_1+n_2-2)S_p^2}{\sigma^2} = \frac{1}{\sigma^2}\sum_{j=1}^{2}\sum_{i=1}^{n_j}(Y_{ji} - \overline{Y}_j)^2\sim \chi^2(n_1+n_2-2)\]

\section{Lecture 20}
\subsection{Two Gaussian Populations with Unequal Variance}
The pivotal quantity 
\[\frac{\overline{Y}_1 - \overline{Y}_2 - (\mu_1-\mu_2)}{\sqrt{\frac{\sigma_1^2}{n_1} + \frac{\sigma_2^2}{n_2}}}\sim G(0,1)\]

\end{document}

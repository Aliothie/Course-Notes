\documentclass[11pt]{article}
\usepackage{amsmath,amssymb,amsthm,enumerate,nicefrac,fancyhdr,hyperref,graphicx,adjustbox}
\hypersetup{colorlinks=true,urlcolor=blue,citecolor=blue,linkcolor=blue}
\usepackage[left=2.6cm, right=2.6cm, top=1.5cm, includehead, includefoot]{geometry}
\usepackage[dvipsnames]{xcolor}
\usepackage[d]{esvect}
\usepackage{pdfpages}

%% commands
%% useful macros [add to them as needed]
% sets
\newcommand{\C}{{\mathbb{C}}} 
\newcommand{\N}{{\mathbb{N}}}
\newcommand{\Q}{{\mathbb{Q}}}
\newcommand{\R}{{\mathbb{R}}}
\newcommand{\Z}{{\mathbb{Z}}}
\newcommand{\F}{{\mathbb{F}}}

% bases
\newcommand{\mA}{\mathcal{A}}
\newcommand{\mB}{\mathcal{B}}
\newcommand{\mC}{\mathcal{C}}
\newcommand{\mD}{\mathcal{D}}
\newcommand{\mE}{\mathcal{E}}
\newcommand{\mL}{\mathcal{L}}
\newcommand{\mM}{\mathcal{M}}
\newcommand{\mO}{\mathcal{O}}
\newcommand{\mP}{\mathcal{P}}
\newcommand{\mS}{\mathcal{S}}
\newcommand{\mT}{\mathcal{T}}

% linear algebra
\newcommand{\diag}{\operatorname{diag}}
\newcommand{\adj}{\operatorname{adj}}
\newcommand{\rank}{\operatorname{rank}}
\newcommand{\spn}{\operatorname{Span}}
\newcommand{\proj}{\operatorname{proj}}
\newcommand{\prp}{\operatorname{perp}}
\newcommand{\refl}{\operatorname{refl}}
\newcommand{\tr}{\operatorname{tr}}
\newcommand{\nul}{\operatorname{Null}}
\newcommand{\nully}{\operatorname{nullity}}
\newcommand{\range}{\operatorname{Range}}
\renewcommand{\ker}{\operatorname{Ker}}
\newcommand{\col}{\operatorname{Col}}
\newcommand{\row}{\operatorname{Row}}
\newcommand{\cof}{\operatorname{cof}}
\newcommand{\Num}{\operatorname{Num}}
\newcommand{\Id}{\operatorname{Id}}
\newcommand{\ipb}{\langle \thinspace, \rangle}
\newcommand{\ip}[2]{\left\langle #1, #2\right\rangle} % inner products
\newcommand{\M}[2]{M_{#1\times #2}(\F)}
\newcommand{\RREF}{\operatorname{RREF}}
\newcommand{\cv}[1]{\begin{bmatrix} #1 \end{bmatrix}}
\newenvironment{amatrix}[1]{\left[\begin{array}{@{}*{\numexpr#1-1}{c}|c@{}}}{\end{array}\right]}
\newcommand{\am}[2]{\begin{amatrix}{#1} #2 \end{amatrix}}

% vectors
\newcommand{\vzero}{\vv{0}}
\newcommand{\va}{\vv{a}}
\newcommand{\vb}{\vv{b}}
\newcommand{\vc}{\vv{c}}
\newcommand{\vd}{\vv{d}}
\newcommand{\ve}{\vv{e}}
\newcommand{\vf}{\vv{f}}
\newcommand{\vg}{\vv{g}}
\newcommand{\vh}{\vv{h}}
\newcommand{\vl}{\vv{\ell}}
\newcommand{\vm}{\vv{m}}
\newcommand{\vn}{\vv{n}}
\newcommand{\vp}{\vv{p}}
\newcommand{\vq}{\vv{q}}
\newcommand{\vr}{\vv{r}}
\newcommand{\vs}{\vv{s}}
\newcommand{\vt}{\vv{t}}
\newcommand{\vu}{\vv{u}}
\newcommand{\vvv}{{\vv{v}}}
\newcommand{\vw}{\vv{w}}
\newcommand{\vx}{\vv{x}}
\newcommand{\vy}{\vv{y}}
\newcommand{\vz}{\vv{z}}

% display
\newcommand{\ds}{\displaystyle}
\newcommand{\qand}{\quad\text{and}}
\newcommand{\qandq}{\quad\text{and}\quad}
\newcommand{\hint}{\textbf{Hint: }}

% misc
\newcommand{\area}{\operatorname{area}}
\newcommand{\vol}{\operatorname{vol}}
\newcommand{\red}[1]{{\color{red} #1}}
\newcommand{\rc}{\red{\checkmark}}

\newcommand{\vDashv}{%
  \mathrel{%
    \text{%
      \ooalign{$\vDash$\cr\reflectbox{$\vDash$}\cr}%
    }%
  }%
}

\newcommand{\vdashv}{%
  \mathrel{%
    \text{%
      \ooalign{$\vdash$\cr\reflectbox{$\vdash$}\cr}%
    }%
  }%
}

%% header
\pagestyle{fancy}
\fancyhead[L]{\bf\large CO 250 \\ Notes}
\fancyhead[R]{\bf\large Fall 2023 \\}
%\fancyfoot[C]{Page \thepage\ of 2}
\setlength{\headheight}{35pt}

\title{CO 250 Notes / Definition}
\author{Thomas Liu}
\begin{document}
\maketitle
\tableofcontents
\newpage

\section{Module 1: Modelling Optimization Problems}
\subsection{Week 1 Linear Programs}
\subsubsection{Definition: affine/linear}
A function $f: \R^n\rightarrow\R$ is affine if $f(x) = a^Tx+\beta$ for $a\in\R^n$, $\beta\in\R$. It is linear if, in addition, $\beta = 0$
\subsubsection{Definition: linear program}
The optimization problem \[min\{f(x):\ g_i(x)\leq b_i,\ \forall 1\leq i\leq m,\ x\in\R^n\}\]
is called a linear program if $f$ is affine and $g_1,\cdots, g_m$ is finite number of linear functions 
\subsubsection*{Comment}
\begin{itemize}
    \item insted of set notation, we often write LPS more verbosely
    \item Often give non-negativity constraints seperately
    \item May use max instead of min
    \item Sometimes replace subject to by s.t.
    \item Often write $x\geq 0$ as a short form for all variables are non-negative
    \item Dividing by variables is not allowed
    \item No strict inequalities 
    \item Must have finite number of constraints 
\end{itemize}
\begin{align*}
    \text{max}\ \ \ \ &-2x_1 + 7x_3 \\
    \text{subject to}\ \ \ \ &x_1+7x_2\leq 3 \\
    &3x_2+4x_3\leq 2 \\
    &x_1, x_3\geq 0
\end{align*}
\subsection{Week 2: Integer Programs / Optimization on Graphs}
\subsubsection{Integer Program}
An integer program is a linear program with added integrality constraints for some / all of the variables 
\subsubsection*{Comments}
\begin{itemize}
  \item Call an IP mixed if there are integer and fractional variables, and pure otherwise 
  \item The difference between LPs and IPs is subtle (LPs are easy to solve, IPs are not)
\end{itemize}
\subsubsection{Running Time}
The running time of an algorithm is the number of steps that an algorithm takes 
\subsubsection*{Comments}
\begin{itemize}
  \item It is stated as a function of $n$: $f(n)$ measures the largest number of steps an algorithm takes on an instance of size $n$
  \item Algoithm is efficient if its running time, $f(n)$, is a polynomial of $n$
\end{itemize}
\subsubsection{Binary Variable}
It is very useful for modeling logical constraints of the form: \[\text{Condition (A or B) and C}\rightarrow D\]
Variables that can take only a value of $0$ or $1$ are called binary 
\subsubsection{Definition: Matching}
A collection $M\subseteq E$ is a matching if no two edges $ij, i'j'\in M\ (ij\neq i'j')$ share an endpoint, $\{i,j\}\cap\{i',j'\}=\emptyset$
\subsubsection{Definition: Perfect}
A matching $M$ is perfect if every vertex $v$ in the graph is incident to an edge in $M$
\subsubsection{Definition: Perfect Matching}
Given $G=(V,E)$, $M\subseteq E$ is a perfect matching iff $M\cap\delta(v)$ contains a single edge for all $v\in V$ \\
$V = $ vertices, $E = $ edges
\subsection{Week 3: Shortest Paths and Nonlinear Models}
\subsubsection{Definition: $\delta(S)$}
For $S\subseteq V$, we let $\delta(S)$ be the set of edges with exactly one endpoint in $S$ \[\delta(S)=\{uv\in E:\ u\in S, v\notin S\}\]
\subsubsection{Definition: s,t-cut}
$\delta(S)$ is an s,t-cut if $s\in S$ and $t\notin S$
\subsubsection{IP for Shortest Paths}
Variable: we have one binary variabl $x_e$ for each edge $e\in E$, want 
\begin{equation*}
  x_e=
  \begin{cases}
    1: e\in P \\
    0: otherwise
  \end{cases}
\end{equation*}
Constraints: we have one constraint for each s,t-cut $\delta(U)$, forcing $P$ to have an edge from $\delta(S)$
\[\sum(x_e: e\in\delta(U))\geq 1\]
for all s,t-cuts $\delta(U)$ \\
Objective: $\displaystyle\sum(c_ex_e: e\in E)$
\subsubsection*{Note}
\begin{itemize}
  \item P is an s,t-path if it is of the form \[v_1v_2, v_2v_3, \cdots, v_{k-1}v_k\] and 
  \begin{itemize}
    \item $v_iv_{i+1}\in E$ for all $i\in\{1,\cdots,k-1\}$
    \item $v_i\neq v_j$ for all $i\neq j$
    \item $v_1=s$ and $v_k=t$
  \end{itemize}
  \item If $P$ is an s,t-path and $\delta(S)$ is an s,t-cut, then $P$ must have an edge from $\delta(S)$
  \item If $S\subseteq E$ contains at least one edge from every s,t-cut, then $S$ contains an s,t-path 
\end{itemize}
\subsubsection{Nonlinear Program}
A nonlinear program (NLP) is of the form 
\begin{align*}
  \text{min} &f(x) \\
  \text{s.t.}&g_1(x)\leq0\\
             &g_2(x)\leq0\\
             &\cdots\\
             &g_m(x)\leq0
\end{align*}
where 
\begin{itemize}
  \item $x\in\R^n$
  \item $f:\R^n\rightarrow \R$, and 
  \item $g_i:\R^n\rightarrow \R$
\end{itemize}
Linear programs are NLPs

\section{Module 2: Solving Linear Programs}
\subsection{Week 4: Possible Outcomes / Certificates / Standard Equality Forms}
\subsubsection{Feasible Solution}
An aassignment of values to each of the variables is a feasible solution if all the constraints are satisfied
\subsubsection{Feasible / Infeasible}
An optimization problem is feasible if it has at least one feasible solution. It is infeasible otherwise 
\subsubsection{Optimal Solution}
\begin{itemize}
  \item For a maximization problem, an optimal solution is a feasible solution that maximizes the objective function 
  \item For a minimization problem, an optimal solution is a feasible solution that minimizes the objective function 
\end{itemize}
\subsubsection{Unbounded}
\begin{itemize}
  \item A maximization problem is unbounded if for every value $M$, there exists a feasible solution with objective value greater than $M$
  \item A minimization problem is unbounded if for every value $M$, there exists a feasible solution with objective value smaller than $M$
\end{itemize}
\subsubsection{Fundamental Theorem of Linear Programming}
For any linear program, EXACTLY ONE of the following holds
\begin{itemize}
  \item It has an optimal solution 
  \item It is infeasible 
  \item It is unbounded 
\end{itemize}
\subsubsection{Proposition}
There is no solution to $Ax=b$, $x\geq0$, if there exists $y$ where 
\[y^TA\geq0^T\ \ \ \text{and}\ \ \ y^Tb<0\]
\subsubsection{Farka's Lemma}
If there is no solution to $Ax=b$, $x\geq0$, then there exists $y$ where 
\[y^TA\geq0^T\ \ \ \text{and}\ \ \ y^Tb<0\]
\subsubsection{Claims to show unboundness}
\begin{itemize}
  \item Claim 1: $x(t)$ is feasible for all $t>0$
  \item Claim 2: $z\rightarrow\infty$ when $t\rightarrow\infty$
\end{itemize}
\subsubsection{Proposition}
The linear program \[\text{max}\{c^Tx:Ax=b, x\geq 0\}\]
is unbounded if we can find $\overline{x}$ and $r$ such that \[\overline{x}\geq0, r\geq0, A\overline{x}=b, Ar=0, \text{ and }c^Tr>0\]
\subsubsection{Definition: Standard Equality Form (SEF)}
A linear program (LP) is in standard equality form (SEF) if 
\begin{enumerate}
  \item it is a maximization problem
  \item for every variable $x_j$, we have the constraint $x_j\geq 0$
  \item all other constraints are equality constraints
\end{enumerate}
\subsubsection{What to do if LP is not in SEF?}
\begin{enumerate}
  \item find an ``equivalent'' LP in SEF 
  \item solve the ``equivalent'' LP using simplex
  \item use the solution of ``equivalent'' LP to get the solution of the original LP
\end{enumerate}
\subsubsection{Definition: Equivalent LP}
A pair of LPs are equivalent if they behave in the same way \\
Linear programs (P) and (Q) are equivalent if 
\begin{itemize}
  \item (P) infeasible $\iff$ (Q) infeasible 
  \item (P) unbounded $\iff$ (Q) unbounded 
  \item can construct optimal solution of (P) from optimal solution of (Q)
  \item can construct optimal solution of (Q) from optimal solution of (P)
\end{itemize}
\subsubsection{Theorem}
Every LP is equivalent to an LP in SEF 
\subsubsection*{Tips}
Set free variable $x_3:=a-b$ where $a,b\geq0$
\subsection{Week 5: Basis, Canonical Form, Formalizing the Simplex}
\subsubsection{Notation}
Let $B$ be a subset of column indices \\
Then $A_B$ is a solumn sub-matrix of $A$ indexed by set $B$ \\
$A_j$ denotes column $j$ of $A$
\subsubsection{Definition: Basis}
Let $B$ be a subset of column indices. $B$ is a basis if
\begin{enumerate}
  \item $A_B$ is a square matrix 
  \item $A_B$ is non-singular (columns are independent)
\end{enumerate}
\subsubsection{Theorem}
Max number of independent columns $=$ max number of independent rows 
\subsubsection{Definition: Basic solution}
$x$ is a basic solution for basis $B$ if 
\begin{enumerate}
  \item $Ax=b$
  \item $x_j=0$ whenever $j\notin B$
\end{enumerate}
\subsubsection{Proposition}
Consider $Ax=b$ and a basis of $A$. Then there exists a unique basic solution $x$ for $B$
\subsubsection{Definition: Basic solution}
Consider $Ax=b$ with independent rows. Vector $x$ is a basic solution if it si a basic solution for some basis $B$
\subsubsection{Definition}
A basic solution $x$ of $Ax=b$ is feasible if $x\geq0$
\subsubsection{Definition: Canonical form}
\[\text{max}\{c^Tx:Ax=b,x\geq0\}\]
Let $B$ be a basis of $A$. Then (P) is in canonical form for $B$ if 
\begin{enumerate}[P1]
  \item $A_b = I$
  \item $c_j=0$ for all $j\in B$
\end{enumerate}
\subsubsection{Proposition}
For any basis $B$, there exists (P') in canonical form for $B$ such that 
\begin{enumerate}
  \item (P) and (P') have the same feasible region
  \item feasible solutions have the same objective value for (P) and (P') 
\end{enumerate}
\subsubsection{Proposition}
Consider $A$ with basis $B$,
$P:$\\
\begin{align*}
  \text{max}\ \ &c^Tx \\
  \text{s.t.}\ \ &Ax=b \\
                 &x\geq\vec{0}
\end{align*}
$P':$\\
\begin{align*}
  \text{max}\ \ &[c^T-y^TA]x + y^Tb \\
  \text{s.t.}\ \ &A^{-1}_BAx = A^{-1}_Bb \\
                 &x\geq\vec{0}
\end{align*}
where $y=A^{-T}_B$ 
\begin{enumerate}
  \item $(P')$ is in canonical form for basis $B$, $\overline{c}_B = 0$ and $A'_B = I$
  \item $(P)$ and $(P')$ have the same feasible region 
  \item feasible solutions have the same objective value for $(P)$ and $(P')$
\end{enumerate}
\subsubsection{The Simplex Algorithm}
\begin{align*}
  \text{max}\ \ &c^Tx \\
  \text{s.t.}\ \ &Ax=b \\
                 &x\geq0
\end{align*}
Input: a feasible basis $B$ \\
Output: an optimal solution or it detects that the LP is unbounded 
\begin{enumerate}
  \item rewrite in canonical form for basis $B$
  \item find a better basis $B$ or get required outcome 
\end{enumerate}
\subsubsection*{Try to find a better basis}
\begin{align*}
  \text{max}\ \ &z=c^T_Nx_N + \overline{z} \\
  \text{s.t.}\ \ &x_B + A_Nx_N = b \\
                 &x\geq0
\end{align*}
\subsubsection{Proposition}
Simplex tells the truth:
\begin{itemize}
  \item If it claims the LP is unbounded, it is unbounded 
  \item if it claims the solution is optimal, it is optimal 
\end{itemize}
\subsubsection{Definition: Bland's Rule}
Bland's rule is as follows:
\begin{itemize}
  \item if we have a choice for the element entering the basis, pick the smallest one
  \item if we have a choice for the element leaving the basis, pick the smallest one 
\end{itemize}
\subsubsection{Theorem}
If we use Bland's rule, then the Simplex algorithm always terminates 
\subsection{Week 6: Feasible Solution, Half Spaces and Convexity, Extreme Points}
\subsubsection{Feasible Solution}
\[\text{max}\ \ \{c^T:Ax=b,x\geq0\}\]
Algorithm 1:
\begin{itemize}
  \item[Input:] $A,b,c$ and a feasible solution 
  \item[Output:] Optimal solution / detect LP unbounded 
\end{itemize}
Algorithm 2:
\begin{itemize}
  \item[Input:] $A$ and $b$
  \item[Output:] Feasible solution / detect there is none 
\end{itemize}
\subsubsection*{Proposition}
Can use Algorithm 1 to get Algorithm 2
\subsubsection{Definition: Feasible Region}
For an optimication problem, the \[\text{feasible region } = \text{ set of all feasible solutions}\]
\subsubsection{Definition: Polyhedron}
$P\subseteq\R^n$ is a polyhedron if there exists a matrix $A$ and a vector $b$ such that \[P=\{x:Ax\leq b\}\]
\subsubsection{Proposition}
The feasible region of an LP is a polyhedron
\subsubsection{Definition: Hyperplane, Halfspace}
Let $a\neq0$ be a vector and $\beta$ a read number 
\begin{enumerate}
  \item $\{x:a^Tx=\beta\}$ is a hyperplane 
  \item $\{x:a^Tx\leq\beta\}$ is a halfspace 
\end{enumerate}
A hyperplane is the set of solutions to a single linear equation \\
A halfspace is the set of solutions to a single linear inequality 
\subsubsection{Definition: Translate}
Let $S,S'\subseteq\R^n$. Then $S'$ is a translate of $S$ if there exists $p\in\R^n$ and \[S'=\{s+p:s\in S\}\]
\subsubsection{Proposition}
Let $a\neq0$ be a vector and $\beta$ a real number and let 
\[H:=\{x:a^Tx=\beta\} and\ H_0:=\{x:a^Tx=0\}\]
It follows that $H$ is a translate of $H_0$
\subsubsection{Proposition}
Let $a\neq0$ be a vector and $\beta$ a real number and let 
\[F:=\{x:a^Tx\leq\beta\}and\ F_0:=\{x:a^Tx\leq0\}\]
It follows that $F$ is a translate of $F_0$
\subsubsection{Proposition}
The dimension of a hyperplane in $\R^n$ is $n-1$
\subsubsection{Definition: Line Through}
Let $x_1,x_2\in\R^n$. The line through $x_1$ and $x_2$ is defined as \[L=\{x=\lambda x_1 + (1-\lambda)x_2:\lambda\in\R\}\]
\subsubsection{Definition: Line Segement}
Let $x_1,x_2\in\R^n$. The line segement $x_1$ and $x_2$ is defined as \[S=\{x=\lambda x_1 + (1-\lambda)x_2:\lambda\in\R,0\leq\lambda\leq1\}\]
\subsubsection{Definition: Convex}
A set $S\subseteq\R^n$ is convex if, for any pair of points $x_1, x_2\in S$, the line segement between $x_1$ and $x_2$ is in $S$
\subsubsection{Proposition}
Let $H=\{x:a^Tx\leq\beta\}$ be a halfspace. It follows that $H$ is convex 
\subsubsection{Corollary}
If $P$ is a polyhedron, then $P$ is convex 
\subsubsection{Definition: Properly Contained}
Point $x\in\R^n$ is properly contained in the line segement $L$ if 
\begin{itemize}
  \item $x\in L$
  \item $x$ is distinct from the endpoints of $L$
\end{itemize}
\subsubsection{Definition: Extreme Point}
Let $S$ be a convex set and $\overline{x}\in S$ \\
It follows that $\overline{x}$ is NOT an exteme point if there exists a line segement $L\subseteq S$ where $L$ properly contains $\overline{x}$
\subsubsection*{Remark}
A convex set may have an infinite number of extreme points \\
A convex set may have NO extreme points 
\subsubsection{Definition: Tight}
Let $P=\{x:Ax\leq b\}$ be a polyhedron and let $x\in P$
\begin{itemize}
  \item A constraint is tight for $x$ if it is satisfied with equality 
  \item the set of all tight constraints is denoted $\overline{A}x\leq\overline{b}$
\end{itemize}
\subsubsection{Theorem}
Let $P=\{x\in\R^n:Ax\leq b\}$ be a polyhedron and let $\overline{x}\in P$
\begin{itemize}
  \item if $\rank(\overline{A})=n$, then $\overline{x}$ is an extreme point 
  \item if $\rank(\overline{A})<n$, then $\overline{x}$ is NOT an extreme point 
\end{itemize}

\section{Module 3: Duality Through Examples}
\subsection{Week 7: Duality Through Examples / Weak Duality}
\subsubsection{Definition}
A width assignment $\{y_U:\delta(U)\ s,t-cut\}$ is feasible if, for every edge $e\in E$, the total width of all cuts containing $e$ is no more than $c_e$
\subsubsection{Proposition}
If $y$ is a feasible width assignment, then any s,t-path must have length at least \[\sum(y_U:U\ s,t-cut)\]
\subsubsection{Theorem: Weak Duality}
If $\overline{x}$ is feasible for $(P)$ and $\overline{y}$ is feasible for $(D)$, then $b^T\overline{y}\leq c^T\overline{x}$
\subsubsection{Theorem: Strong Duality}
Let $P$ and $D$ be a pair of primal-dual LPs. Then 
\begin{itemize}
  \item if there exists an optimal solution $\overline{x}$ of $P$, then there exists an optimal solution $\overline{y}$ of $D$
  \item the value of $\overline{x}$ in $P$ equals the value of $\overline{y}$ in $D$
\end{itemize}
\subsection{Week 8: Shortest Path Algorithm and Correctness}
\subsubsection{Definition: Slack}
Let $y$ be a feasible dual solution. The slack of an edge $e\in E$ is defined as 
\[\text{slack}_y(e) = c_e-\sum(y_U: \delta(U)\ s,t-\text{cut}, e\in\delta(U))\]
\subsubsection{Shortest Path Algorithm}
\textbf{Input}: Graph $G = (V,E)$, costs $c_e\geq0$ for all $e\in E$, $s,t\in V$ where $s\neq t$\\ 
\textbf{Output}: A shortest st-path $P$
\begin{enumerate}
  \item $y_W:=0$ for all st-cuts $\delta(W)$. Set $U:=\{s\}$
  \item while $t\notin U$ do 
  \item $\ \ $ Let $ab$ be an edge in $\delta(U)$ of smallest slack for $y$ where $a\in U$, $b\notin U$
  \item $\ \ $ $y_U:=\text{slack}_y(ab)$
  \item $\ \ $ $U:= U\cup\{b\}$
  \item $\ \ $ change edge $ab$ into an arc $\overrightarrow{ab}$
  \item end while 
  \item return A directed st-path $P$
\end{enumerate}
\subsubsection{Proposition}
Let $y$ be a feasible dual solution, and $P$ and s,t-path. $P$ is a shortest if 
\begin{itemize}
  \item all edges on $P$ are equality edges, and 
  \item every active cut $\delta(U)$ has exactly one edge of $P$
\end{itemize}
\subsubsection{Proposition: Correctness of the Shortest Path Algorithm}
The Shortest Path Algorithm maintains throughout its execution if:
\begin{itemize}
  \item $y$ is a feasible dual 
  \item arcs are equality arcs (have $0$ slack)
  \item no active cut $\delta(U)$ has an entering arc: an arc $wu$ with $w\notin U$, and $u\in U$
  \item for every $u\in U$ there is a directed s,u-path, and
  \item arcs have both ends in $U$
\end{itemize}

\section{Module 4: Duality Theory}
\subsection{Week 9: Duality Theory}
\subsubsection{Primal-Dual Pairs}
The following table shows constraints and variables in primal and dual LPs correspond \\
\includegraphics[width=\textwidth]{Screeshots/primal-dual pairs.png}
\subsubsection{Theorem: Weak Duality}
Let ($P_{max}$) and ($P_{min}$) represent the above. If $\overline{x}$ and $\overline{y}$ are feasible for two LPs, then \[c^T\overline{x}\leq b^T\overline{y}\]
If $c^T\overline{x} = b^T\overline{y}$, then $\overline{x}$ is optimal for $(P_{max})$ and $\overline{y}$ is optimal for $(P_{min})$
\subsubsection{Consequence of Weak Duality}
\begin{itemize}
  \item $(P_{max})$ is unbounded $\rightarrow$ $(P_{min})$ infeasible 
  \item $(P_{min})$ is unbounded $\rightarrow$ $(P_{max})$ infeasible 
  \item $(P_{max})$ and $(P_{min})$ feasible $\rightarrow$ both must have optimal solutions 
\end{itemize}
\subsubsection{Theorem: Strong Duality}
If $(P_{max})$ has an optimal solution $\overline{x}$, then $(P_{min})$ has an optimal solution $\overline{y}$
such that $c^T\overline{x} = b^T\overline{y}$ \\
Let $(P)$ and $(D)$ be a primal-dual pairs of LPs. If $(P)$ has an optimal solution, then $(D)$ has one, and their objective values equal 
\subsubsection{Strong Duality Theorem - Feasibility Version}
Let $(P)$ and $(D)$ be primal-dual pairs of LPs. If both are feasible, then both optimal solutions of the same objective value 
\subsubsection{Complementary Slackness - special case}
Let $\overline{x}$ and $\overline{y}$ be feasible for $(P)$ and $(D)$ \\
Then $\overline{x}$ and $\overline{y}$ are optimal if and only if 
\begin{itemize}
  \item $\overline{y}_i = 0$, or 
  \item the $i$th constraint of $(P)$ is tight for $\overline{x}$, for every row index $i$
\end{itemize}
\subsubsection{Complementary Slackness Theorem}
Let $(P)$ and $(D)$ be an arbitrary primal-dual pair of LPs, and let $\overline{x}$ and $\overline{y}$ be feasible solutions. Then
these solutions are optimal if and only if the CS conditions hold 
\subsubsection{Definition}
Let $a^{(1)}, \cdots, a^{(k)}$ be vectors in $\R^n$. The cone generated by these vectors is given by 
\[C=\{\lambda_1a^{(1)} + \lambda_2a^{(2)} + \cdots + \lambda_ka^{(k)}: \lambda\geq 0\}\]
\subsubsection{Theorem}
Let $\overline{x}$ be a feasible solution to 
\[\text{max}\{c^Tx: Ax\leq b\}\]
Then $\overline{x}$ is optimal if and only if $c$ is in the cone of tight constraints for $\overline{x}$

\section{Module 5: Solving Integer Programs}
\subsection{Week 10: Solving Integer Programs}
\subsubsection{Definition: Convex Hull}
Let $C$ be a subset of $\R^n$ \\
The convex hull of $C$ is the smallest convex set that contains $C$
\subsubsection{Meyer's Theorem}
Consider $P=\{x:Ax\leq b\}$ where $A,b$ are rational \\
Then, the convex hull of all integer poitns in $P$ is a polyhedron 
\subsubsection{Theorem}
\begin{itemize}
  \item IP is infeasible if and only if LP is infeasible 
  \item IP is unbounded if and only if LP is unbounded 
  \item an optimal solution to IP is an optimal solution to LP 
  \item an extreme optimal solution to LP is an optimal solution to IP
\end{itemize}
\subsubsection{Cutting Plane Scheme}
\[\text{max}\ \{c^Tx:Ax\leq b, x\text{integer}\} \ \ \ (IP)\]
\begin{itemize}
  \item Let $(P)$ denote $\text{max}\{c^Tx: Ax\leq b\}$
  \item If $(P)$ is infeasible, then STOP. $(IP)$ is also infeasible 
  \item Let $\overline{x}$ be the optimal solution to $(P)$ 
  \item If $\overline{x}$ is integral, then STOP. $\overline{x}$ is also optimal for $(IP)$
  \item Find a cutting plane $a^Tx\leq\beta$ for $\overline{x}$
  \item Add a constraint $a^Tx\leq\beta$ to the system $Ax\leq b$
  \item Repeat 
\end{itemize}
\subsubsection{Definition: Floor}
Let $a\in\R$, then $\lfloor a\rfloor$ denotes the largest integer $\leq a$

\section{Module 6: Nonlinear Optimization}
\subsection{Week 11: Nonlinear Optimization}
\subsubsection{Definition: Nonlinear Program}
A nonlinear program (NLP) is a problem of the form 
\begin{align*}
  \text{min}\ \ \ &f(x) \\
  \text{s.t.}\ \ \ &g_i(x)\leq 0\quad (i=1,\cdots,k)
\end{align*}
\subsubsection{Definition: Local Optimum}
Consider \[min\{f(x):x\in S\} \ \ \quad (P)\]
$s\in S$ is a local optimum if there exists $\delta>0$ such that 
\[\forall x'\in S\ \text{where}\ ||x'-x||\leq \delta\ \text{and we have}\ f(x)\leq f(x')\]
\subsubsection{Proposition}
Consider \[min\{c^Tx:x\in S\}\ \ \quad(p)\]
If $S$ is convex and $x$ is a local optimum, then $x$ is optimal
\subsubsection{Proposition}
If $g_1,\cdots,g_k$ are all convex, then the feasible region of $(P)$ is convex
\subsubsection{Definition}
Function $f:\R^n\rightarrow\R$ is convex if for all $a,b\in\R^n$
\[f(\lambda a+(1-\lambda)b)\leq \lambda f(a) + (1-\lambda)f(b)\ \text{for all}\ 0\leq\lambda\leq1\]
\subsubsection{Proposition}
Let $g:\R^n\rightarrow\R$ be a convex function $\beta\in\R$ \\
It follows that $S=\{x\in\R^n:g(x)\leq\beta\}$ is a convex set 
\subsubsection{Proposition}
$(P)$:
\begin{align*}
  \text{min}\ \ &c^Tx \ \quad\\
  \text{s.t.}\ \ &g_i(x)\leq 0\ \ (i=1,\cdots,k)
\end{align*}
If all functions $g_i$ are convex, then the feasible region of $(P)$ is convex 
\subsubsection{Definition: Epigraph}
Let $f:\R^n\rightarrow\R$ be a function. The epigraph of $f$ is then given by 
\[epi(f) = \{\cv{y\\x}:y\geq f(x), x\in\R\}\subseteq\R^{n+1}\]
\subsubsection{Proposition}
Let $f:\R^n\rightarrow\R$ be a function. It follows that 
\begin{enumerate}
  \item $f$ is convex $\Rightarrow$ $epi(f)$ is convex 
  \item $epi(f)$ is convex $\Rightarrow$ $f$ is convex 
\end{enumerate}
\subsubsection{Defintion: Subgradient}
Let $f:\R^n\rightarrow\R$ be a convex function and $\overline{x}\in\R^n$ \\
Then, $s\in\R^n$ is a subgradient of $f$ at $\overline{x}$ if 
\[h(x):= f(\overline{x}) + s^T(x-\overline{x})\leq f(x)\ \text{for all }x\in\R^n\]
\subsubsection{Definition: Supporting}
Let $C\in\R^n$ be a convex set and let $\overline{x}\in C$\\
The halfspace $F=\{x:s^Tx\leq\beta\}$ is supporting $C$ at $\overline{x}$ if 
\begin{enumerate}
  \item $C\subseteq F$ and 
  \item $s^T\overline{x}=\beta$. That is, $\overline{x}$ is on the boundary of $F$
\end{enumerate}
\subsubsection{Proposition}
Let $g:\R^n\rightarrow\R$ be convex and let $\overline{x}$ where $g(\overline{x}) = 0$ \\
Let $s$ be a subgradient of $g$ at $\overline{x}$ \\
Let $C=\{x:g(x)\leq 0\}$ \\
Let $F=\{x:h(x):=g(\overline{x})+s^T(x-\overline{x})\leq0\}$ \\
Then, $F$ is a supporting halfspace of $C$ at $\overline{x}$
\subsubsection{Proposition}
\begin{align*}
  \text{min}\ \ &c^Tx \\
  \text{s.t.}\ \ &g_i(x)\leq 0 \ \ (i=1,\cdots,k)
\end{align*}
$g_1,\cdots, g_k$ all convex \\
$\overline{x}$ is a feasible solution \\
$\forall i\in I$, $g_i(\overline{x}) = 0$ \\
$\forall i\in I$, $s^i$ subgradient for $g_i$ at $\overline{x}$ \\
If $-c\in\text{cone}\{s^i:i\in I\}$ then $\overline{x}$ is optimal 
\subsubsection{Proposition}
Let $f:\R^n\rightarrow\R$ be a convex function and let $\overline{x}\in\R^n$ \\
If the gradient $\nabla f(\overline{x})$ of $f$ exists at $\overline{x}$, then it is a subgradient 
\subsubsection{Proposition}
Let $f:\R^n\rightarrow\R$ be function and let $\overline{x}\in\R^n$ \\
If the partial derivative $\dfrac{\partial f(x)}{\partial x_j}$ exists for $f$ at $\overline{x}$ for all $j=1,\cdots,n$, then 
the gradient $\nabla f(\overline{x})$ is obtained by evaluating for $\overline{x}$ 
\[\cv{\dfrac{\partial f(x)}{\partial x_1} &\cdots &\dfrac{\partial f(x)}{\partial x_n}}^T\]
\subsubsection{Definition: Slater Point}
A feasible solution to $\overline{x}$ is a slater point of 
\begin{align*}
  \text{min}\ \ &c^Tx \\
  \text{s.t.}\ \ &g_i(x)\leq 0 \ \ (i=1,\cdots,k)
\end{align*}
if $g_i(\overline{x})<0$ for all $i=1,\cdots,k$ 
\subsubsection{The Karush-Kuhn-Tucker (KKT) Theorem}
Consider the following NLP:
\begin{align*}
  \text{min}\ \ &c^Tx \\
  \text{s.t.}\ \ &g_i(x)\leq 0 \ \ (i=1,\cdots,k)
\end{align*}
Suppose that 
\begin{enumerate}
  \item $g_1,\cdots,g_k$ are all convex 
  \item there exists a slater point 
  \item $\overline{x}$ is a feasible solution 
  \item $I$ is the set of indices $i$ for which $g_i(\overline{x}) = 0$
  \item for all $i\in I$ there exists a gradient $\nabla g_i(\overline{x})$ of $g_i$ at $\overline{x}$
\end{enumerate}
Then $\overline{x}$ is optimal $\iff$ $-c\in cone\{\nabla g_i(\overline{x}):i\in I\}$

\end{document}
\documentclass[12pt, letterpaper]{article}
\usepackage{amsmath,amssymb,amsthm,enumerate,nicefrac,fancyhdr,graphicx,adjustbox,titlesec}
\title{Math 135 Notes}
\author{Thomas Liu}
\begin{document}
\maketitle
\tableofcontents

\newpage

% Chapter 1
\section{Chapter 1 Introduction to the Language of Mathematics}
\subsection{Introducing Sets}
Definition: a set is a well-defined unordered collection of distinct elements. can write down a set by listing its member \\
Example:
\begin{itemize}
    \item $\{\pi, *, 7, \&, \%\} = \{\&, \%, \pi, *, 7\}$
    \item $\{\pi, 7, \pi\}$ is not a set 
    \item $\{\pi, \{*, \pi\}\}$ is a set with 2 elements
    \item $* \in \{\pi, *, 7\}$
    \item $135 \notin \{\pi, *, 7\}$
    \item $\{\} = \emptyset$
    \item $\emptyset \neq \{\emptyset\}$
    \item $\emptyset \notin \emptyset$
    \item $\{7\} \notin \{\pi, 7, *\}$
\end{itemize}
\subsection{Familar Sets}
$\mathbb{Z}$ is a set of integer \\
$\mathbb{N}$ is a set of natural number \\
$\mathbb{Q}$ is a set of rational number ($\dfrac{p}{q}$, $p$ and $q$ are integers and $q$ is not zero) \\
$\mathbb{R}$ is a set of real number
\subsection{Statement}
Definition: a statement is a sentence that is true or false \\
An open sentence is a sentence that becomes a statement if values are assigned to all variables in sentence
\subsection{Negation}
Suppose $P$ is statement \\
The negation of $P$ is statement $\neg P$ which is true when $P$ is false and false when $P$ is true. $P$ and $\neg(\neg P)$ always have the same truth value
\subsection{Universally Quantified Statements}
$\forall x \in \mathbb{N}$, $x^2-x \geq 0$ \\
$\forall$ is quantifier/for all, $x$ is variable, $\mathbb{N}$ is domain, $x^2-x \geq 0$ is open sentence
\subsection{Existential Statements}
$\exists x \in S$, $P(x)$ \\
Example: $\exists x \in \mathbb{Z}$, $\dfrac{x-7}{2x+4} = 5$ \\
$\exists$ is there exists
\begin{itemize}
    \item $\forall$ is true for all $x$
    \item $\forall$ is false at least one $x$
    \item $\exists$ is true at least one $x$
    \item $\exists$ is false for all $x$
\end{itemize}
\subsection{Negating Quantifiers}
$\neg (\forall x \in S, P(x)) = \exists x \in S, (\neg P(x))$ \\
$\neg (\exists x \in S, P(x)) = \forall x \in S, (\neg P(x))$

% Chapter 2
\section{Chapter 2 Logical Analysis of Mathematical Statements}
\subsection{Logic}
Given a statement (variable), we can build more complex logical expressions using logical operators \\
The truth value of logical expression can be defined using truth table 
\begin{center}
\begin{tabular}{|l|l|}
    A   &$\neg A$ \\
    T   &F        \\
    F   &T        \\
\end{tabular}
\end{center}
\subsection{And}
The definition of $A$ and $B$, $A \wedge B$ is 
\begin{center}
\begin{tabular}{|l|l|l|}
    A  &B  &$A \wedge B$ \\
    T  &T  &T \\
    F  &T  &F \\
    T  &F  &F \\
    F  &F  &F \\
\end{tabular}
\end{center}
$A \wedge B$ is only true when both $A$ and $B$ are true 
\subsection{Or}
The definition of $A$ or $B$, $A \vee B$ is 
\begin{center}
\begin{tabular}{|l|l|l|}
    A  &B  &$A \vee B$ \\
    T  &T  &T \\
    F  &T  &T \\
    T  &F  &T \\
    F  &F  &F \\
\end{tabular}
\end{center}
$A \vee B$ is only false when both $A$ and $B$ are false 
\subsection{Logical Equivalence}
Two logical expression are logically equivalent if they have the same truth value of all choices 
of values for their opponent statement variables. Their truth value match in a truth table 
\subsection{De Morgan's Rule}
$\neg(A \vee B) \equiv (\neg A) \wedge (\neg B)$ \\
$\neg(A \wedge B) \equiv (\neg A) \vee (\neg B)$ \\
$A \wedge (B \vee C) \equiv (A \wedge B) \vee (A \wedge C)$
\subsection{Implication}
An implication is a sentence of the form "If $H$ then $C$" or $H \Rightarrow C$
\begin{itemize}
    \item logically equivalent to $(\neg H) \vee C$
    \item $H$ is hypothesis, $C$ is conclusion 
\end{itemize}
\begin{center}
\begin{tabular}{|l|l|l|}
    H  &C  &$H \Rightarrow C$ \\
    T  &T  &T \\
    T  &F  &F \\
    F  &T  &T \\
    F  &F  &T \\
\end{tabular}
\end{center}
\subsection{Negation of Implication}
The negation of $H \Rightarrow C$ is logically equivalent to $H \wedge (\neg C)$ \par 
$\neg(H \Rightarrow C) \equiv \neg((\neg H) \vee C) \equiv (\neg(\neg H)) \wedge (\neg C) \equiv H \wedge (\neg C)$
\subsection{Contrapositive}
\begin{itemize}
    \item Contrapositive of $A \Rightarrow B$ is implication $\neg B \Rightarrow \neg A$
    \item They are logically equivalent
\end{itemize}
\subsection{Converse}
\begin{itemize}
    \item Converse of $A \Rightarrow B$ is implication $B \Rightarrow A$
    \item not logically equivalent
\end{itemize}
\subsection{If and Only If}
$\iff$ read as "if and only if" / iff
\begin{center}
\begin{tabular}{|l|l|l|}
    A  &B  &$A \iff B$ \\
    T  &T  &T \\
    T  &F  &F \\
    F  &T  &F \\
    F  &F  &T 
\end{tabular}
\end{center}

% Chapter 3
\section{Chapter 3 Prove Mathematical Statements}
\subsection{Statement}
Example: For all real $x, y \in \mathbb{R}$, $x^4+x^2y+y^2 \geq 5x^2y-3y^2$
\subsection{Divisibility}
Definition: an integer $m$ divides an integer $n$ if there exists an integer $k$ so that $n = km$, write $m \mid n$
\subsection{Transitivity of Divisibility (TD)}
For $a,b,c \in \mathbb{Z}$, if $a \mid b$ and $b \mid c$, then $a \mid c$
\subsection{Divisibility of Integer Combinations (DIC)}
For all $a,b,c \in \mathbb{Z}$, if $a\mid b$ and $a\mid c$, then $a\mid (bx+cy)$ for all $x,y \in \mathbb{Z}$
\subsection{Proposition 8}
For $a,b,c \in \mathbb{Z}$, if $a\mid b$ or $a\mid c$, then $a\mid bc$
\subsubsection{Proof by Contradiction}
We prove that a statement $P$ is true by
\begin{itemize}
    \item assume $\neg P$ is true, then based on assumption
    \item prove both $Q$ and $\neg Q$ to prove statement $B$
\end{itemize}
\subsection{Uniqueness}
Two approaches \\
we can prove that there is a unique value satisfying some property by showing such a value exists and then 
\begin{itemize}
    \item assume it is satisfied by $x$ and $y$ and showing $x=y$
    \item use by contradiction
\end{itemize}

% Chapter 4
\section{Chapter 4 Mathematical Induction}
\subsection{Principle of Mathematical Induction (POMI)}
Let $P(n)$ is a statement that depends on $n \in \mathbb{N}$ \\
If statements 1 and 2 are both true 
\begin{enumerate}
    \item $P(1)$ is true 
    \item For all $k \in \mathbb{N}$, if $P(k)$, then $P(k+1)$
    \item Then, for all $n \in \mathbb{N}$, $P(n)$
\end{enumerate}
\subsection{Binomial Coefficients}
For non-negative integers $n$ and $m$, we define:
\begin{itemize}
    \item ${n \choose m} = \dfrac{m!}{(n-m)!m!}$ when $m \leq n$
    \item ${n \choose m} = 0$ when $m > n$
\end{itemize}
\subsection{Pascal's Identity (PI)}
For all positive integers $n$ and $m$ with $m < n$, we have \\
${n \choose m} = {{n-1} \choose {m-1}} + {{n-1} \choose m}$
\subsection{The Binomial Theorem}
\subsection{Binomial Theorem Version 1 (BT1)}
For all integer $n \geq 0$ and $x \in \mathbb{R}$
\begin{align*}
    (1+x)^n = \sum_{m=0}^{n} {n \choose m} x^m
\end{align*}
\subsection{Binomial Theorem Version 2 (BT2)}
For all integer $n \geq 0$ and $a,b \in \mathbb{R}$
\begin{align*}
    (a+b)^n = \sum_{m=0}^{n} {n \choose m} a^{n-m} b^m
\end{align*}
\subsection{Principle of Strong Induction (POSI)}
Let $P(n)$ be a statement that depends on $n \in \mathbb{N}$, if 
\begin{enumerate}
    \item $P(1)$ is true
    \item $\forall k \in \mathbb{N}$, $[(P(1)\wedge P(2)\wedge \cdots P(k)) \Rightarrow P(k+1)]$ \\
          then $P(n)$ is true for all $n \in \mathbb{N}$
\end{enumerate}

% Chapter 5
\section{Chapter 5 Set}
$\{x \in U : P(x) \}$ \\
$\{f(x) : P(x)\}$ \\
$\{f(x) : x \in U, P(x)\}$
\subsection{Set-difference}
The set-difference of two sets $S$ and $T$, written $S-T$ or $S \setminus T$ is the set of all elements belonging to $S$ but not $T$
\subsection{Set Complement}
The complement of a $S$, written $\overline{S}$, is the set of all elements in $U$ but not in $S$, $\overline{S} = U - S$
\subsection{Subset}
If $S$ and $T$ are sets, we say $S$ is a subset of $T$, write $S \subseteq T$ if every element of $S$ is an element of $T$

% Chapter 6
\section{Chapter 6 The Greatest Common Divisor}
\subsection{Bounds by Divisibility (BBD)}
For all $a,b \in \mathbb{Z}$, if $b \mid a$ and $a \neq 0$, then $b \leq |a|$
\subsection{Division Algorithm (DA)}
For all $a \in \mathbb{Z}$ and for all $b \in \mathbb{N}$ \\
There exists unique integers $q$ and $r$ such that $a = bq + r$ where $0 \leq r < b$
\subsection{GCD Formal Definition}
Let $a,b \in \mathbb{Z}$ \\
When $a$ and $b$ are not both zero, we say an integer $d>0$ is the Greatest Common Divisor of $a$ and $b$, and writes $\gcd(a,b)$ iff 
\begin{itemize}
    \item $d \mid a$, $d \mid b$, and 
    \item for all integers $c$, if $c \mid a$ and $c \mid b$ then $c \leq d$
\end{itemize}
Fact: \\
For all $a,b \in \mathbb{Z}$, $\gcd(3a+b, a) = \gcd(a,b)$
\subsection{GCD with Remainders (GCDWR)}
For all $a,b,q,r \in \mathbb{Z}$, if $a = bq+r$, then $\gcd(a,b) = \gcd(b,r)$
\subsection{Euclidean Algorithm (EA)}
Process to compute $\gcd(a,b)$ for $a,b \in \mathbb{N}$
\subsection{GCD Characterization Theorem (GCDCT)}
For $a,b,d \in \mathbb{Z}$ where $d \geq 0$ \\
If $d \mid a$ and $d \mid b$ and there exists $s,t \in \mathbb{Z}$ such that $as+bt=d$, then $d = \gcd(a,b)$
\subsection{B\'ezout's Lemma (BL)}
For all $a,b \in \mathbb{Z}$, there exists $s,t \in \mathbb{Z}$ such that $as+bt = \gcd(a,b)$
\subsection{Extended Euclidean Algorithm (EEA)}
\subsection{Common Divisor Divides GCD (CDD GCD)} 
For all integers $a,b,c$, if $c\mid a$ and $c\mid b$, then $c\mid \gcd(a,b)$
\subsection{Coprimeness Characterization Theorem (CCT)}
For all integers $a$ and $b$, $\gcd(a, b) = 1$ if and only if there exist integers $s$ and $t$ such that
$as + bt = 1$
\subsection{Division by the GCD (DB GCD)}
For all integers $a$ and $b$, not both zero, $\gcd(\frac{a}{d}, \frac{b}{d}) = 1$, where $d = \gcd(a,b)$
\subsection{Coprimeness and Divisibility (CAD)}
For all integers $a, b, c$, if $c\mid ab$ and $\gcd(a,c) = 1$, then $c \mid b$
\subsection{Prime Factorization (PF)}
Every integer greater than 1 can be written as the product of primes 
\subsection{Euclid Theorem (ET)}
There are infinitely many primes 
\subsection{Euclid Lemma (EL)}
For all $a,b \in \mathbb{Z}$ and prime $p$, if $p\mid ab$, then $p\mid a$ or $p\mid b$
\subsection{Divisors From Prime Factorization (DFPF)}
Let $n > 1$ be an integer and let $n = p_1^{\alpha_1}p_2^{\alpha_2}p_3^{\alpha_3}\cdots p_k^{\alpha_k}$ where $p$ are prime and $\alpha$ are
positive integers. A positive integer $c$ divides $n$ iff there exists integers $\beta_1, \beta_2, \cdots, \beta_k$ such that 
$c = p_1^{\beta_1}p_2^{\beta_2}p_3^{\beta_3}\cdots p_k^{\beta_k}$ and $0 \leq \beta_i \leq \alpha_i$ for $i = 1, 2, \cdots, k$
\subsection{GCD From Prime Factorization (GCDPF)}
If $a = p_1^{\alpha_1}p_2^{\alpha_2}p_3^{\alpha_3}\cdots p_k^{\alpha_k}$ and $b = p_1^{\beta_1}p_2^{\beta_2}p_3^{\beta_3}\cdots p_k^{\beta_k}$
where $p_1, p_2, \cdots, p_k$ are prime, all exponents are integers greater than or equal to $0$, then 
$\gcd(a,b) = p_1^{\gamma_1}p_2^{\gamma_2}p_3^{\gamma_3}\cdots p_k^{\gamma_k}$ where $\gamma_i = min\{\alpha_i, \beta_i\}$ for $i = 1, 2, \cdots, k$

% Chapter 7
\section{Chapter 7 Linear Diophantine Equations}
Given $a,b,c \in \mathbb{Z}$, find $x,y \in \mathbb{Z}$ such that $ax+by=c$
\begin{itemize}
    \item Is there a solution? (LDET1)
    \item If so, how can we find one? (EEA)
    \item And can we find all solutions? (LDET2)
\end{itemize}
\subsection{LDET1}
Let $a,b \in \mathbb{Z}$, both not zero and $d = \gcd(a,b)$. Then LDE $ax+by=c$ has a solution iff $d \mid c$
\subsection{LDET2}
Let $\gcd(a,b) = d$ where $a,b \neq 0$ \\
If $(x,y) = (x_0, y_0)$ is one particular integer solution to the LDE $ax+by=c$, then the complete integer solution is 
\begin{align*}
    \{(x_0 + \frac{b}{d}n, y_0 - \frac{a}{d}n) : n \in \mathbb{Z} \}
\end{align*}

% Chapter 8
\section{Chapter 8 Congruence and Modular Arithmetic}
Congruence: -1 is congruent to 7 modulo 8 
\subsection{Definition}
Let $a,b \in \mathbb{Z}$, let $m \in \mathbb{N}$ \\
We say a is congruent to b modulo m when $m \mid (a-b)$, write $a \equiv b \pmod{m}$. Otherwise we write $a \not\equiv b \pmod{m}$ \\ 
Note: let $a,b \in \mathbb{Z}$, let $m \in \mathbb{N}$ 
\subsection{Congruence is an Equivalent Relations (CER)}
For $a,b,c \in \mathbb{Z}$, $m \in \mathbb{N}$
\begin{enumerate}
    \item $a \equiv b \pmod{m}$
    \item $a \equiv b \pmod{m} \Rightarrow b \equiv a \pmod{m}$
    \item $a \equiv b \pmod{m}$ and $b \equiv c \pmod{m} \Rightarrow a \equiv c \pmod{m}$
\end{enumerate}
\subsection{Congruence Add and Multiply (CAM)}
For $n \in \mathbb{Z^+}$, for all integers $a_1, a_2, ... , a_n$ and $b_1, b_2, ... , b_n$, if $a_i \equiv b_i \pmod{m}$ for all $1 \leq i \leq n$ then
\begin{enumerate}
    \item $a_1+ ... +a_n \equiv b_1+ ... +b_n \pmod{m}$
    \item $a_1a_2...a_n \equiv b_1b_2...b_n \pmod{m}$
\end{enumerate}
\subsection{Congruence Power (CP)}
For all positive integer $n$ and $a,b,c \in \mathbb{Z}$ \\
If $a \equiv b \pmod{m}$ then $a^n \equiv b^n \pmod{m}$ 
\subsection{Congruence Division (CD)}
Let $a,b,c \in \mathbb{Z}$, let $m \in \mathbb{N}$ \\
If $ac \equiv bc \pmod{m}$ and $\gcd(c, m) = 1$, then $a \equiv b \pmod{m}$ 
\subsection{Congruent Iff Same Remainder (CISR) and Congruent To Remainder (CTR)}
$\forall a,b \in \mathbb{Z}$, $m \in \mathbb{N}$ \\
CISR: $a \equiv b \pmod{m}$ or $a$ and $b$ have the same remainder when divides $m$ \\
CTR: $0 \leq b < m$, $a \equiv b \pmod{m}$ iff $a$ has remainder $b$ when divided by $m$
\subsection{Linear Congruence}
Let $m \in \mathbb{Z}$ \\
Let $a,c \in \mathbb{Z}$ where $a \neq 0$ \\ 
Find all $a \in \mathbb{Z}$ such that $ax \equiv c \pmod{m}$
\begin{itemize}
    \item Is there a solution?
    \item If so, can we find one?
    \item If so, can we find all?
\end{itemize}
\subsection{Linear Congruence Theorem (LCT)}
For all integers $a$ and $c$ with $a$ non-zero, the linear congruence $ax \equiv c \pmod{m}$ has a
solution iff $d \mid c$ where $\gcd(a, m) = d$ \\
Moreover, if $x = x_0$ is a particular solution, then the complete solution is $\{x \in \mathbb{Z} : x \equiv x_0 \pmod{\dfrac{m}{d}}\}$ \\
or equivalently: $\{x \in \mathbb{Z} : x \equiv x_0, x_0 + \dfrac{m}{d}, x_0 + 2\dfrac{m}{d}, ... , x_0 + (d-1)\dfrac{m}{d} \pmod{m}\}$ \\
Informally, LCT tells us there is 
\begin{itemize}
    \item one solution modulo $\dfrac{m}{d}$ 
    \item $d$ solutions modulo $m$
\end{itemize}
\subsection{Congruence Class Definition}
Let $m \in \mathbb{N}$, let $a \in \mathbb{Z}$ \\
The congruence class of $a$ modulo $m$ is $[a] = \{x \in \mathbb{Z} : x \equiv a \pmod{m} \}$
The congruence modulo $m$ is $\mathbb{Z}_m = \{[0], [1], [2], ... , [m-1]\}$ or $= \{[x] : x \in \mathbb{Z}\}$ \\
$|\mathbb{Z}_m| = m$ 
\subsection{Operations}
Let $m \in \mathbb{N}$, let $a, b \in \mathbb{Z}$ We define 
\begin{itemize}
    \item $[a] + [b] = [a+b]$
    \item $[a][b] = [ab]$
\end{itemize}
\subsection{Different ways of saying the same thing}
Let $m \in \mathbb{N}$, $a, b \in \mathbb{Z}$ 
\begin{itemize}
    \item $a \equiv b \pmod{m}$
    \item $m \mid (a-b)$
    \item $\exists k \in \mathbb{Z}, a-b = km$
    \item $\exists k \in \mathbb{Z}, a = km + b$
    \item $a$ and $b$ have the same remainder when divided by $m$
    \item $[a] = [b]$ in $\mathbb{Z}_m$
\end{itemize}
\subsection{Identities and Inverses in ${\mathbb{Z}_m}$}
Let $[a] \in \mathbb{Z}_m$
\begin{itemize}
    \item $[0]$ is the additive identity because $[a] + [0] = [a]$
    \item $[1]$ is the multiplicative identity because $[a][1] = [1][a] = [a]$
    \item $[-a]$ is the additive inverse of $[a]$ because $[a] + [-a] = 0$
    \item The multiplicative inverse of $[a]$ (if it exists) is an element $[b]$ such that $[a][b] = [b][a] = [1]$ and we write $[b] = [a]^{-1}$
\end{itemize}
\subsection{Modular Arithmetic Theorem (MAT)}
Let $\gcd(a, m) = d \neq 0$ \\
The equation $[a][x] = [c]$ in $\mathbb{Z}_m$ has a solution iff $d \mid c$. Moreover, if $[x] = [x_0]$
is one particular solution, the complete solution is \\
$\{[x_0], [x_0 + \dfrac{m}{d}], [x_0 + 2\dfrac{m}{d}], ... , [x_0 + (d-1)\dfrac{m}{d}]\}$ 
\subsection{Multiplicative Inverses}
Inverses in $\mathbb{Z}_m$ (INV $\mathbb{Z}_m$) \\
Let $a \in \mathbb{Z}$ with $0 \leq a \leq m-1$. Then element $[a] \in \mathbb{Z}_m$ has a multiplicative inverse 
iff $\gcd(a, m) = 1$. Moreover, when $\gcd(a, m) = 1$, the multiplicative inverse is unique inverses in $\mathbb{Z}_q$
(INV $\mathbb{Z}_q$) \\
For all primes numbers $p$ and non-zero elements $[a] \in \mathbb{Z}_q$ has a unique multiplicative inverse 
\subsection{Fermat's Little Theorem (F$\ell$T)}
Let $p$ be prime. Let $a \in \mathbb{Z}$. If $p \nmid a$, then $a^{p-1} \equiv 1 \pmod{p}$
\subsection{Corollary to F$\ell$T}
Let $p$ be prime, let $a \in \mathbb{Z}$ \\
Then $a^p \equiv a \pmod{p}$
\subsection{Chinese Remainder Theorem (CRT)}
Suppose $\gcd(m_1, m_2) = 1$ and $a_1, a_2 \in \mathbb{Z}$ \\
There is a unique solution modulo $m_1m_2$ to the system 
\begin{align*}
    x &\equiv a_1 \pmod{m_1} \\
    x &\equiv a_2 \pmod{m_2}
\end{align*}
That is, once we have one solution $x = x_0$, CRT also tells us that the full solution is $x \equiv x_0 \pmod{m_1m_2}$
\subsection{General CRT (GCRT)}
If $m_1, m_2, ... , m_k \in \mathbb{N}$ and $\gcd(m_i, m_j) = 1$ whenever $i \neq j$, then for any choice of integers 
$a_1, a_2, ... , a_k$, there exists solution to simultaneous congruences
\begin{align*}
    n &\equiv a_1 \pmod{m_1} \\
    n &\equiv a_2 \pmod{m_2} \\
    &\cdots \\
    n &\equiv a_k \pmod{m_k}
\end{align*}
Moreover, if $n = n_0$ is one integer solution, then the complete solution is $n \equiv n_0 \pmod{m_1m_2...m_k}$
\subsection{Splitting the Modulus Theorem (SMT)}
Let $m_1, m_2$ be coprime positive integers, then for any two integers $x$ and $a$
$$
x \equiv a \pmod{m_1m_2} \Longleftrightarrow
\begin{cases}
    x \equiv a \pmod{m_1} \\
    x \equiv a \pmod{m_2}
\end{cases}
$$

% Chapter 9
\section{Chapter 9 The RSA Public-Key Encryption Scheme}
For all integers $p, q, n, e, d, M, C and R$, if 
\begin{enumerate}
    \item $p$ and $q$ are distinct primes 
    \item $n = pq$
    \item $e$ and $d$ are positive integers such that $ed \equiv 1 \pmod{(p-1)(q-1)}$ and $1 < e$, $d < (p-1)(q-1)$
    \item $0 \leq M < n$
    \item $M^e \equiv C \pmod{n}$ where $0 \leq C < n$
    \item $C^d \equiv R \pmod{n}$ where $0 \leq R < n$
\end{enumerate}
then $R = M$

% Chapter 10
\section{Chapter 10 Complex Numbers}
A complex number in standard form is an expression of form $x + yi$ where $x,y \in \mathbb{R}$. $\mathbb{C} = \{x + yi : x,y \mathbb{R}\}$
\subsection{Arithmetic}
\begin{itemize}
    \item $(a+bi) + (c+di) = (a+c) + (b+d)i$
    \item $(a+bi)(c+di) = (ac-bd) + (ad+bc)i$
\end{itemize}
Informally, we can treat elements of $\mathbb{C}$ as "normal" algebraic expressions where $i^2 = -1$ and "everything works"
\begin{itemize}
    \item 0 is the additive identity and in $\mathbb{C}$
    \item $-z$ is the additive inverse of $z$ in $\mathbb{C}$
    \item 1 is the multiplicative identity in $\mathbb{C}$
    \item $\dfrac{a-bi}{a^2+b^2}$ is the unique multiplicative inverse $a+bi \neq 0$
\end{itemize}
\subsection{Properties of Complex Arithmetic (PCA)}
Let $u,v,z \in \mathbb{C}$ with $z = x+yi$
\begin{enumerate}
    \item $(u+v)+z = u+(v+z)$
    \item $u+v = v+u$
    \item $z+0 = z$ where $0+0i = 0$
    \item $z+(-z) = 0$ where $-z = -x-yi$
    \item $(uv)w = u(vw)$
    \item $uv = vu$
    \item $z \cdot 1 = z$ where $1 = 1+0i$
    \item $z \neq 0 \Rightarrow zz^{-1} = 1$ where $z^{-1} = \dfrac{x-yi}{x^2+y^2}$
    \item $z(u+v) = zu + zv$
\end{enumerate}
\subsection{More about Complex Numbers}
\begin{itemize}
    \item For $z \in \mathbb{C}$, we define $z^0 = 1$, $z^1 = z$, and $z^{k+1} = zz^k$ for $k \in \mathbb{N}$
    \item For $z \in \mathbb{C}$, we define $z^{-k} = (z^k)^{-1}$ for $k \in \mathbb{N}$
    \item Exponent laws with integer exponents hold for complex numbers
    \item The Binomial Theorem (BT) is true when $a, b \in \mathbb{C}$
    \item The complex numbers connot be put "in order"
    \begin{itemize}
        \item $z < w$ and $z \leq w$ do not mean anything for $z, w \in \mathbb{C}$
    \end{itemize}
    \item Let $r \in \mathbb{R}$ where $r > 0$
    \begin{itemize}
        \item $(\sqrt{r}i)^2 = (0 + \sqrt{r}i)\cdot(0 + \sqrt{r}i) = -r$ 
    \end{itemize}
\end{itemize}
\subsection{Complex Conjugate}
Let $z = a+bi$ be a complex number in standard form \\
The complex conjugate of $z$ is $\overline{z} = a-bi$
\subsection{Properties of Complex Conjugates (PCJ)}
Let $z,w \in \mathbb{C}$. Then 
\begin{enumerate}
    \item $\overline{(\overline{z})} = z$
    \item $\overline{z+w} = \overline{z} + \overline{w}$
    \item $z + \overline{z} = 2Re(z)$ and $z - \overline{z} = 2Im(z)i$
    \item $\overline{zw} = \overline{z}$ $\overline{w}$
    \item $z\neq 0 \Rightarrow {\overline{z^{-1}}} = (\overline{z})^{-1}$
\end{enumerate}
\subsection{Modulus}
Let $z = x+yi \in \mathbb{C}$. The modulus of $z$ is $|x+yi| = \sqrt{x^2+y^2}$
\subsection{Properties Modulus (PM)}
\begin{enumerate}
    \item $|z| = 0$ iff $z = 0$
    \item $|\overline{z}| = |z|$
    \item $z\overline{z} = |z|^2$
    \item $|zw| = |z||w|$
    \item If $z \neq 0$, then $|z^{-1}| = |z|^{-1}$
\end{enumerate}
\subsection{Corollary 6}
For all positive integers $n$ and complex number $z_1, z_2, \cdots, z_n$, we have
\begin{enumerate}
    \item $\overline{z_1+z_2+\cdots+z_n} = \overline{z_1}+\overline{z_2}+\cdots+\overline{z_n}$
    \item $\overline{z_1z_2\cdots z_n} = \overline{z_1}\overline{z_2}\cdots\overline{z_n}$
    \item $|z_1z_2\cdots z_n| = |z_1||z_2|\cdots|z_n|$
\end{enumerate}
\subsection{Triangle Inequality (TIQ)}
For all $z,w \in \mathbb{C}$, we have $|z+w| \leq |z|+|w|$
\subsection{Complex Plane}
$\overline{z}$ is the reflection of $z$ in the real axis \\
$|z|$ is the distance from $z$ to the origin \\
$z+w$ is connected to vector addition \\
Definition: \\
The polar form of a complex number $z$ is $z = r(\cos\theta + i\sin\theta)$ \\
$r = |z|$ and $\theta$ is an angle measured counter-clockwise from the positive real axis 
\subsection{Polar Multiplication of Complex Number (PM$\mathbb{C}$)}
For all complex number $z_1 = r_1(\cos\theta_1 + i\sin\theta_1)$, $z_2 = r_2(\cos\theta_2 + i\sin\theta_2)$
\begin{align*}
    z_1z_2 = r_1r_2(\cos(\theta_1 + \theta_2) + i\sin(\theta_1 + \theta_2))
\end{align*}
\subsection{De Moivre's Theorem (DMT)}
For all $n \in \mathbb{Z}$, and $\theta \in \mathbb{R}$
\begin{align*}
    (\cos\theta + i\sin\theta)^n = \cos (n\theta) + i \sin (n\theta)
\end{align*}
\subsection{Collary to DMT}
For all $n \in \mathbb{Z}$, complex number $z = r(\cos\theta + \sin\theta) \neq 0$
\begin{align*}
    z^n = r^n (\cos (n\theta) + i \sin (n\theta))
\end{align*}
\subsection{Complex n-th Roots Theorem (CNRT)}
Let $n \in \mathbb{N}$, if $r(\cos\theta + i\sin\theta)$ is the polar form of a complex number $a$, then the solutions to $z^n = a$ are 
\begin{align*}
    \sqrt[n]{r}(\cos(\dfrac{\theta + 2k\pi}{n}) + i\sin(\dfrac{\theta + 2k\pi}{n})) \text{for} k = 0, 1, 2, \cdots, n-1
\end{align*}

% Chapter 11
\section{Chapter 11 Polynomials}
For all $a,b,c \in \mathbb{C}$ with $a \neq 0$, the solution to $ax^2 + bx + c = 0$ are $\dfrac{-b \pm w}{2a}$ where $w^2 = b^2 - 4ac$
\subsection{Field}
Definition: \\
In this course, the only examples of a field we will encounter are $\mathbb{Q}$, $\mathbb{R}$ and $\mathbb{C}$ (and in notes noly, $\mathbb{Z}_p$ where $p$ is prime)
\begin{itemize}
    \item all non-zero numbers have a multiplicative inverse 
    \item $ab = 0$ iff $a=0$ or $b=0$
\end{itemize}
\subsection{Polynomial Definition}
Let $n \in \mathbb{Z}$, $n \geq 0$ and $a_n, a_{n-1}, \cdots, a_1, a_0 \in \mathbb{F}$ where $\mathbb{F}$ is a field
An expression of form 
\begin{align*}
    a_n x^n + a_{n-1} x^{n-1} + \cdots + a_1 x + a_0
\end{align*}
is a polynomial in $x$ over $\mathbb{F}$ \\
$iz^3 + (2+3i)z + \pi$
\begin{itemize}
    \item $(2+3i)$ - coefficients
    \item $iz^3$ - term 
    \item $z$ - indeterminate
    \item complex polynomial
    \item degree is 3
    \item cubic polynomial 
    \item in $\mathbb{C}[z]$
\end{itemize}
\subsection{Arithmetic with Polynomials}
Let $f(x) = \sum_{i=0}^{m} a_i x^i$ and $g(x) = \sum_{j=0}^{n} b_j x^j$ be the polynomials in $\mathbb{F}[x]$
\begin{itemize}
    \item \textbf{Addition} of $f(x)$ and $g(x)$ is defined as 
        \begin{align*}
            f(x) + g(x) = \sum_{k=0}^{max\{m,n\}} (a_k + b_k)x^k
        \end{align*}
    where $a_k = 0$ for $k > m$, and $b_k = 0$ for $k > n$ 
    \item \textbf{Multiplication} of $f(x)$ and $g(x)$ is defined as 
    \begin{align*}
        f(x)g(x) = \sum_{i=0}^{m} \sum_{j=0}^{n} a_i b_j x^{i+j} = \sum_{\ell = 0}^{m+n} c_{\ell} x^{\ell}
    \end{align*}
    where 
    \begin{align*}
        c_{\ell} = a_0b_{\ell} + a_1b_{\ell - 1} + \cdots + a_{\ell - 1}b_1 + a_{\ell}b_0 = \sum_{i=0}^{\ell} a_i b_{\ell - i}
    \end{align*}
    for $\ell = 0, 1, \cdots, m+n$ and where again $a_k=0$ for $k > m$, and $b_k = 0$ for $k > n$
\end{itemize}
\subsection{Degree of a Product (DP)}
For all fields $\mathbb{F}$, and all non-zero polynomials $f(x)$ and $g(x)$ in $\mathbb{F}[x]$, we have 
\begin{align*}
    \text{deg} f(x)g(x) = \text{deg} f(x) + \text{deg} g(x)
\end{align*}
\subsection{Division Algorithm for Polynomials (DAP)}
For all fields $\mathbb{F}$, and all polynomials $f(x)$ and $g(x)$ in $\mathbb{F}[x]$ with $g(x)$ not the zero polynomial, 
there exist unique polynomial $q(x)$ and $r(x)$ in $\mathbb{F}[x]$ such that 
\begin{align*}
    f(x) = q(x)g(x) + r(x)
\end{align*}
where $r(x)$ is the zero polynomial, or deg $r(x) < $ deg $g(x)$
\subsection{Remainder Theorem (RT)}
Suppose $f(x) \in \mathbb{F}[x]$ and $c \in \mathbb{F}$. The remainder when $f(x)$ is divided by $x-c$ is the constant polynomial $f(c)$
\subsection{Factor Theorem (FT)}
Suppose $f(x)$ and $x-c$ are in $\mathbb{F}[x]$ and $c \in \mathbb{F}$ \\
Then $x-c$ is a factor of $f(x)$ if and only if $f(c) = 0$ \\
Equivalently, $x-c$ is a factor $f(x)$ iff c is a root of $f(x)$
\subsection{Fundamental Theorem of Algebra (FTA)}
Every complex polynomial of positive degree has a complex root
\subsection{Complex Polynomials of Degree n Have n Roots (CPN)}
If $f(x)$ is a complex polynomial of degree $n \geq 1$, then there exist complex numbers $c_1, c_2, \cdots, c_n$ and $c \neq 0$ such that 
\begin{align*}
    f(z) = c(z-c_1)(z-c_2)\cdots(z-c_n)
\end{align*}
Moreover, the roots of $f(z)$ are $c_1, c_2, \cdots, c_n$
\subsection{Proposition 7}
Let $f(x) \in \mathbb{F}[x]$ where $\mathbb{F}$ is a field \\
Let $n$ be the degree of $f(x)$ \\
The number of roots of $f(x)$ is at most $n$
\subsection{Multiplicity}
The multiplicity of a root of $c$ of a polynomial $f(x)$ is the largest positive integer $k$ such that $(x-c)^k$ is a factor of $f(x)$
\subsection{Reducible and Irreducible Polynomial}
A polynomial in $\mathbb{F}[x]$ of positive degree is a reducible polynomial in $\mathbb{F}[x]$ when it can be written as the product of two polynomials in
$\mathbb{F}[x]$ of positive degree. Otherwise, we say that the polynomial is an irreducible polynomial in $\mathbb{F}[x]$
\subsection{Conjugate Roots Theorem (CJRT)}
Let $f(x)$ be a polynomial with real coefficients. If $z\in\mathbb{C}$ and $f(z) = 0$, then $f(\overline{z}) = 0$
\subsection{Real Quadratic Factors (RQF)}
Let $f(x) \in \mathbb{R}[x]$. If $f(c) = 0$ for some $c\in\mathbb{C}$ with $lm(c) \neq 0$, then there exists a real quadratic irreducible polynomial $g(x)$ and a real
polynomial $q(x)$ such that $f(x) = g(x) q(x)$
\subsection{Real Factors of Real Polynomials}
Every non-constant polynomial with real coefficients can be written as a product of real 
linear and real quadratic factors.
\begin{align*}
    (x-r_1)(x-r_2) \cdots (x-r_k)(x-c_1)(x-\overline{c_1}) &(x-c_2)(x-\overline{c_2}) \cdots (x-c_\ell)(x-\overline{c_\ell}) \\
    (x-c_1)(x-\overline{c_1}) &= x^2 - (c_1 + \overline{c_1})x + c_1 \overline{c_1} \\
                              &= x^2 - 2Re(c_1)x + |c_1|^2 \\
                              &\in \mathbb{R}[x]
\end{align*}
\end{document}

\documentclass[12pt, letterpaper]{article}
\usepackage{amsmath,amssymb,amsthm,enumerate,nicefrac,fancyhdr,graphicx,adjustbox,titlesec}
\title{Math 138 Notes}
\author{Thomas Liu}
\begin{document}
\maketitle
\tableofcontents

\newpage

% Week 1
\section{Week 1}
\subsection{Lecture 1 - Riemann Sums and Area}
\subsubsection{Areas under a curve}
Approximate the area under the curve by a bunch of rectangles \\
Example: $f(x) = x^2$ from 0 to 1
\begin{itemize}
    \item 1 rectangle: $1 \times 1 = 1$
    \item 2 rectangles (same width): $\frac{1}{2} + \frac{1}{2}\times\frac{1}{4} = \frac{5}{8}$
    \item 3 rectangles: $\frac{1}{3}\times\frac{1}{9} + \frac{1}{3}\times\frac{4}{9} + \frac{1}{3}\times 1 = \frac{14}{27}$
    \item n rectangles: base $\times$ height
    \begin{itemize}
        \item base: $\frac{1}{n}$
        \item height: $(\frac{i}{n})^2$
        \item $A_i = \frac{1}{n}\cdot(\frac{i}{n})^2$
        \item approximate area under $f(x)$: $R_n = A_1 + A_2 + \cdots + A_n = \displaystyle\sum_{i=1}^{n}A_i = \sum_{i=1}^{n}\frac{i^2}{n^3}$
    \end{itemize} 
\end{itemize}
Area under $f(x) = x^2$ from 0 to 1: $R_n = \displaystyle\sum_{i=1}^{n} f(c_i)\Delta t_i$
\begin{itemize}
    \item $\Delta t_i$ is the width of the $i^{th}$ rectangle, $\Delta t_i = \frac{1}{n}$
    \item $c_i$ is the right end point of the $i^{th}$ rectangle, making $f(c_i)$ the height of the rectangle, $c_i = \frac{i}{n}$
    \item $f(c_i)\Delta t_i$ is the area of the $i^{th}$ rectangle
\end{itemize}
\subsubsection{Riemann sums}
Let $f$ be a bounded function on $[a,b]$
\begin{enumerate}
    \item divide the interval $[a,b]$ into $n$ segments
    \begin{itemize}
        \item $a = t_0 < t_1 < t_2 < \cdots < t_n = b$
        \item set of points $\{t_i\}$ is called a \textbf{partition} $P$
        \item $\Delta t_i = t_i - t_{i-1}$, width of the subinterval $[t_{i-1}, t_i]$
        \item width of the largest subinterval is called the \textbf{norm of the partition}, $||P|| = \displaystyle\max_i \{\Delta t_i\}$
    \end{itemize}
    \item determine the height of the rectangles
    \begin{itemize}
        \item for each $i = 1, 2, \cdots, n$ choose a point $c_i \in [t_{i-1}, t_i]$
        \item height of the $i^{th}$ rectangle if $f(c_i)$, can be negative
    \end{itemize}
\end{enumerate}
The \textbf{Riemann sum} is defined as $S_n = \displaystyle\sum_{i=1}^{n} f(c_i)\Delta t_i$ \\
The value of the sum depends on 
\begin{itemize}
    \item partition (how many subintervals, the choice of $t_i$)
    \item choice of $c_i$ within each subinterval
\end{itemize}
Terminology:
\begin{itemize}
    \item if all the widths of the $n$-subintervals are the same, call it a \textbf{regular n-partition} of $[a,b]$
    \begin{itemize}
        \item width: $\Delta t_i = \Delta t = \frac{b-a}{n}$
        \item $t_i = a + i\Delta t$
    \end{itemize}
    \item choose $c_i = t_i$ for all $i$, call it a \textbf{right hand Riemann sum}, denoted $R_n$
    \item choose $c_i = t_{i-1}$ for all $i$, call it a \textbf{left hand Riemann sum}, denoted $L_n$
\end{itemize}
\subsubsection{The Definite Integral}
Let $f$ be a bounded function on $[a,b]$
\begin{itemize}
    \item $f$ is called \textbf{integrable on [a,b]} if there is a unique $I\in\mathbb{R}$ such that if for any sequence of partitions $P_n$ with $\displaystyle\lim_{n\to\infty} ||P_n||$, the sequence of Riemann sums $S_n$ corresponding to $P_n$ to I, $\displaystyle\lim_{n\to\infty} S_n = I$
    \item $I$ is called the \textbf{integral of $f$ from a to b}, $I = \int_{a}^{b} f(t) dt$
    \item if a sequence of partitions $P_n$ for which $S_n$ does not converge, then it shows the function is not integrable 
    \item if two sequence of partitions $P_n$ and $Q_n$ where respective Riemann sums converge to different values, the function is not integrable
\end{itemize}
\textbf{Theorem}: \\
Let $f$ be continuous (or be bounded and have finitely many discontinuities) on $[a,b]$. Then 
$f$ is integrable on $[a,b]$ and $\int_{a}^{b} f(t)dt = \displaystyle\lim_{n\to\infty} \displaystyle\sum_{i=1}^{n} f(c_i)\Delta t_i$
where the right hand side is the limit of the Riemann sum with any regular n-partition \\
In particular, $\int_{a}^{b} f(t)dt = \displaystyle\lim_{n\to\infty}R_n = \displaystyle\lim_{n\to\infty}L_n$ \\
\textbf{Theorem: Integrability Theorem for Continuous Functions} \\
Let $f$ be continuous on $[a,b]$. Then $f$ is integrable on $[a,b]$ \\
\textbf{Definition: Regular n-partition} \\
When for the interval $[a,b]$ the partition has $n$ subintervals of equal length. That is, $\Delta t = \frac{b-a}{n}$ and $t_i = t_0 + i\Delta t$ \\
\textbf{Definition: Regular Right Hand Riemann Sum} \\
We take $c_i = t_i$ for all $i$: 
\begin{align*}
    S_n = \displaystyle\sum_{i=1}^{n} f(t_i)\Delta t_i = \sum_{i=1}^{n} f(c_i)\Delta t = \sum_{i=1}^{n} f(t_i)\frac{b-a}{n}
\end{align*}
We can also define LHRS similarly
\begin{align*}
    S_n = \displaystyle\sum_{i=1}^{n} f(t_{i-1})\Delta t_i = \sum_{i=1}^{n} f(t_{i-1})\Delta t = \sum_{i=1}^{n} f(t_{i-1})\frac{b-a}{n}
\end{align*}
Put the ideas together
\begin{align*}
    \int_{a}^{b}f(x)dx = \displaystyle\lim_{n\to\infty}\sum_{i=1}^{n}f(t_i)\frac{b-a}{n}
\end{align*}
\subsection{Lecture 2}
\subsubsection*{Note}
\begin{itemize}
    \item The integrability theorem also holds for functions that are bounded and have finitely many discontinuous
    \item Since a definite integral is the limit of a sequence, many limit laws will hold
\end{itemize}
\subsubsection{Theorem: Properties of Integrals}
If $f$ is integrable on $[a,b]$, then
\begin{enumerate}
    \item For $c\in\mathbb{R}$, $\displaystyle\int_{a}^{b}cf(x)dx = c\int_{a}^{b}f(x)dx$
    \item $\displaystyle\int_{a}^{b}(f+g)(x)dx = \int_{a}^{b}f(x)dx + \int_{a}^{b}g(x)dx$
    \item If $m\leq f(x)\leq M$ for $x\in[a,b]$, then $m(b-a) \leq \displaystyle\int_{a}^{b}f(x)dx \leq M(b-a)$
    \item If $0\leq f(x)$ for $x\in[a,b]$, then $0\leq \displaystyle\int_{a}^{b}f(x)dx$
    \item If $f(x)\leq g(x)$ for $x\in[a,b]$, then $\displaystyle\int_{a}^{b}f(x)dx \leq \int_{a}^{b}g(x)dx$
    \item $|f|$ is integrable on $[a,b]$ and $\displaystyle|\int_{a}^{b}f(x)dx|\leq \int_{a}^{b}|f(x)|dx$
\end{enumerate}
\subsubsection{Definition: Additional Properties of Integrals}
\begin{enumerate}
    \item If $f(a)$ is defined, then $\displaystyle\int_{a}^{a}f(x)dx = 0$
    \item If $f$ is integrable on $[a,b]$, then $\displaystyle\int_{a}^{b}f(x)dx = -\int_{b}^{a}f(x)dx$
    \item Theorem: If $f$ is integrable on an interval $I$ containing $a$, $b$, and $c$, then $\displaystyle\int_{a}^{b}f(x)dx = \int_{a}^{c}f(x)dx + \int_{c}^{b}f(x)dx$
    \item[*] $c$ does NOT need to be between $a$ $\&$ $b$
\end{enumerate}
\subsubsection{Geometric Interpretation of the Integral}
We note that $\displaystyle\int_{a}^{b}f(x)dx$ returns the signed area between $f$ and the x-axis. That is,
if $f(x)\leq 0$, then $\displaystyle\int_{a}^{b}f(x)dx \leq 0$ \\
That is, in general, $\displaystyle\int_{a}^{b}f(x)dx$ is the unsigned area under $f$ that lies above the x-axis \underline{minus}
your unsigned area above the graph $\&$ under the x-axis
\subsection{Lecture 3 - Average Value of a Function}
\subsubsection{Definition: Average Value of Function on $[a,b]$}
If $f$ is continuous on $[a,b]$, the average value of $f$ on $[a,b]$ is $\frac{1}{b-a}\displaystyle\int_{a}^{b}f(x)dx$ \\
\textbf{A Geometric Interpretation} \\
If $f$ is continuous on $[a,b]$, EVT says there exists $m,M\in\mathbb{R}$ such that $m\leq f(x)\leq M$ for $x\in[a,b]$
and $f(c_1)=m$, $f(c_2)=M$ for some $c_1, c_2\in[a,b]$
\begin{align*}
    m(b-a) \leq \displaystyle&\int_{a}^{b}f(x)dx \leq M(b-a) \\
    m \leq &\frac{1}{b-a}\displaystyle\int_{a}^{b}f(x)dx \leq M \\
    f(c_1) \leq &\frac{1}{b-a}\displaystyle\int_{a}^{b}f(x)dx \leq f(c_2)
\end{align*}
IVT says there exists $c$ between $c_1$ and $c_2$ such that $f(c) = \frac{1}{b-a}\displaystyle\int_{a}^{b}f(x)dx$
\subsubsection{Theorem: Average Value Theorem / MVT for Integrals}
Assume $f$ is continuous on $[a,b]$. There exists $c\in[a,b]$ such that
\begin{align*}
    f(c) = \frac{1}{b-a}\displaystyle\int_{a}^{b}f(x)dx
\end{align*}
\textbf{Note}: the theorem also holds even if $b < a$
\begin{align*}
    f(c) &= \frac{1}{a-b}\displaystyle\int_{b}^{a}f(x)dx \\
         &= \frac{1}{a-b}(-\displaystyle\int_{a}^{b}f(x)dx) \\
         &= \frac{1}{b-a}\displaystyle\int_{a}^{b}f(x)dx
\end{align*}

% Week 2
\section{Week 2}
\subsection{Lecture 4}
\subsubsection{Integral Function}
Let $f$ be continuous on $[a,b]$ \\
Define $G(x) = \displaystyle\int_{a}^{x}f(t)dt$ for $x\in[a,b]$
\begin{itemize}
    \item $G(x)$ is the function  that returns the signed area under $f$ from $a$ to $x$
\end{itemize}
\subsubsection{Theorem: Fundamental Theorem of Calculus (Part I)}
If $f$ is continuous on an open interval $I$ containing $x=a$ and if \[G(x) = \displaystyle\int_{a}^{x}f(t)dt\]
Then $G(x)$ is differentiable for all $x\in I$ and \[G'(x)=f(x)\] That is \[\frac{d}{dx}\displaystyle\int_{a}^{x}f(t)dt = f(x)\]
\subsubsection{Definition: Anti-Derivative}
Given a function $f$, an anti-derivative of $f$ is a function $F$ such that $F'(x)=f(x)$ \\
*Note: Anti-derivative are \underline{not} unique \\
The collection of all anti-derivatives of $f(x)$ is denoted by $\displaystyle\int f(x)dx = F(x)+c$
where $c\in\mathbb{R}$ and $F$ is any anti-derivative. This is called an indefinite integral (no bounds) \\
*Note: by the Anti-Derivative Theorem, any two anti-derivatives of $f$ differe by a constant
\subsubsection*{A bunch of anti-derivatives}
\begin{itemize}
    \item $\displaystyle\int x^n dx = \frac{x^{n+1}}{n+1} + c$, if $n\neq-1$
    \item $\displaystyle\int \frac{1}{x}dx = \ln|x|+c$
    \item $\displaystyle\int e^x = e^x+c$
    \item $\displaystyle\int \sin(x)dx = -\cos(x)+c$, $\displaystyle\int \cos(x)dx = \sin(x)+c$
    \item $\displaystyle\int \sec^2(x)dx = \tan(x)+c$
    \item $\displaystyle\int \frac{1}{1+x^2}dx = \arctan(x)+c$
    \item $\displaystyle\int \frac{1}{\sqrt{1-x^2}}dx = \arcsin(x)+c$, $\displaystyle\int \frac{-1}{\sqrt{1-x^2}}dx = \arccos(x)+c$
    \item $\displaystyle\int \sec(x)\tan(x)dx = \sec(x)+c$
    \item $\displaystyle\int a^x dx = \frac{a^x}{\ln (a)} + c$, for $a>0, a\neq 1$
\end{itemize}
\subsection{Lecture 5}
\subsubsection{Theorem: Fundamental Theorem of Calculus II (FTC2)}
If $f$ is continuous on $[a,b]$ and $F$ is any anti-derivative of $f$, then \[\displaystyle\int_{a}^{b}f(x)dx = F(b)-F(a) = [F(x)]_a^b\]
\subsubsection{Corollary: Extended Version of FTC}
If $f$ is continuous and $g$ \& $h$ and differentiable, then: 
\[\displaystyle\frac{d}{dx}[\int_{g(x)}^{f(x)}f(t)dt] = f(h(x))h'(x) - f(g(x))g'(x)\]
Also known as Leibniz Formula
\subsubsection{U-Substitution}
This technique is in essen a reverse chain-rule
\[\displaystyle\int f(g(x))g'(x)dx = \int f(u)du\]
Good time for u-substitution: when the function and its anti-derivative are presented
\subsection{Lecture 6}
\subsubsection{Theorem: Change of Variable}
\[\displaystyle\int_{x=a}^{x=b}f(g(x))g'(x)dx = \int_{u=g(a)}^{u=g(b)}f(u)du\]

% Week 3
\section{Week 3}
\subsection{Lecture 7}
\subsubsection{Trignometry Substitution}
There are 3 situations where this technique is useful
\begin{center}
\begin{tabular}{l|l|l}
    If you see                               & Trig substitution                                & Range for $\theta$ \\ 
    $\sqrt{a^2 - x^2}$                       & $x = a\sin\theta$                                & $-\frac{\pi}{2} < \theta < \frac{\pi}{2}$ \\
    $\sqrt{a^2 + x^2}$                       & $x = a\tan\theta$                                & $-\frac{\pi}{2} < \theta < \frac{\pi}{2}$ \\
    $\sqrt{x^2 - a^2}$                       & $x = a\sec\theta$                                & $0 \leq \theta \leq \frac{\pi}{2}$ or $\pi \leq \theta \leq \frac{3\pi}{2}$
\end{tabular}
\end{center}
\subsubsection*{Notes}
\begin{itemize}
    \item The range of $\theta$ is key so that that $\sin / \tan / \sec$ are invertible
    \item You don't have to state the restriction when you saw a trig substitution
    \item Remember: \begin{itemize}
        \item[] $a^2 - a^2\sin^2\theta = a^2\cos^2\theta$
        \item[] $a^2 + a^2\tan^2\theta = a^2\sec^2\theta$
        \item[] $a^2\sec^2\theta - a^2 = a^2\tan^2\theta$
    \end{itemize}
\end{itemize}
\subsection{Lecture 8}
\subsubsection{Integration By Part (IBP)}
Integration by part is related to product rule \\
Product Rule: \[\frac{d}{dx}[u(x)v(x)] =  u'(x)v(x) + u(x)v'(x)\]
Integrate both sides: \[\int\frac{d}{dx}[u(x)v(x)]dx = \int u'(x)v(x)dx + \int u(x)v'(x)dx\]
By FTC: \[u(x)v(x) = \int u'(x)v(x)dx + \int u(x)v'(x)dx\]
Rearrange: 
\begin{align*}
    \int u(x)v'(x)dx &= u(x)v(x) - \int u'(x)v(x)dx \\
    \int udv &= uv - \int vdu \\
    &\text{(Integration By Parts Formula)}
\end{align*}
\subsubsection{IBP Rule of Thumb}
Let $u$ be whatever shows up first in the acronym
\begin{align*}
    &\text{L I A T E} \\
    &\text{L: log} \\
    &\text{I: inverse trig} \\
    &\text{A: algebric (polynomials, etc.)} \\
    &\text{T: trig} \\
    &\text{E: exponent}
\end{align*}
\subsubsection*{Note}
For definite integral: 
\begin{itemize}
    \item find indefinite integral first, and sub in the bounds after
    \item do the bounds as you go along
\end{itemize}
\[\int_{a}^{b}udv = [uv]_a^b - \int_{a}^{b}vdu\]
\[[uv]_a^b = u(b)v(b) - u(a)v(a)\]
\subsection{Letcture 9}
\subsubsection{Integration By Partial Functions}
This is used when integrating rational functions \[\int\frac{p(x)}{q(x)}dx\] when $p$ \& $q$ are polynomials \\
Specifically when $deg(p) < deg(p)$, if $deg(p) \geq deg(q)$, we have to do long division (or synthetic devision) first \\
We need to decompose $\frac{p}{q}$ into partial fractions \\
The first step is to ensure that $q$ is in fully factored form \\
Once we are in a form when $deg(p) < deg(q)$ and $q$ is fully factored, we decompose on the factors we see in $q$
\begin{center}
\begin{tabular}{c|c|c}
    Name                         &Appearance                        &Split \\ \hline
    Dintinct Linear              &$(5x+1)(x-3)\cdots$               &$\dfrac{A}{5x+1}+\dfrac{B}{x-3}+\cdots$ \\ \hline
    Repeated Linear              &$(5x+1)^7$                        &$\dfrac{A}{5x+1}+\dfrac{B}{(5x+1)^2}+$ \\ 
                                 &                                  &$\cdots+\dfrac{G}{(5x+1)^7}$ \\ \hline
    Dintinct, Irreducible Linear &$(5x^2+3x+1)(x^2+3)\cdots$        &$\dfrac{Ax+B}{5x^2+3x+1}+\dfrac{Cx+D}{x^2+3}+\cdots$ \\ \hline 
    Repeated, Irreducible Linear &$(5x^2+3x+1)^7$                   &$\dfrac{Ax+B}{5x^2+3x+1}+\dfrac{Cx+D}{(5x^2+3x+1)^2}+$ \\
                                 &                                  &$\cdots+\dfrac{Mx+N}{(5x^2+3x+1)^7}$
\end{tabular}
\end{center}
Next, we'd set $\frac{p}{q}$ equal to our decomposition and solve for coefficients. Then integrate. 

% Week 4
\section{Week 4}
\subsection*{More Integrals}
\begin{enumerate}
    \item Continuous functions over infinite intervals
    \item Functions with infinite discontinuities
\end{enumerate}
\subsection{Lecture 10 - Continuous functions over infinite intervals}
\subsubsection{Type I Improper Integrals: Infinite Intervals}
Integrals of form: \[\int_{-\infty}^{a}f(x)dx, \int_{a}^{\infty}f(x)dx, \int_{-\infty}^{\infty}f(x)dx\]
\subsubsection*{Key Idea: }
Approach problem with limits
\subsubsection{Type I Solution:}
\begin{itemize}
    \item $\displaystyle\int_{-\infty}^{a}f(x)dx = \lim_{b\to-\infty}\int_{b}^{a}f(x)dx$
    \item $\displaystyle\int_{a}^{\infty}f(x)dx = \lim_{b\to\infty}\int_{a}^{b}f(x)dx$ 
    \item $\displaystyle\int_{-\infty}^{-\infty}f(x)dx = \lim_{b_1\to-\infty}\int_{b_1}^{0}f(x)dx + \lim_{b_2\to\infty}\int_{0}^{b_2}f(x)dx$
    \item[*] Do not try to say $\displaystyle\int_{-\infty}^{-\infty}f(x)dx = \lim_{b\to\infty}\int_{-b}^{b}f(x)dx$
\end{itemize}
We say an improper integral converges if all of the limits exist (and are finite). The integral \underline{diverges} 
if \underline{any} limit fails to exist (include: $\pm\infty$)
\subsubsection{Theorem: P-Test for Type I Integrals}
$\displaystyle\int_{1}^{\infty}\frac{1}{x^p}dx$ converges if and only if $p>1$ \\
If $p > 1$, $\displaystyle\int_{1}^{\infty}\frac{1}{x^p}dx = \frac{1}{p-1}$
\subsubsection{Properties of Type I Integrals}
Suppose $\displaystyle\int_{a}^{\infty}f(x)dx$ \& $\displaystyle\int_{a}^{\infty}g(x)dx$ converges 
\begin{enumerate}
    \item $\displaystyle\int_{a}^{\infty}cf(x)dx$ converges for any $c\in\mathbb{R}$, and $\displaystyle\int_{a}^{\infty}cf(x)dx = c\int_{a}^{\infty}f(x)dx$
    \item $\displaystyle\int_{a}^{\infty}[f(x)+g(x)]dx$ converges and $\displaystyle\int_{a}^{\infty}[f(x)+g(x)]dx = \int_{a}^{\infty}f(x)dx + \int_{a}^{\infty}g(x)dx$
    \item If $f(x) \leq g(x)$ for all $x\geq a$, then $\displaystyle\int_{a}^{\infty}f(x)dx \leq \int_{a}^{\infty}g(x)dx$
    \item If $a<c<\infty$, then $\displaystyle\int_{c}^{\infty}f(x)dx$ converges and $\displaystyle\int_{a}^{\infty}f(x)dx = \int_{a}^{c}f(x)dx + \int_{c}^{\infty}f(x)dx$
\end{enumerate}
In general, finding the value an integral converges to (if in fact it does) can become quite difficult. We do however have a way of 
comparing a more complex improper integral to a simpler one 
\subsubsection{Theorem: Comparison Theorem (For Type I)}
Supppose $f$ \& $g$ are continuous functions where $0\leq g(x)\leq f(x)$ for $x\geq a$
\begin{enumerate}
    \item If $\displaystyle\int_{a}^{\infty}f(x)dx$ converges, then $\displaystyle\int_{a}^{\infty}g(x)dx$ converges too 
    \item If $\displaystyle\int_{a}^{\infty}g(x)dx$ diverges, then $\displaystyle\int_{a}^{\infty}f(x)dx$ diverges too 
\end{enumerate}
\subsection{Lecture 11}
\subsubsection{Definition: Absolute Convergence}
Let $f$ be integrable on $[a,b]$ for all $b\geq a$ \\
We say that $\displaystyle\int_{a}^{\infty}f(x)dx$ converges absolutely if $\displaystyle\int_{a}^{\infty}|f(x)|dx$ converges
\subsubsection{Theorem: Absolute Converges Theorems (ACT)}
Let $f$ be integrable on $[a,b]$ for all $b\geq a$ \\
Then $|f|$ is integrable on $[a,b]$ for all $b\geq a$, and if $\displaystyle\int_{a}^{\infty}|f(x)|dx$ converges, then 
so does $\displaystyle\int_{a}^{\infty}f(x)dx$ \\
In particular, if $|f(x)| \leq g(x)$ for $x\geq a$, and if $\displaystyle\int_{a}^{\infty}g(x)dx$ converges, so does $\displaystyle\int_{a}^{\infty}f(x)dx$
\subsubsection{Type II: Improper Integrals}
Consider $\displaystyle\int_{a}^{b}f(x)dx$: 
\begin{itemize}
    \item If $f$ has an infinite discontinuities at $x=a$, then $\displaystyle\int_{a}^{b}f(x)dx = \lim_{t\to a^+}\int_{t}^{b}f(x)dx$
    \item If $f$ has an infinite discontinuities at $x=a$, then $\displaystyle\int_{a}^{b}f(x)dx = \lim_{t\to b^-}\int_{a}^{t}f(x)dx$
    \item If $f$ is not continuous at $c$, $a<c<b$, then $\displaystyle\int_{a}^{b}f(x)dx = \int_{a}^{c}f(x)dx + \int_{c}^{b}f(x)dx$ and use two previous points to replace problem bounds with limits 
\end{itemize}
Again, if all limits exist, we say the integral converges \\
If any limit fials to exist, we say the integral diverges 
\subsection{Lecture 12}
\subsubsection{Theorem: P-Test for Type II}
$\displaystyle\int_{0}^{1}\frac{1}{x^p}dx$ converges if and only if $p<1$, $\displaystyle\int_{0}^{1}\frac{1}{x^p}dx = \frac{1}{1-p}$
\subsubsection{Area between two curves}
Suppose we want to calculate the are between two curves, $f\& g$ \\
This is the area under $f(x)$ minus the area $g(x)$ \\
Using all of the same idea as before, we divide the area into infinitely many thin rectangles, and integrate \\
So for $f\geq g$, the area bounded by $f\&g$ from $x=a$ to $x=b$ is 
\[\int_{a}^{b}f(x)-g(x)dx \text{       top - bottom}\]
\subsubsection*{Actual Formula}
Area between $f\&g$ from $x=a$ to $x=b$ is \[\int_{a}^{b}|f(x)-g(x)|dx\]
So, to deal with this we just split the integral where top/bottom switch \\
We can use the same idea to compute areas where $x$ is a function of $y$, "right - left"

% Week 5
\section{Week 5}
\subsection{Lecture 13}
\subsubsection{Disk Method}
\[V=\int_{a}^{b}\pi[R(x)]^2dx\] $R(x)$ is the radius at each x-value
\subsubsection{Washers Method}
\[V=\int_{a}^{b}\pi([R(x)]^2 - [r(x)]^2)dx\] $R(x)$ is the radius of the outer circle, $r(x)$ is the radius of the inner circle
\subsection{Lecture 14}
\subsubsection{Method: Cylindrical Shells}
Imagine a 3D shape created by rotation (around y-axis) \\
We can build the shape from the inside out by building concentric cylindrical shells of thickness $\Delta x(dx)$
\[V_{shell} = 2\pi r\Delta x = \int_{a}^{b}2\pi rh dx\]
\subsection{Lecture 15}
\begin{center}
\begin{tabular}{c|c|c}
                                    &functions of $x$                   &functions of $y$ \\ \hline
    rotate around a vertical line   &shells                             &washers / disks  \\ \hline
    rotate around a horizontal line &washers / disks                    &shells
\end{tabular}
\end{center}
\begin{itemize}
    \item If you go down one path, and meet restriction (region split or nasty integral) then reframe the problem 
    \begin{itemize}
        \item change the function to be in other variables
    \end{itemize}
\end{itemize}
\subsubsection{Differentiable Equations (DEs)}
\subsubsection*{Definition: Ordinary Differential Equations (ODE)}
An equation containing derivatives of a dependent variable (ie. function) $y=f(x)$ is called an ordinary differential equation \\
(To contrast: partial DEs (PDE) are multivariable functions)
\subsubsection*{Definition: Order}
The order of a DE is the highest derivative present 
\subsubsection*{Definition: Linear}
An ODE is called linear if it only contains linear functions of $y$ ($y, y', y''$, etc.)
\subsubsection*{Definition: General Solution}
The general solution of an ODE is the collection of all possible functions that solve the ODE, including arbitrary constants
\subsubsection*{Definition: Particular Solution}
A solution in which all arbitrary constants have been determined
\subsubsection*{Definition: Initial Condition (IC)}
Extra information about y and/or its derivatives at specific points that allow us to find a particular solution 
\subsubsection*{Definition: Initial Value Problem (IVP)}
An DE with IC(s)
\subsubsection*{Definition: Constant Solution / Equilibrium Solution}
A solution to a DE which is a constant function
\subsubsection*{Definition: Direction Field}
For ODEs of form \[\frac{dy}{dx} = f(x,y)\] we can create a visualization called a direction field \\
The direction field shows the family of curves (general solution) satisfying the DE. An IC would be point,
and te particular solution follows the field through the point

% Week 6
\section{Week 6}
\subsection{Lecture 16}
\subsubsection{Definition: Separable DE}
A separable ODE is a first order DE that can be written as \[\frac{dy}{dx} = g(y)h(x)\]
that is, we can factor the RHS into a product of a function of y and a function of x 
\subsection*{Solving Technique}
\begin{align*}
    \frac{dy}{dx} &= g(y)h(x) \\
    \frac{1}{g(y)}\frac{dy}{dx} &= h(x) \\
    \int\frac{1}{g(y)}\frac{dy}{dx}dx &= \int h(x)dx \\
    \int\frac{1}{g(f(x))}f'(x)dx &= \int h(x)dx \\
    \int\frac{1}{g(y)}dy &= \int h(x)dx
\end{align*}
\subsection{Lecture 17}
\subsubsection{Definition: First Order Linear DE (FOLDE)}
A DE of form: \[A(x)y' + B(x)y = C(x) \text{ where } A(x)\neq0\]
that can be written as: \[y' + P(x)y = Q(x)\]
is a FOLDE \\
Solve such DE using an "integrating factor" \\
our FOLDE algorithm:
\begin{enumerate}
    \item Write in form $\dfrac{dy}{dx} + P(x)y = Q(x)$
    \item Find $\mu(x) = e^{\int P(x)dx}$
    \item Multiply both sides by $\mu(x)$, collapse the LHS by product rule 
    \item Integrate both sides (+C !) and solve for $y$
\end{enumerate}
\subsubsection{Formula for FOLDE solution}
\[y = \frac{1}{\mu(x)}[\int\mu(x)Q(x)dx]\text{, where } \mu(x) = e^{\int P(x)dx}\]
\subsubsection{Newton's Law for Cooling}
\[\frac{dT}{dt} = -k(T-T_{room}) \quad k>0\]
\begin{itemize}
    \item $\dfrac{dT}{dt}$ rate of change in temp
    \item $k$ constant 
    \item $T$ temperature function 
    \item $T_{room}$ srround room temperature
\end{itemize}
\[T(t) = Ce^{-kt} + T_{room} \quad \text{For $c,k\in\mathbb{R}$}\]
\subsection{Lecture 18}
\subsubsection{Population Growth}
\begin{enumerate}
    \item Expoential / Natural Growth 
    \[\frac{dP}{dt} = kP\]
    \item Logistic Model
    \[\frac{dP}{dt} = kP(1-\frac{P}{M})\]
    \begin{itemize}
        \item $k>0$
        \item $M$ carrying capacity
    \end{itemize}
\end{enumerate}

% Week 7
\section{Week 7}
\subsection{Lecture 19}
\subsubsection{Definition: Infinite Series}
Let $\{a_n\}_{n=1}^\infty$ be a sequence. An infinite series is an expression of the form 
\[a_1 + a_2 + a_3 + \cdots = \sum_{n=1}^{\infty}a_n\]
\subsubsection{Definition: Sequence of Partial Sum}
If $\displaystyle\sum_{n=1}^{\infty}a_n$ is a series, we define its sequence of partial sums $\{S_n\}$ as:
\[S_n = a_1 + a_2 + \cdots + a_{n-1} + a_n\]
\subsubsection{Defintion: Convergence / Divergence of Series}
A series $\displaystyle\sum_{n=1}^{\infty}a_n$ $(\sum a_n)$ converges to $S\in\mathbb{R}$ if $\displaystyle\lim_{n\to\infty}S_n = S$.
Here $S$ is called the sum of the series. \\
If $\{S_n\}$ diverges, we say the series diverges
\subsubsection{Geometric Series}
A geometric series is of form $\displaystyle\sum_{n=0}^{\infty}r^n = 1 + r + r^2 + \cdots + r^n + \cdots$ for some $r\in\mathbb{R}$ \\
$\displaystyle\sum_{n=0}^{\infty}r^n = \frac{1}{1-r}$ if $|r|<1$ and diverges otherwise 
\subsubsection*{A common other version for Geometric Series}
\[\sum_{n=k}^{\infty}ar^n = \frac{ar^k}{1-r} \quad \text{if $|r|<1$}\]
\subsection{Lecture 20}
\subsubsection{Theorem: Arithmetic for Series}
Suppose $\displaystyle\sum_{n=1}^{\infty}a_n = A$ and $\displaystyle\sum_{n=1}^{\infty}b_n = B$ and $k\in\mathbb{R}$
\begin{enumerate}
    \item $\displaystyle\sum_{n=1}^{\infty}ka_n = kA$
    \item $\displaystyle\sum_{n=1}^{\infty}(a_n \pm b_n) = A \pm B$
    \item If $\displaystyle\sum_{n=1}^{\infty}a_n$ converges, then $\displaystyle\sum_{n=j}^{\infty}a_n$ also converges for each $j\geq1$
    \item If $\displaystyle\sum_{n=j}^{\infty}a_n$ converges for some $j$, then $\displaystyle\sum_{n=1}^{\infty}a_n$ converges 
    \item[] Convergence only depends on the tail 
    \item[] Add or Subtract finite number terms does not affect convergence
\end{enumerate}
\subsection{Lecture 21 - Series Tests}
\subsubsection{Theorem}
If $\displaystyle\sum_{n=1}^{\infty}a_n$ converges, then $\displaystyle\lim_{n\to\infty}a_n=0$
\subsubsection{Theorem: Divergence Test}
If $\displaystyle\lim_{n\to\infty}a_n\neq0$ (or DNE), then $\displaystyle\sum_{n=1}^{\infty}a_n$ diverges \\
Note:
\begin{itemize}
    \item always the first test to try 
    \item only tell about divergence, not convergence 
    \begin{itemize}
        \item result: $\displaystyle\sum_{n=1}^{\infty}a_n$ \begin{itemize}
            \item[] $\neq 0$, series diverges 
            \item[] $=0$, no information  
        \end{itemize}
    \end{itemize}
\end{itemize}

% Week 8
\section{Week 8}
\subsection{Lecture 22}
\subsubsection{Integral Test}
Let $\sum a_n$ be a positive series. Suppose that $a_n = f(n)$ where $f$ is a continuous, positive, decreasing function for all $x\geq N$. Then,
$\displaystyle\sum_{n=N}^{\infty}a_n$ and $\displaystyle\int_{N}^{\infty}f(x)dx$ either both converge or both diverge 
\subsubsection*{Note}
\begin{itemize}
    \item Must always state (if a simple function) or show (complicated function) the continuous, positive, decreasing native
    \item These qualities only need to take place beyond cutoff $N$
    \item The integral test does not say the series converges to the integral value
\end{itemize}
\subsubsection{P-Series}
$\displaystyle\sum_{n=1}^{\infty}\frac{1}{n^p}$ converges if $p>1$ and diverges if $p\leq 1$
\subsubsection{Bounds on the Remainder (Error) in the Integral Test}
\[\int_{n+1}^{\infty}f(x)dx \leq R_n \leq \int_{n}^{\infty}f(x)dx\text{ where $R_n = S-S_n$}\]
We can rewrite:
\[\int_{n+1}^{\infty}f(x)dx \leq S-S_n \leq \int_{n}^{\infty}f(x)dx\]
\[S_n+\int_{n+1}^{\infty}f(x)dx \leq S \leq S_n+\int_{n}^{\infty}f(x)dx\]
\subsection{Lecture 23}
\subsubsection{Theorem: (Direct) Comparison Test}
Assume $0\leq a_n\leq b_n$ for $n\in\mathbb{N}$ (eventually)
\begin{enumerate}
    \item If $\sum b_n$ converges, then $\sum a_n$ converges too 
    \item If $\sum a_n$ diverges, then $\sum b_n$ diverges too 
\end{enumerate}
\subsubsection{Theorem: Limit Comparison Test (LCT)}
If $a_n\geq 0$ and $b_n\geq 0$ for $n\in\mathbb{N}$ (eventually) and $\displaystyle\lim_{n\to\infty}\frac{a_n}{b_n} = L$, 
$L\neq0$, $L<\infty$, then either both $a_n$ and $b_n$ converge or both diverge
\subsubsection*{Note}
Use LCT for:
\begin{itemize}
    \item $\displaystyle\sum\frac{\text{powers of n}}{\text{powers of n}} \rightarrow$ dominating terms
    \item "almost" geometric series
\end{itemize}
\subsubsection{Theorem: LCT (full version)}
If $a_n\geq 0$ and $b_n > 0$ for $n\in\mathbb{N}$ (eventually) and $\displaystyle\lim_{n\to\infty}\frac{a_n}{b_n} = L$, then:
\begin{enumerate}
    \item If $0<L<\infty$ then $\sum a_n$ converges iff $\sum b_n$ converges 
    \item If $L=0$ and $\sum b_n$ converges then $\sum a_n$ converges (if $\sum a_n$ diverges then $\sum b_n$ diverges)
    \item If $L=\infty$ and $\sum a_n$ converges then $\sum b_n$ converges (if $\sum a_n$ diverges then $\sum b_n$ diverges)
\end{enumerate}
\subsection{Lecture 24}
\subsubsection{Alternating Series Test}
Assume that 
\begin{enumerate}
    \item $a_n>0$ for all n 
    \item $a_{n+1}\leq a_n$ for all n 
    \item $\displaystyle\lim_{n\to\infty}a_n = 0$
\end{enumerate}
Then the alternating series $\displaystyle\sum_{n=1}^{\infty}(-1)^{n-1}a_n$ converges
\subsubsection{Theorem: Error/Remainder Bound for Alternating Test}
\[|R_n| = |S - S_n| \leq a_{n+1} \text{, $a_{n+1}$ is the next unused term}\]

\section{Week 9}
\subsection{Lecture 25}
\subsubsection{Definition: Absolutely Convergent}
A series $\displaystyle\sum_{n=1}^{\infty}a_n$ is absolutely convergent if $\displaystyle\sum_{n=1}^{\infty}|a_n|$ converges
\subsubsection{Definition: Conditional Convergence}
A series is conditionally convergent if it is convergent but not absolutely convergent 
\subsubsection{Theorem: Absolute Convergence Theorem (ACT)}
If $\displaystyle\sum_{n=1}^{\infty}|a_n|$ converges then $\displaystyle\sum_{n=1}^{\infty}a_n$ converges 
\subsection{Lecture 26}
\subsubsection{Theorem: Ratio Test}
Let $\sum a_n$ be any series and suppose $\displaystyle\lim_{n\to\infty}|\frac{a_{n+1}}{a_n}| = L$ where $L\in\mathbb{R}$ or $L=\infty$ 
\begin{enumerate}
    \item If $L<1$, then $\sum a_n$ converges absolutely
    \item If $L>1$ or $L=\infty$, then $\sum a_n$ diverges 
    \item If $L=1$, inconclusive (no info)
\end{enumerate}
\subsubsection*{Note}
Ratio test is good to use when 
\begin{itemize}
    \item factorial 
    \item 'almost' geo series
\end{itemize}
\subsubsection{Theorem}
For $x\in\mathbb{R}$, $\displaystyle\lim_{n\to\infty}\frac{x^n}{n!} = 0$
\subsubsection{Theorem: Root Test}
Let $\displaystyle\sum_{n=1}^{\infty}a_n$ be any series and assume $\displaystyle\lim_{n\to\infty}\sqrt[n]{|a_n|} = L$, $L\in\mathbb{R}$ or $L=\infty$
\begin{enumerate}
    \item If $L<1$, then $\sum a_n$ converges absolutely
    \item If $L>1$, then $\sum a_n$ diverges 
    \item If $L=1$, inconclusive (no info)
\end{enumerate}
\subsection{Lecture 27 - Practices}
No notes

\section{Week 10} 
\subsection{Lecture 28}
\subsubsection{Definition: Power Series}
$\displaystyle\sum_{n=0}^{\infty}a_nx^n = a_0 + a_1x + a_2x^2 + \cdots\rightarrow$ center 0 \\
$\displaystyle\sum_{n=0}^{\infty}a_n(x-a)^n = a_0 + a_1(x-a) + a_2(x-a)^2 + \cdots\rightarrow$ center a \\
Where $a_i\in\mathbb{R}\forall i$ \\
The domain of a power series is the collection of all $a\in\mathbb{R}$ for which the power series converges \\
Domain is never empty, series always converges to $a_0$ at centre
\subsubsection*{Convention for $\displaystyle\sum_{n=0}^{\infty}a_n(x-a)^n$}
\begin{enumerate}
    \item When $n=0$, the term is $a_0$ for all $x$, including $x=a$
    \item If the first feew term are zero ($a_0=a_1=\cdots=a_k=0$), then $\displaystyle\sum_{n=0}^{\infty}a_n(x-a)^n = \sum_{n=k+1}^{\infty}a_n(x-a)^n$ (throw out term where $a_n = 0$)
\end{enumerate}
\subsubsection{Theorem}
For a given power series $\displaystyle\sum_{n=0}^{\infty}a_n(x-a)^n$, there are 3 possibilities:
\begin{enumerate}
    \item The series converges only when $x=a$
    \item The series converges $\forall x\in\mathbb{R}$ 
    \item There exists $R\in\mathbb{R}$ such that the series converges absolutely for $|x-a|<R$, diverges if $|x-a|>R$ and may converge or diverge when $|x-a|=R$
\end{enumerate}
\subsection{Lecture 29}
\subsubsection{Geometric Series}
$\displaystyle\frac{1}{1-x} = \sum_{n=0}^{\infty}x^n$ for $|x| < 1$, $I = (-1, 1), R = 1$
\subsubsection{Theorem: Abel's Theorem}
If $f(x) = \displaystyle\sum_{n=0}^{\infty}a_n(x-a)^n$ has interval of convergence $I$, then $f$ is continuous on $I$ \\
Let $f(x) = \displaystyle\sum_{n=0}^{\infty}a_n(x-a)^n$ \& $g(x) = \displaystyle\sum_{n=0}^{\infty}b_n(x-a)^n$ with radii of convergence $R_f$ \& $R_g$ and invervals of convergence $I_f$ \& $I_g$
\begin{enumerate}
    \item $f(x)\pm g(x) = \displaystyle\sum_{n=0}^{\infty}(a_n\pm b_n)(x-a)^n$ \\
    If $R_f\neq R_g$ then the radius of convergence is $R=\{R_f, R_g\}$ and the interval of convergence is $I = I_f\cap I_g$ \\
    If $R_f = R_g$, then $R\geq R_f$
    \item $(x-a)^kf(x) = \displaystyle\sum_{n=0}^{\infty}a_n(x-a)^{n+k}$ \\
    Where the radius is $R_f$ \& the interval is $I_f$ (no change)
    \item If $c\in\mathbb{R}$, $c\neq0$, and $a=0$ (so $\displaystyle\sum_{n=0}^{\infty}a_nx^n$) \\
    Then $f(cx^k) = \displaystyle\sum_{n=0}^{\infty}a_nc^nx^{nk}$ ||
    Where the radius comes from solving $|cx^k|<R_f$ \\
    If $R_f=\infty$ then $R=\infty$ \\
    The interval is $I = \{x\in\mathbb{R} | cx^k\in I_f\}$
\end{enumerate}
\subsubsection{Theorem}
If $f(x) = \displaystyle\sum_{n=0}^{\infty}a_n(x-a)^n$ with radius of convergence $R>0$, then $f(x)$ is differentiable (thus continuous, integrable) on $(a-R, a+r)$ and 
\begin{enumerate}
    \item $f'(x) = \displaystyle\sum_{n=1}^{\infty}na_n(x-a)^{n-1}$ (change starting index since if $n=0$, term is zero)
    \item $\displaystyle\int f(x) = \sum_{n=0}^{\infty}(\frac{a_n(x-a)^{n+1}}{n+1})+C$
\end{enumerate}
\subsubsection*{Note}
While the radius of convergence will not change via differentiate/integrate, the interval of convergence may change
\subsection{Lecture 30}
\subsubsection{Proposition}
$e^x = \displaystyle\sum_{n=0}^{\infty}\frac{x^n}{n!}$ for all $x\in\mathbb{R}$

\section{Week 11} 
\subsection{Lecture 31}
\subsubsection{Definition: $n^{th}$ degree Taylor Polynomials}
If $f$ is n-times differentiable at $x=a$, the $n^{th}$ degree Taylor polynomials for $f$ centered $x=a$ is 
\[T_{n,a}(X) = \sum_{k=0}^{\infty}\frac{f^{(k)}(a)}{k!}(x-a)^k\]
\subsubsection{Definition: Remainder}
If $f$ is n-times differentiable at $x=a$, we define the $n^{th}$ degree Taylor remainder function centered at $x=a$ to be 
\[R_{n,a}(x) = f(x) - T_{n,a}(x)\]
The error in using $T_{n,a}(x)$ to approximate $f(x)$ is given by Error$=|R_{n,a}(x)|$
\subsubsection{Theorem: Taylor's Theorem}
Assume $f$ is $(n+1)$ times differentiable on an interval $I$ containing $x=a$. Let $x\in I$. Then, there exists
a point $c$ between $x\& a$ such that
\[f(x)-T_{n,a}(x) = R_{n,a}(x) = \frac{f^{n+1}(c)}{(n+1)!}(x-a)^{n+1}\]
\subsubsection{Corollary: Taylor's Inequality}
$|R_{n,a}(x)| \leq \displaystyle\frac{M|x-a|^{n+1}}{(n+1)!}$ where $|f^{n+1}(c)|\leq M$ for all $c$ between $x\& a$
\subsubsection{Theorem}
If $f(x)$ has a power series representation about $x=a$, $f(x) = \displaystyle\sum_{n=0}^{\infty}a_n(x-a)^n$ for $|x-a|<R$, $R>0$ then $a_n=\displaystyle\frac{f^{n}(a)}{n!}$ \\
That is $f(x) = \displaystyle\sum_{n=0}^{\infty}\frac{f^{n}(a)}{n!}(x-a)^n$ (The Taylor series for $f$ centered at $x=a$) \\
Special case: $a=0$, $f(x)=\displaystyle\sum_{n=0}^{\infty}\frac{f^{n}(a)}{n!}x^n$ is called the Maclaurin series for $f$ 
\subsubsection{Theorem: Convergence Theorem for Taylor Series}
Assume $f$ has derivatives of all orders on an interval $I$ containing $x=a$ \\
Assume also there exists $M\in\mathbb{R}$ such that $|f^{k}(x)|\leq M$ for all $k\in\mathbb{N}$ and $x\in I$ \\
Then, \[f(x) = \sum_{n=0}^{\infty}\frac{f^{n}(a)}{n!}(x-a)^n\text{ for $x\in I$}\]
\subsection{Lecture 32}
\subsubsection{Corollary}
$e^x = \displaystyle\sum_{n=0}^{\infty}\frac{x^n}{n!}$ for all $x\in\mathbb{R}$
\subsubsection{Corollary}
Both $\sin(x)$ \& $\cos(x)$ are equal to their Taylor series for all $x\in\mathbb{R}$
\[\cos(x) = \sum_{n=0}^{\infty}\frac{(-1)^nx^{2n}}{(2n)!} \text{ for all $x\in\mathbb{R}$}\]
\[\sin(x) = \sum_{n=0}^{\infty}\frac{(-1)^nx^{2n+1}}{(2n+1)!} \text{ for all $x\in\mathbb{R}$}\]
\subsection{Lecture 33}
\subsubsection{Theorem: Generalized Binomial Theorem}
Let $k\in\mathbb{R}$. Then for all $x\in(-1, 1)$: 
\[(1+x)^k = \sum_{n=0}^{\infty}{k\choose n}x^n \text{ where } {k\choose n} = \frac{k(k-1)\cdots(k-n+1)}{n!}\]
${k\choose 0} = 1$


\end{document}
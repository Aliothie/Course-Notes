\documentclass[11pt]{article}
\usepackage{amsmath,amssymb,amsthm,enumerate,nicefrac,fancyhdr,hyperref,graphicx,adjustbox}
\hypersetup{colorlinks=true,urlcolor=blue,citecolor=blue,linkcolor=blue}
\usepackage[left=2.6cm, right=2.6cm, top=1.5cm, includehead, includefoot]{geometry}
\usepackage[dvipsnames]{xcolor}
\usepackage[d]{esvect}

%% commands
%% useful macros [add to them as needed]
% sets
\newcommand{\C}{{\mathbb{C}}} 
\newcommand{\N}{{\mathbb{N}}}
\newcommand{\Q}{{\mathbb{Q}}}
\newcommand{\R}{{\mathbb{R}}}
\newcommand{\Z}{{\mathbb{Z}}}
\newcommand{\F}{{\mathbb{F}}}

% bases
\newcommand{\mA}{\mathcal{A}}
\newcommand{\mB}{\mathcal{B}}
\newcommand{\mC}{\mathcal{C}}
\newcommand{\mD}{\mathcal{D}}
\newcommand{\mE}{\mathcal{E}}
\newcommand{\mL}{\mathcal{L}}
\newcommand{\mM}{\mathcal{M}}
\newcommand{\mO}{\mathcal{O}}
\newcommand{\mP}{\mathcal{P}}
\newcommand{\mS}{\mathcal{S}}
\newcommand{\mT}{\mathcal{T}}

% linear algebra
\newcommand{\diag}{\operatorname{diag}}
\newcommand{\adj}{\operatorname{adj}}
\newcommand{\rank}{\operatorname{rank}}
\newcommand{\spn}{\operatorname{Span}}
\newcommand{\proj}{\operatorname{proj}}
\newcommand{\prp}{\operatorname{perp}}
\newcommand{\refl}{\operatorname{refl}}
\newcommand{\tr}{\operatorname{tr}}
\newcommand{\nul}{\operatorname{Null}}
\newcommand{\nully}{\operatorname{nullity}}
\newcommand{\range}{\operatorname{Range}}
\renewcommand{\ker}{\operatorname{Ker}}
\newcommand{\col}{\operatorname{Col}}
\newcommand{\row}{\operatorname{Row}}
\newcommand{\cof}{\operatorname{cof}}
\newcommand{\Num}{\operatorname{Num}}
\newcommand{\Id}{\operatorname{Id}}
\newcommand{\ipb}{\langle \thinspace, \rangle}
\newcommand{\ip}[2]{\left\langle #1, #2\right\rangle} % inner products
\newcommand{\M}[2]{M_{#1\times #2}(\F)}
\newcommand{\RREF}{\operatorname{RREF}}
\newcommand{\cv}[1]{\begin{bmatrix} #1 \end{bmatrix}}
\newenvironment{amatrix}[1]{\left[\begin{array}{@{}*{\numexpr#1-1}{c}|c@{}}}{\end{array}\right]}
\newcommand{\am}[2]{\begin{amatrix}{#1} #2 \end{amatrix}}

% vectors
\newcommand{\vzero}{\vv{0}}
\newcommand{\va}{\vv{a}}
\newcommand{\vb}{\vv{b}}
\newcommand{\vc}{\vv{c}}
\newcommand{\vd}{\vv{d}}
\newcommand{\ve}{\vv{e}}
\newcommand{\vf}{\vv{f}}
\newcommand{\vg}{\vv{g}}
\newcommand{\vh}{\vv{h}}
\newcommand{\vl}{\vv{\ell}}
\newcommand{\vm}{\vv{m}}
\newcommand{\vn}{\vv{n}}
\newcommand{\vp}{\vv{p}}
\newcommand{\vq}{\vv{q}}
\newcommand{\vr}{\vv{r}}
\newcommand{\vs}{\vv{s}}
\newcommand{\vt}{\vv{t}}
\newcommand{\vu}{\vv{u}}
\newcommand{\vvv}{{\vv{v}}}
\newcommand{\vw}{\vv{w}}
\newcommand{\vx}{\vv{x}}
\newcommand{\vy}{\vv{y}}
\newcommand{\vz}{\vv{z}}

% display
\newcommand{\ds}{\displaystyle}
\newcommand{\qand}{\quad\text{and}}
\newcommand{\qandq}{\quad\text{and}\quad}
\newcommand{\hint}{\textbf{Hint: }}
\newcommand{\tri}{\triangle}

% misc
\newcommand{\area}{\operatorname{area}}
\newcommand{\vol}{\operatorname{vol}}
\newcommand{\red}[1]{{\color{red} #1}}
\newcommand{\rc}{\red{\checkmark}}

\title{STAT 332 Notes}
\author{Thomas Liu}
\begin{document}
\maketitle
\tableofcontents

\newpage 
\section{Sample Survey Issues}
\subsection{Terminology}
\subsubsection{Surveys}
Any activity that collects information in an organized and methodical manner about characteristics of 
the units of a population and compiles such info into useful summary form \\
Conduct surveys to learn about a population \\
In survey sampling, population is finite
\subsubsection{Census}
A censs is survey where we examine every unit \\
A sample survey is preferred over census because
\begin{itemize}
    \item lower cost 
    \item timeliness 
    \item carefully conducted survey provides better quality estimates than sloppy census 
\end{itemize}
\subsubsection{Observation Unit}
An entity on which measurement may be taken 
\subsubsection{Target Population}
A collection of observation units we want to study
\subsubsection{Sampling Unit}
A collection of observation units that may be sampled
\subsubsection{Sampling Frame}
The device which allows access to the sampling units in the population from which sample may be selected 
\subsubsection{Study Population}
The collection of all possible observation units that may have been measured in sample 
\subsubsection*{Example}
Do a survey among current UW undergrad, obtain a list of email address of students who volunteered during orientation, pick 100 of them, send the survey to them 
\begin{itemize}
    \item observation units: current UW undergrad
    \item target population: current UW undergrad 
    \item sampling unit: UW undergrad who volunteered during orientation 
    \item sampling frame: list of email address of UW undergrad 
    \item study population: UW undergrad who volunteered during orientation 
\end{itemize}
\subsection{Sampling}
\subsubsection{Sampling Protocol}
The mechanism by which we choose our sample 
\begin{itemize}
    \item Probability sampling protocol: probabilistic method is used to select the sample from the frame 
    \item Non-probability sampling protocol: samples are selected based on subjective judgement of the interviewer
\end{itemize}
\subsubsection{Non-probability Sampling Protocols}
\begin{itemize}
    \item convenience sampling: units are sampled based on whats's easily available 
    \item self-selection sampling: units choose themselves 
    \item quota sampling: units selected so some attributes of the sample match down attributes in the target population 
    \item judgement sampling: units selected so samplers think the sample will be representative of the whole population 
\end{itemize}
\subsubsection*{Issue with These Protocols}
\begin{itemize}
    \item convenience: does the sample really represent the target population? 
    \item self-selection: does the sample volunteered really represent the whole class?
    \item quota: there are some other important attributes not take into account
    \item judgement: judgement might be biased
\end{itemize}
\subsection{Errors}
\begin{itemize}
    \item study error: difference in attributes of interest between target and study population 
    \item sample error: difference in attributes of interest between study population and sample 
    \item measurement error: difference between response measured and true value of attributes of interest for respondents
\end{itemize}
Thus, we assume that 
\begin{itemize}
    \item sampling frame is complete (contain everyone in the target population)
    \item there is no non-response 
    \item measurements are accurate 
\end{itemize}

\section{Simple Probability Sample}
\subsection{Sampling Protocol / Sampling Design}
Let $\mD$ be the set of all possible samples under chosen sampling design 
\begin{itemize}
    \item probability sampling design is determined by assigning to each possible sample $s$, where $s\subseteq\mD$, a probability $P(s)$ ($P(s)>0$)
    \item $\ds\sum_{s\in\mD}P(s)=1$
\end{itemize}
\subsection{Inclusion Probabilities}
\begin{itemize}
    \item 
    $p_i=P(i\in S)$: the first order inclusion probability for unit $i$ is probability that unit $i$ is in random sample $S$
    \item 
    $p_{ij}=P(i\in S, j\in S)$: the second order inclusion probability for units $i$ and $j$ is probability that both units $i$ and $j$ are in random sample $S$
    \item 
    Note $p_{ii}=p_i$
\end{itemize}
\subsection{Sampling Weights}
Define it as \[w_i=\frac{1}{p_i}\] for any sampling design, inverse of first order inclusion probability 
\subsection{Common Protocols}
\subsubsection{Simple Random Sampling Without Replacement}
Sampling protocol in which every one of $N\choose n$ distinct 
samples has an equal chance of being drawn
\begin{align*}
    P(s) = 
    \begin{cases}
        \frac{1}{{N \choose n}} & \text{if $s$ is of size $n$} \quad \rightarrow \quad |s| = n, \\
        0 & \text{otherwise}.
    \end{cases}
\end{align*}
\subsubsection{Stratified Random Sampling}
Divide population of size $N$ into $H$ non-overlapping and non-empty sub-populations (called strata).
In each stratum take a simple random sample without replacement of size $n_h$, $h=1,\cdots,H$
such that the total sample size from $H$ strata is $n$
\subsubsection{Systematic Sampling}
Consider fram $U=\{1,2,\cdots,N\}$ \\
To take a systematic sample of size $n$ from population of size $N$, choose a number from $1$ to 
$k=\frac{N}{n}$, and call this number $r$. Then take all units $r,r+k,r+2k,\cdots,r+(n-1)k$ from population,
then the systematic sample is $s=\{y_j:j=r,r+k,r+2k,\cdots,r+(n-1)k\}$
\subsubsection{Cluster Sampling}
Divide population of size $N$ into $C$ non-overlapping and non-empty sub-populations (called clusters).
Take a simple random sample of clustes without replacement, and in eawch cluster, take a census 
\subsubsection{Two-stage Sampling}
Divide population of size $N$ into $C$ non-overlapping and non-empty sub-populations (like clusters).
Take a simple random sample of clusters without replacement, and in each cluster, take a simple random sample without replacement 
\subsection{Population Parameters}
Suppose out target population is $U=\{1,2,3,\cdots,U\}$
\begin{itemize}
    \item $N$: population size 
    \item study variable or response of interest is \[y_i\text{ for individual }i\]
    \item population average: $\mu=\ds\frac{1}{N}\sum_{i=1}^{N}y_i$
    \item population total: $\tau=\ds\sum_{i=1}^{N}y_i$
    \item population variance: $\sigma^2=\ds\frac{1}{N-1}\sum_{i=1}^{N}(y_i-\mu)^2$, this is unbiased
\end{itemize}
\subsubsection*{Population Parameters for Binary Response}
Suppose the study variance or response of interest is binary \\
We use an indicator variable $z_i$
\begin{align*}
    z_i = 
    \begin{cases}
        1 &\text{if (condition)} \\
        0 &\text{otherwise}
    \end{cases}
\end{align*}
The population total is $\tau=\ds\sum_{i=1}^{N}z_i$ and population average is a proportion 
\[\mu_z = \frac{1}{N}\sum_{i=1}^{N}z_i=\frac{\tau}{N}=\pi\]
\subsubsection{Variance Properties}
\begin{itemize}
    \item For any response we have \[\sigma^2=\frac{1}{N-1}\sum_{i=1}^{N}(y_i-\mu)^2=\frac{1}{N-1}[(\sum_{i=1}^{N}y_i^2) - N_\mu^2]\]
    \item For any binary responses we have \[\sigma_z^2\approx\pi(1-\pi)\] for large $N$
\end{itemize}
We use $\hat{}$ for sample estimates, and $\tilde{}$ for estimator \\
Estimates are fixed values, but estimators are random variables 
\begin{itemize}
    \item $E[\tilde{\mu}]=\sum_{s\in D}\overline{y}(s)P(S=s)$
    \item $Var[\tilde{\mu}] = \sum_{s\in D}(\overline{y}(s)-E[\tilde{\mu}]^2)P(S=s)$
\end{itemize}
\subsection{Random Indicator Variables}
\begin{align*}
    I_i=I(i\in S)=
    \begin{cases}
        1 &\text{if unit $i$ is in sample $S$} \\
        0 &\text{otherwise}
    \end{cases}
\end{align*}
\subsubsection{Useful Results}
\begin{itemize}
    \item $P(I_i=1)=p_i$
    \item $P(I_i=0)=1-p_i$
    \item $E[I_i]=0(1-p_i)+1p_i=p_i$
    \item $I_i^2=I_i$
    \item $Var(I_i)=p_i(1-p_i)$
    \item $Cov(I_i,I_j)=p_{ij}-p_ip_j$
\end{itemize}
\subsection{Simple Random Sampling Without Replacement (SRSWOR)}
Select $n$ units from frame of $N$ units so that each sample of size $n$
has the same probability of selection 
\begin{itemize}
    \item sampling frame: $U=\{1,2,\cdots,N\}$
    \item sample size: fixed at $n$
    \item $N\choose n$ possible samples of size $n$
    \item sampling protocol 
    \begin{align*}
        P(S=s)=
        \begin{cases}
            \frac{1}{{N\choose n}} &\text{if $s$ has $n$ distinct elements} \\
            0 &\text{otherwise}
        \end{cases}
    \end{align*}
\end{itemize}
\subsubsection*{Inclusion Probabilities}
First order inclusion probabilities: $p_i=\frac{n}{N}$ \\
Second order inclusion probabilities: $p_{ij}=\frac{n(n-1)}{N(N-1)}, i\neq j$
\subsection{Estimator Results}
\begin{itemize}
    \item $\mu$ and $\sigma^2$: our population average and variance 
    \item $\tilde{\mu}$ and $\tilde{\sigma}^2$: corresponding estimator 
    \item Under SRSWOR, \[E(\tilde{\mu})=\mu\quad Var(\tilde{\mu})=(1-\frac{n}{N})\frac{\sigma^2}{n}\]
    \item Under SRSWOR, $\tilde{\mu}$ is unbiased estimator for population average $\mu$
\end{itemize}
\subsubsection*{Helpful Derivations}
When sampling design is SRSWOR, we have 
\begin{itemize}
    \item $Var(I_i)=p_i(1-p_i)=\frac{n}{N}(1-\frac{n}{N})$
    \item $Cov(I_i,I_j)=p_{ij}-p_ip_j=\frac{n(n-1)}{N(N-1)}-\frac{n^2}{N^2}=-\frac{n}{N}\frac{1}{(N-1)}(1-\frac{n}{N})$
\end{itemize}
\subsubsection*{Variance Estimator}
Under SRSWOR
\begin{itemize}
    \item $E[\tilde{\sigma}^2]=\sigma^2$, $\tilde{\sigma}^2$ is unbiased 
    \item $Var(\tilde{\mu})=(1-\frac{n}{N})\frac{\sigma^2}{n}$
    \item $f=\frac{n}{N}$ is sampling fraction
    \item $1-f=1-\frac{n}{N}$ is finite population correction factor (fpc)
\end{itemize}
we can write $Var(\tilde{\mu})=(1-f)\frac{\sigma^2}{n}$
\subsubsection*{Standard Error}
\[\hat{SD}(\tilde{\mu})=\sqrt{\hat{Var}(\tilde{\mu})}=\sqrt{(1-\frac{n}{N})\frac{\hat{\sigma}^2}{n}}=\hat{\sigma}\sqrt{\frac{(1-f)}{n}} = s.e.(\hat{\mu})\]
It is not the sample standard deviation, it is smaller. 
\subsection{Confidence Intervals}
Central Limit Theorem (for finite populations) tells if $N, n$ and $N-n$ are large, then  
\[\frac{\tilde{\mu}-\mu}{\sqrt{(1-f)\frac{\tilde{\sigma}^2}{n}}}\sim N(0,1)\]
A large sample $100(1-\alpha)\%$ confidence interval for population average $\mu$ is 
\[\hat{\mu}\pm c\times s.e.(\hat{\mu})=\hat{\mu}\pm c\sqrt{(1-f)\frac{\hat{\sigma}^2}{n}}\]
where $c$ is chosen such that $P(Z\leq c)=1-\alpha/2$ or $P(Z>c)=\alpha/2$ for standard normal distribution $Z\sim N(0,1)$
\subsubsection*{Estimating Total}
\begin{itemize}
    \item $\tau=N\mu$
    \item $E(\tilde{\tau})=\tau$
    \item $Var(\tilde{\tau})=N^2Var(\tilde{\mu})$, $SD(\tilde{\tau})N\times SD(\tilde{\mu})$
    \item $100(1-\alpha)\%$ CI for $\tau$ is \[\hat{\tau}\pm c\times s.e.(\hat{\tau}) = N(\hat{\mu}\pm c\times s.e.(\hat{\mu}))\]
\end{itemize}
\subsubsection*{Estimating a Proportion: Standard Error}
We know if $y_i\in\{0,1\}$, the standard deviation is 
\[\sigma=\sqrt{\frac{N}{N-1}\pi(1-\pi)}\approx\sqrt{\pi(1-\pi)}\]
which results in 
\[s.e.(\hat{\pi})=\sqrt{1-\frac{n}{N}}\frac{\hat{\sigma}}{\sqrt{n}}\approx \sqrt{1-\frac{n}{N}}\frac{\sqrt{\hat{\pi}(1-\hat{\pi})}}{\sqrt{n}}\]
$100(1-\alpha)\%$ confidence interval for proportion $\pi$ is 
\[\hat{\pi}\pm c\times s.e.(\hat{\pi}) = \hat{\pi}\pm c\sqrt{1-\frac{n}{N}}\frac{\sqrt{\hat{\pi}(1-\hat{\pi})}}{\sqrt{n}}\]
in which $P(Z\leq c)=1-\alpha/2$
\subsection{Margin of Error}
We want a confidence interval of length $2L$, or \[\hat{\mu}\pm L\]
\[L=c\sqrt{(1-\frac{n}{N})\frac{\hat{\sigma^2}}{n}}\]
Then we can write $n$ as \[n=(\frac{1}{N}+\frac{L^2}{c^2\hat{\sigma}^2})^{-1}\approx \frac{c^2\hat{\sigma}^2}{L^2}\]
with approximation hold when $N$ is large 
\subsection{Pilot Study}
To determine sample size, we need the $s.e.$ of estimate, which requires $\hat{\sigma}$ \\
Pilot Study: a small porportion of the sample is collected to estimate the nuisance parameters (such as $\sigma$ in the estimation of $\mu$). This relatively small sample is called 'pilot' \\
pilot sample + additional sample = n

\section{Ratio and Regression Estimation with SRS}
\subsection{Ratio Estimation}
\[\theta=\frac{\tau_Y}{\tau_X}=\frac{N\mu_Y}{N\mu_X}=\frac{\mu_Y}{\mu_X}\] is the ration of means \\
Note that 
\[E(\frac{\tilde{\mu}_Y}{\tilde{\mu}_X}) \neq \frac{E(\tilde{\mu}_Y)}{\tilde{\mu}_X}\]
We have \[\tilde{\theta}=\frac{\tilde{\mu}_Y}{\tilde{\mu}_X}\approx\frac{\mu_Y}{\mu_X}+\frac{1}{\mu_X}(\tilde{\mu}_Y-\mu_Y)-\frac{\mu_Y}{\mu_X^2}(\tilde{\mu}_X-\mu_X)\]
With \[E(\tilde{\theta})\approx\frac{\mu_Y}{\mu_X}=\theta\]
\[Var(\tilde{\theta})=\frac{1}{\mu^2_X}Var(\tilde{\mu}_Y-\theta\tilde{\mu}_X)\]
\subsubsection{Residual / Redefined Estimator}
We define residual as $\tilde{\mu}_Y-\theta\tilde{\mu}_X$, and 
\[\tilde{\mu}_Y-\theta\tilde{\mu}_X=\tilde{\mu}_r\]
\subsubsection{Properties of $\tilde{\mu}_r$}
\begin{itemize}
    \item $E[\tilde{\mu}_r]=\mu_r$
    \item $Var(\tilde{\mu}_r) = (1-\frac{n}{N})\frac{\sigma^2_r}{n}$ where $\sigma^2_r=\frac{1}{N-1}\sum_{i\in U}[y_i-\theta x_i-(\mu_Y-\theta\mu_X)]^2$
\end{itemize}
\subsubsection{Ratio Estimation Confidence Interval}
If we assume a large enough sample, $\tilde{\theta}$ is approximately Gaussian, and corresponding 
$100(1-\alpha)\%$ confidence interval for ratio $\theta$ can be constructed as 
\[\hat{\theta}\pm c\times S.E.(\hat{\theta})\]
where standard error is 
\[s.e.(\hat{\theta})=\sqrt{\hat{Var}(\tilde{\theta})} = \frac{1}{|\hat{\mu}_X|}\sqrt{(1-\frac{n}{N})\frac{\hat{\sigma}^2_r}{n}}\]
\subsection{Ratio Estimation of the Mean}
Suppose the relationship between $X$ and $Y$ is 
\[\text{response}=\text{signal}+\text{noise}\]
\[y_i=\theta x_i+R_i\]
\subsubsection{Ratio Estimation of the Mean}
If $X$ and $Y$ are correlated, knowing things about $X$ could help improve estimateion of the mean of $Y$ 
\subsubsection{Ratio Estimate of $\mu_Y$}
\[\hat{\mu}_{ratio}=\hat{\theta}\mu_X=(\frac{\hat{\mu}_Y}{\hat{\mu}_X})\mu_X = \hat{\mu}_Y[\frac{\mu_X}{\hat{\mu}}]\]
We also have 
\[E(\tilde{\mu}_{ratio}) \approx \mu_y\]
\[Var(\tilde{\mu}_{ratio}) \approx (1-\frac{n}{N})\frac{\sigma^2_r}{n}\]
with $\sigma^2_r = \frac{1}{n-1}\sum_{i\in S}(y_i-\hat{\theta}x_i)^2$
\subsubsection{Confidence Interval}
\[\hat{\mu}_{ratio}\pm c\times s.e.(\hat{\mu}_{ratio})\]
where \[s.e.(\hat{\mu}_{ratio})=\sqrt{Var(\tilde{\mu}_{ratio})}\]
\subsection{Regression Estimation}
We define the regression estimate of $\mu_Y$ 
\[\hat{\mu}_{reg}=\hat{\alpha}+\hat{\beta}\mu_X = \hat{\mu}_y+\hat{\beta}(\mu_x-\hat{\mu}_x)\]
where \[\hat{\alpha}=\overline{y}-\hat{\beta}\overline{x}\quad\hat{\beta} = \frac{\sum_{i\in s}(x_i-\overline{x})y_i}{\sum_{i\in s}(x_i-\overline{x})^2}\]
\subsubsection{Expectation and Variance}
\begin{itemize}
    \item $E(\tilde{\mu}_{reg})\approx\mu_Y$
    \item $Var(\tilde{\mu}_{reg})=(1-\frac{n}{N})\frac{\hat{\sigma}^2_{reg}}{n}$ where $\sigma^2_{reg}=\frac{1}{n-1}\sum_{i\in S}[y_i-\hat{\alpha}-\hat{\beta}x_i]^2$
\end{itemize}
\subsubsection{Confidence Interval}
\[\hat{\mu}_{reg}\pm c\times s.e.(\hat{\mu}_{reg})\]
where \[s.e.(\hat{\mu}_{reg})=\sqrt{\frac{1}{n}(1-\frac{n}{N})\frac{\sum_{i\in S}(y_i-\overline{y}-\hat{\beta}(x_i-\overline{x}))^2}{n-1}}\]

\section{Stratified Sample}
Divide population into $H$ mutualy exclusive and non-empty strata, each statum has size $N_h$ \\
$N=N_1+N_2+\cdots+N_h$ is the number of units in the population \\
Variate of interest will be $y$, and write 
\begin{itemize}
    \item $y_{hj}$: the observed value for $j$th element in the $h$th stratum
    \item $\mu_h=\ds\frac{1}{N_h}\sum_{j=1}^{N_h}y_{hj}$: the mean for stratum $h$ (population level)
    \item $\mu = \ds\dfrac{1}{N}\sum_{h=1}^{H}\sum_{j=1}^{N_h}y_{hj}$: the overall population mean
    \item $\sigma^2_h=\ds\frac{1}{N_h-1}\sum_{j=1}^{N_h}(y_{hj}-\mu_h)^2$: variance for statrum $h$
    \item $\sigma^2=\ds\frac{1}{N-1}\sum_{h=1}^{H}\sum_{j=1}^{N_h}(y_{hj}-\mu)^2$: population variance
\end{itemize}
\subsubsection{Population Mean and Stratum Weight}
We can also write the population mean as 
\[\mu=\frac{1}{N}\sum_{h=1}^{H}\sum_{j=1}^{N_h}y_{hj} = \frac{1}{N}\sum_{h=1}^{H}N_h\frac{\sum_{j=1}^{N_h}y_{hj}}{N_h} = \sum_{h=1}^{H}\frac{N_h}{N}\mu_h = \sum_{h=1}^{H}W_h\mu_h\]
where $W_h=\ds\frac{N_h}{N}$ is the stratum weight
\subsection{Estimation}
For a particular stratum $h$, $h=1,\cdots,H$ 
\begin{itemize}
    \item $\hat{\mu}_h = \overline{y}_h = \frac{1}{n_h}\sum_{j\in s_h}y_{hj}$: sample stratum mean 
    \item $\hat{\sigma}^2_h = \frac{1}{n_h-1}\sum_{j\in s_h}(y_{hj}-\hat{\mu}_h)^2$: sample stratum variance
\end{itemize}
Suppose SRSWOR sampling used within each stratum, then 
\begin{itemize}
    \item $E(\tilde{\mu}_h)=\mu_h$
    \item $Var(\tilde{\mu}_h)=(1-\frac{n_h}{N_h})\frac{\sigma^2_h}{n_h}$
\end{itemize}
The estimate of population average is 
\[\hat{\mu}_s=\sum_{h=1}^{H}W_h\hat{\mu}_h = \sum_{h=1}^{H}\frac{N_H}{N}\hat{\mu}_h\]
with corresponding estimator $\tilde{\mu}_s$
\begin{itemize}
    \item $E[\tilde{\mu}_s(y)]=\mu$
    \item $Var(\tilde{\mu}_s) = \ds\sum_{h=1}^{H}W_h^2(1-\frac{n_h}{N_h})\frac{\sigma_h^2}{n_h}$ where $\sigma_h^2 = \ds\frac{1}{n_h-1}\sum_{j\in s_h}(y_{hj}-\mu_h)^2$
\end{itemize}
\subsection{Confidence Interval for Mean $\mu$}
A $100(1-\alpha)\%$ confidence interval for $\mu$ based on a random stratified sample is 
\[\hat{\mu}_s\pm c\times SE(\hat{\mu}_s)\]
\begin{itemize}
    \item $\hat{\mu}_s=\sum_{h=1}^{H}W_h\hat{\mu}_h$
    \item $SE(\hat{\mu}_s)=\sqrt{\ds\sum_{h=1}^{H}W_h^2(1-\frac{n_h}{N_h})\frac{\hat{\sigma}_h^2}{n_h}}$
    \item $P(Z\leq c)=1-\frac{\alpha}{2}$, use $Z$ over $T$ when $\sigma$ is known 
\end{itemize}
\subsection{Proportional Allocation}
We select the same proportion of units from each stratum, $n_h=\ds\frac{n}{N}N_h$ \\
Then we have $W_h=\frac{N_h}{N}=\frac{n_h}{n}=w_h$, the variance of the stratified estimator is 
\[Var(\tilde{\mu}_s)=\frac{1}{n}(1-\frac{n}{N})\sum_{h=1}^{H}W_h\sigma^2_h\]
\subsubsection{Sample Size Determination}
We have \[L=c\sqrt{\frac{1}{n}(1-\frac{n}{N})\sum_{h=1}^{H}W_h\hat{\sigma}^2_h}\]
where $W_h=\frac{N_h}{N} = \frac{n_h}{n}$, use $\frac{N_h}{N}$ here to keep only one $n$  
\subsection{Post Stratification}
The post-stratified population average estimate is 
\[\hat{\mu}_{post} = W_1\hat{\mu}_1+\cdots+W_H\hat{\mu}_H\]
where $\hat\mu_h = \frac{1}{n_h}\sum_{j\in s_h}y_{hj}$ is the stratum $h$ sample average 
\subsubsection{Conditional Expectation and Variance}
Condition on the fact that $\tilde{n}_1=n_1, \cdots, \tilde{n}_H=n_H$ \\
Then we have 
\[E(\tilde{\mu}_{post}|\tilde{n}_1=n_1, \cdots, \tilde{n}_H=n_H) = \mu\]
\[Var(\tilde{\mu}_{post}|\tilde{n}_1=n_1, \cdots, \tilde{n}_H=n_H) = \sum_{h=1}^{H}W_h^2(1-\frac{n_h}{N_h})\frac{\sigma^2_h}{n_h}\]
\subsubsection{Unconditional Expectation and Variance}
\[E[\tilde{\mu}_{post}] = E[E(\tilde{\mu}_{post}|\tilde{n}_1=n_1, \cdots, \tilde{n}_H=n_H)]\approx \mu\]
\[Var(\tilde{\mu}_{post}) = \sum_{h=1}^{H}W_h^2(\frac{1}{n_h}-\frac{1}{N_h})\sigma^2_h\]
\subsection{Cluster Sampling}
\subsubsection{Equal Size}
\begin{itemize}
    \item $N$ is number of clusters 
    \item number of units within each cluster is $M$
    \item population size is $NM=T$
    \item $y_{ij}$ is the response from unit $j$ in cluster $i$
    \item cluster average is $\mu_i=\overline{y}_i = \frac{1}{M}\sum_{j=1}^{M}y_{ij}$
    \item population average is \[\mu = \frac{1}{T}\sum_{i=1}^{N}\sum_{j=1}^{M}y_{ij} = \frac{1}{N}\sum_{i=1}^{N}\overline{y}_i\]
\end{itemize}
The point estimate for population average is 
\[\hat\mu = \frac{1}{nM}\sum_{i\in s}\sum_{j=1}^{M}y_{ij} = \frac{1}{n}\sum_{i\in s}\overline{y}_i\]
\[E[\tilde{\mu}]=\mu_z = \frac{1}{N}\sum_{i=1}^{N}\overline{y}_i = \mu\]
\[Var(\tilde{\mu}) = (1-\frac{n}{N})\frac{\sigma_z^2}{n}\]
where \[\sigma^2_z = \frac{1}{N-1}\sum_{i=1}^{N}(\overline{y}_i - \mu)^2\]
\subsubsection{Unequal Size}
\begin{itemize}
    \item $N$ is number of clusters 
    \item number of units within each cluster is $M_i$
    \item population size is $T=\sum_{i=1}^{N}M_i$
    \item $y_{ij}$ is response from unit $j$ in cluster $i$
    \item cluster average is $\mu_i=\overline{y}_i = \frac{1}{M_i}\sum_{j=1}^{M_i}y_{ij}$
    \item cluster total is $t_i = \sum_{j=1}^{M_i}y_{ij}$
    \item population average is \[\mu = \frac{1}{T}\sum_{i=1}^{N}\sum_{j=1}^{M_i}y_{ij} = \frac{\mu_T}{\mu_M}\]
\end{itemize}
\[Var(\hat\mu) = \frac{1}{\hat\mu^2_M}(1-\frac{n}{N})\frac{\hat\sigma^2_u}{n}\]
where \[\hat\sigma^2_u = \frac{1}{n-1}\sum_{i\in s}(t_i - \hat\mu M_i)^2\]
\subsection{Non-Response}
Response rate is 
\[RR = \frac{\text{number of complete interviews with reporting units}}{\text{number of eligible reporting units in the sample}}\]
We have \\
\begin{tabular}{c|c|c}
    \textbf{Component} & \textbf{Notation} & \textbf{Number of} \\ \hline
Eligible responders & I & Complete interviews \\ 
 & P & Partial interviews \\ \hline
Eligible non-responders & R & Refusals or break-offs \\ 
 & NC & Non-contacts \\ 
 & O & Other eligible non-respondents \\ \hline
Non-responders with unknown eligibility & UH & Unknown household occupancy \\ 
 & UO & Others with unknown eligibility \\ \hline
\end{tabular}
We care most about about RR1 and RR6
\begin{itemize}
    \item RR1: $\ds\frac{I}{I+P+R+NC+O+UH+UO}$
    \item RR6: $\ds\frac{I+P}{I+P+R+NC+O}$
\end{itemize}
\subsubsection{Estimation}
\begin{itemize}
    \item $N_R$: numbere of respondents 
    \item $N_M$: number of non-respondents
    \item $\mu_R$: mean of respondents
    \item $\mu_M$: mean of non-respondents
\end{itemize}
\[\mu = W_R\mu_R + W_M\mu_M = \frac{N_R}{N}\mu_R = \frac{N_M}{N}\mu_M\]

\section{Fundamental of Experimental Plans}
\subsection{The Fundamental Principles}
\subsubsection{Observational Study}
An observational study is one in which data are collected about a population or process without any attempt to change the value of one or more variates for the sampled units 
\subsubsection{Experimental Study}
In experimental study we deliberately change one or more of the process input to investigate the effect of the change on the process output 
\subsubsection*{Difference}
Observational plans provide insight about the effect of changes in varying inputs, but con't conlude causality \\
In experimental plans, people who conduct the experiment change one or more input variates on the sample units before the response variate is measured 
\subsubsection{Terminology}
\begin{itemize}
    \item Factor: a single explanatory variate (input) that will be changed or set on each unit 
    \begin{itemize}
        \item prescribed treatment 
        \item additional therapy
    \end{itemize}
    Can be 
    \begin{itemize}
        \item Quantitative: occlusion dose in hours
        \item Qualitative: type of therapy 
    \end{itemize}
    \item Factor levels: the set of values assigned to any factor in the plan 
    \begin{itemize}
        \item low or high does (2 levels)
        \item glasses, drops, none (3 levels)
    \end{itemize}
    \item Treatment: combination of levels of factors that can be applied to a unit 
    \item Experimental unit: object or individual that we apply treatment to 
    \item Response variate: outcome or observation in the study. Can be continuous 
\end{itemize}
\subsubsection{Fundamental Principles}
\begin{itemize}
    \item Replication: applying each treatment to more than one experimental unit 
    \item Random assignment (random allocation): process of assigning treatments to experimental unit using a probabilistic mechanism
    \item Blocking: groups in which one or more explanatory variates are held fixed while different treatments are applied to the units within the group 
\end{itemize}
\subsubsection{Single Treatment}
If we want to use the model
\[Y_i = \mu+R_i\quad i=1,2,\cdots,20\]
with $R_i\sim N(0,\sigma^2)$, and all $R_i$ are mutually independent 
We know the estimate for the residual variance $\sigma^2$ is 
\[\hat{\sigma}^2 = \frac{1}{n-1}\sum_{i=1}^{n}(y_i-\hat\mu)^2 = \frac{1}{n-1}\sum_{i=1}^{n}(y_i-\overline{y})^2\]
The corresponding estimators have the following properties 
\[\tilde{\mu}\sim N(\mu, \frac{\sigma^2}{n})\quad \frac{\tilde{\sigma}^2(n-1)}{\sigma^2}\sim\chi^2_{n-1}\]
The pivotal quantity is 
\[\frac{\frac{\tilde{\mu}-\mu}{\sqrt{\sigma^2/n}}}{\sqrt{\frac{\tilde{\sigma}^2(n-1)}{\sigma}/(n-1)}} = \frac{\tilde{\mu}-\mu}{\sqrt{\frac{\tilde{\sigma}^2}{n}}}\sim t_{n-1}\]
The $100(1-\alpha)\%$ confidence interval is computed by 
\[\hat\mu\pm c\times s.e.(\hat\mu)\]
where $c$ is chosen such that $P(|T_{n-1}|\leq c)=1-\alpha$, and $s.e.(\hat\mu) = \hat\sigma / \sqrt{n}$
\subsection{Comparative Experimental Plans without Blocking}
\subsubsection{A Model for Comparing Two Treatments}
\[Y_{ij}=\mu+\tau_i+R_{ij}\quad i=1,2\quad j=1,2,\cdots,n\]
where
\begin{itemize}
    \item $Y_{ij}$ is response for the jth unit that received treatment $i$
    \item $R_{ij}\overset{\text{iid}}{\sim} N(0,\sigma^2)$ is the experimental variability (error), with equal variance across groups 
\end{itemize}
Also apply constraint $n_1\tau_1+n_2\tau_2=0$ so that 
\begin{itemize}
    \item $\mu$ is the mean response across our two treatment groups 
    \item $\tau_1,\tau_2$ represent treatment effect, increase or decrease from the mean response due to treatment 
\end{itemize}
\subsubsection{Notation}
\begin{itemize}
    \item $\ds Y_{i+} = \sum_{j=1}^{n_i}Y_{ij}$
    \item $\ds\overline{Y}_{i+} = \frac{1}{n_i}Y_{i+}$
    \item $\ds Y_{++} = \sum_{i=1}^{2}\sum_{j=1}^{n_i}Y_{ij}$
    \item $\ds\overline{Y}_{++} = \frac{1}{n_1+n_2}Y_{++} = (\frac{n_1}{n_1+n_2})\overline{Y}_{1+} + (\frac{n_2}{n_1+n_2})\overline{Y}_{2+}$
\end{itemize}
where \[E\left[\overline{Y}_{++}\right] = \mu\]
\begin{itemize}
    \item $E[\overline{Y}_{i+}] = \mu+\tau_i$
    \item $E[\overline{Y}_{1+} - \overline{Y}_{2+}] = \tau_1-\tau_2 = \theta$
\end{itemize}
$\tilde{\theta} = \overline{Y}_{1+} - \overline{Y}_{2+}$ is the unbiased estimator for $\theta$ \\
We want to minimize $\mu,\tau_1,\tau_2$ from 
\[\sum_{i=1}^{2}\sum_{j=1}^{n_i}(y_{ij}-\mu-\tau_i)^2\]
such that $\sum_{i=1}^{2}n_i\tau_i=0$ \\
We get 
\begin{itemize}
    \item $\hat\mu = \overline{y}_{++}$
    \item $\hat\tau_i = \overline{y}_{i+}-\overline{y}_{++}$
    \item $\hat\tau_1 - \hat\tau_2 = \overline{y}_{1+}-\overline{y}_{2+}$
\end{itemize}
\subsubsection*{Variance Estimation}
The sample variance within treatment $i$ is 
\[\hat\sigma_i^2 = \frac{1}{n_i - 1}\sum_{j=1}^{n_i}(y_{ij}-\hat\mu-\hat\tau_i)^2 = \frac{1}{n_i-1}\sum_{j=1}^{n_i}(y_{ij}-\overline{y}_{i+})^2\]
The pooled overall sample variance is 
\[\hat\sigma^2 = \frac{1}{n_1+n_2-2}\sum_{i=1}^{2}\sum_{j=1}^{n_i}(y_{ij}-\hat\mu-\hat\tau_i)^2 = \frac{(n_1-1)\hat\sigma_1^2 + (n_2-1)\hat\sigma_2^2}{n_1+n_2-2}\] 
\subsubsection{Confidence Interval and Hypothesis Test}
Given the model for comparative studies without blocking 
\[Y_{ij}=\mu+\tau_i+R_{ij}\quad R_{ij}\overset{iid}{\sim}N(0,\sigma^2)\quad i=1,2\quad j=1,\cdots,n_i\]
subject to $\sum_{i=1}^{2}n_i\tau_i=0$, we have $\overline{Y}_{i+}\sim N(\mu+\tau_i,\sigma^2/n_i)$
\subsubsection*{Estimators}
\begin{itemize}
    \item $\overline{Y}_{i+}\sim N(\mu+\tau_i,\sigma^2/n_i)$, so 
    \[\tilde{\theta}=\tilde{\tau}_1-\tilde{\tau}_2 = \overline{Y}_{1+}-\overline{Y}_{2+}\sim N(\tau_1-\tau_2, \sigma^2(\frac{1}{n_1}+\frac{1}{n_2}))\] 
    \item Variance: $\ds\frac{(n_i-1)\tilde{\sigma}_i^2}{\sigma^2}\sim \chi^2_{n_i-1}$, and so 
    \[\frac{(n_1+n_2-2)\hat{\sigma}^2}{\sigma^2}\sim\chi^2_{n_1+n_2-2}\]
\end{itemize}
\subsubsection*{Confidence Interval for $\theta = \tau_1-\tau_2$}
Putting all together to get 
\[\frac{\tilde{\theta}-\theta}{\tilde{\sigma}\sqrt{\frac{1}{n_1}+\frac{1}{n_2}}} = \frac{(\tilde{\tau}_1-\tilde{\tau}_2)-(\tau_1-\tau_2)}{\tilde{\sigma}\sqrt{\frac{1}{n_1}+\frac{1}{n_2}}} \sim t_{n_1+n_2-2}\]
Thus the $100(1-\alpha)\%$ confidence interval for $\theta$ is
\[\hat\tau_1-\hat\tau_2\pm c\times s.e.(\hat\tau_1-\hat\tau_2)\]
where $c$ is chosen such that $P(|T_{n_1+n_2-2}|\leq c)=1-\alpha$, and $s.e.(\hat\tau_1-\hat\tau_2) = \hat\sigma\sqrt{\frac{1}{n_1}+\frac{1}{n_2}}$
\[\hat\sigma = \sqrt{\frac{(n_1-1)\hat\sigma_1^2+(n_2-1)\hat\sigma_2^2}{n_1+n_2-2}}\]
\subsubsection*{Hypothesis Test for $\theta=\tau_1-\tau_2$}
To test $H_0:\theta=\theta_0$, calculate observed test statistic (or discrepancy measure)
\[t_{obs} = \frac{(\hat\tau_1-\hat\tau_2)-\theta_0}{\hat\sigma\sqrt{\frac{1}{n_1}+\frac{1}{n_2}}}\]
There is evidence against $H_0:\theta=\theta_0$ if p-value is small 
\subsection{Comparative Experimental Plans with Blocking}
\subsubsection{Model}
\[Y_{ij}=\mu+\tau_i+\beta_j+R_{ij}\quad i=1,2\quad j=1,2,\cdots,b\quad R_{ij}\overset{iid}{\sim}N(0,\sigma^2)\]
subject to $\ds\sum_{i=1}^{2}\tau_i=0$ and $\ds\sum_{j=1}^{b}\beta_j=0$
\begin{itemize}
    \item $\mu$: overall average improvement in visual acuity
    \item $\tau_i$: treatment effect for treatment $i$
    \item $\beta_j$: block effect for block $j$
    \item $n=2b$: sample size 
\end{itemize}
\subsubsection{Estimation}
Want to minimize $\mu,\tau_i,\beta_j,\lambda_1,\lambda_2$ for 
\[\sum_{i=1}^{2}\sum_{j=1}^{b}(y_{ij}-\mu-\tau_i-\beta_j)^2 + \lambda_1\sum_{i=1}^{2}\tau_i + \lambda_2\sum_{j=1}^{b}\beta_j\]
This gives 
\begin{itemize}
    \item $\hat\mu=\overline{y}_{++}$ for overall mean 
    \item $\hat\tau_i=\overline{y}_{i+}-\overline{y}_{++}$ for treatment effect
    \item $\hat\beta_j=\overline{y}_{+j}-\overline{y}_{++}$ for block effect
\end{itemize}
We are interested in estimating the difference between treatment effects 
\[\hat\theta = \hat\tau_1-\hat\tau_2 = \overline{y}_{1+}-\overline{y}_{2+}\]
The estimated variance is 
\begin{align*}
    \hat\sigma^2 &= \frac{1}{b-1}\sum_{i=1}^{2} \sum_{j=1}^{b}(y_{ij}-\hat\mu-\hat\tau_i-\hat\beta_j)^2 \\
    &= \frac{1}{b-1}\sum_{i=1}^{2}\sum_{j=1}^{b}(y_{ij}-\overline{y}_{i+}-\overline{y}_{+j}+\overline{y}_{++})^2
\end{align*}
\subsubsection{Degrees of Freedom for Two Treatments with Blocking}
The degree of freedom is 
\[sample\ size - number\ of\ parameters\ estimated + number\ of\ constraints\]
\subsubsection{Confidence Interval and Hypothesis Test}
Estimators for $\tau_i$ is 
\[\overline{Y}_{i+}\sim N(\mu+\tau_i, \sigma^2/n)\]
Then 
\[\tilde{\theta}=\tilde{\theta}_1-\tilde{\theta}_2=\overline{D}\sim N(\tau_1-\tau_2, \frac{2\sigma^2}{b})\]
Since $\tilde{\sigma}^2\sim\chi^2_{b-1}$, we have
\[\frac{\tilde{\theta}-\theta}{\tilde{\sigma}\sqrt{2/b}}\sim t_{b-1}\]
The $(1-\alpha)\times 100\%$ confidence interval for $\theta=\tau_1-\tau_2$ is 
\[\hat\tau_1 - \hat\tau_2\pm c\times s.e.(\hat\tau_1-\hat\tau_2)\]
where $c$ is chosen such that $P(|T_{b-1}|\leq c)=1-\alpha$, and $s.e.(\hat\tau_1-\hat\tau_2)=\hat\sigma\sqrt{2/b}$

\section{Experiemntal Plans for More Than Two Treatments}
\subsection{Completely Randomized Designs}
\subsubsection{Balanced Design}
The design is balanced if the same number of units receive each treatment \\
Assuming a balanced plan, we observe a response $y_{ij}$ for the $j^{th}$ unit receiving the $i^{th}$ treatment, and the model is
\[Y_{ij}=\mu+\tau_i+R_{ij}\quad i=1,2,\cdots,t\quad j=1,2,\cdots,n\quad R_{ij}\overset{iid}{sim}N(0,\sigma^2)\]
where
\begin{itemize}
    \item $r$: number of replicates for each tretment 
    \item $t$: number of treatments
    \item $\mu$: mean response across all treatments 
    \item $\tau_i$: treatment effect for treatment $i$
    \item The balanced design constraint is $\sum_{i=1}^{t}n_i\tau_i=0$
\end{itemize}
We have 
\[E(\overline{Y}_{++}) = E(\tilde{\mu}) = \mu\]
\[E(\overline{Y}_{i+}) = \mu+\tau_i\Rightarrow\tau_i = E(\overline{Y}_{i+})=E(\overline{Y}_{i+}-\overline{Y}_{++})\]
\subsubsection{Parameter Estimation}
Using the Lagrange multiplier method and solve the system of $t+2$ equations, we get 
\begin{itemize}
    \item $\hat\mu=\overline{y}_{++}$: overall average 
    \item $\hat\tau_i=\overline{y}_{i+}-\overline{y}_{++}$: treatment effect
    \item $\hat\tau_i-\hat\tau_h = \overline{y}_{i+}-\overline{y}_{h+}$: difference in treatment effect
\end{itemize}
The estimate of the error variance $\sigma^2$ is 
\[\hat\sigma^2 = \frac{1}{rt-(1+t)+1}\sum_{i,j}(y_{ij}-\hat\mu-\hat\tau_i)^2 = \frac{1}{rt-t}\sum_{i,j}(y_{ij}-\overline{y}_{i+})^2\]
The degree of freedom in the denominator is $rt-t$
\[total\ sample\ size - number\ of\ parameters\ estimated + number\ of\ constraints\]
\subsubsection{Dsitributions of the Estimators}
\begin{itemize}
    \item $Y_{ij}=\mu+\tau_i+R_{ij}$, $R_{ij}\overset{iid}{\sim}N(0,\sigma^2)$
    \item $\tilde{\tau}_i=\overline{Y}_{i+}-\overline{Y}_{++}$, $\tilde{\tau}_i-\tilde{\tau}_h = \overline{Y}_{i+}-\overline{Y}_{h+}$
    \item $\overline{Y}_{i+}\sim N(\mu+\tau_i, \sigma^2/n)$
    \item Then, $\overline{Y}_{i+}-\overline{Y}_{h+}\sim N(\tau_i-\tau_h, \sigma^2(1/r+1/r))$
    \item $Z=\dfrac{\tilde{\tau}_i-\tilde{\tau}_h-(\tau_i-\tau_h)}{\sqrt{\sigma^2(1/r+1/r)}}\sim N(0,1)$
    \item $U=\dfrac{t(r-1)\tilde{\sigma}^2}{\sigma^2}\sim \chi^2_{t(r-1)}$, independent of $Z$ 
    \item Therefore, \[\dfrac{Z}{\sqrt{U/(t(r-1))}} = \dfrac{\tilde{\tau}_i-\tilde{\tau}_h-(\tau_i-\tau_h)}{\sqrt{\sigma^2(1/r+1/r)}}\sim t_{t(r-1)}\]
\end{itemize}
\subsubsection{Contrast}
For this course, we need the following condition to be satisfied
\[\theta=\sum_{i=1}^{t}a_i\tau_i=0\quad\sum_{i=1}^{t}a_i=0\]
\subsubsection*{Estimate and Estimator}
\begin{itemize}
    \item Parameter: $\theta=\sum_{i=1}^{t}a_i\tau_i$
    \item Estimate: $\hat\theta=\sum_{i=1}^{t}a_i\hat\tau_i = \sum_{i=1}^{t}a_i\overline{y}_{i+}$
    \item Estimator: $\tilde{\theta}=\sum_{i=1}^{t}a_i\tilde{\tau}_i = \sum_{i=1}^{t}a_i\overline{Y}_{i+}$
\end{itemize}
We have 
\[\tilde{\theta}\sim N(\theta, \sigma^2\sum_{i=1}^{t}a_i^2(1/r))\]
\subsubsection{Confidence Interval and Hypothesis Test}
\[\tilde{\theta}\sim N(\theta, \sigma^2\sum_{i=1}^{t}a_i^2(1/r))\Rightarrow Z=\frac{\tilde{\theta}-\theta}{\sqrt{\sigma^2\sum_{i=1}^{t}a_i^2(1/r)}}\sim N(0,1)\]
It turns that that 
\[\frac{Z}{\sqrt{U/(t(r-1))}} = \frac{\tilde{\theta}-\theta}{\sqrt{\sigma^2\sum_{i=1}^{t}a_i^2(1/r)}}\sim t_{t(r-1)}\]
A $100(1-\alpha)\%$ confidence interval for $\theta$ is
\[\hat\theta\pm c\times s.e.(\hat\theta)\]
where $c$ is chosen such that $P(T_{t(r-1)}\leq c)=1-\alpha/2$, and $s.e.(\hat\theta)=\hat\sigma\sqrt{\sum_{i=1}^{t}a_i^2(1/r)}$ \\
Similarly, the observed test statistic for $H_0:\theta=\theta_0$ is
\[t_{obs} = \frac{\hat\theta-\theta_0}{\hat\sigma\sqrt{\sum_{i=1}^{t}a_i^2(1/r)}}\]
\subsection{Unbalanced Design}
\subsubsection{Model}
\[Y_{ij}=\mu+\tau_i+R_{ij}\quad i=1,2,\cdots,t\quad j=1,2,\cdots,n_i\quad R_{ij}\overset{iid}{\sim}N(0,\sigma^2)\]
where
\begin{itemize}
    \item constraint: $\sum_{i=1}^{t}n_i\tau_i=0$
    \item $\mu = E[\overline{Y}_{++}] = \dfrac{\sum_{i=1}^{t}r_iE[\overline{Y}_{i+}]}{\sum_{i=1}^{t}r_i}$, a weighted average of treatment means 
    \item $\mu+\tau_i=$ average response for treatment $i$
\end{itemize}
\subsubsection{Parameter Estimation}
\begin{itemize}
    \item Estimates: $\hat\mu=\overline{y}_{++}$, $\hat\tau_i=\overline{y}_{i+}-\overline{y}_{++}$
    \item Estimators: $\tilde{\mu}=\overline{Y}_{++}$, $\tilde{\tau}_i=\overline{Y}_{i+}-\overline{Y}_{++}$
    \item Variance: \[\hat\sigma^2 = \dfrac{\sum_{i=1}^{t}\sum_{j=1}^{r_i}(y_{ij}-\overline{y}_{i+})^2}{\sum_{i}(r_i-1)}\]    
\end{itemize}
\subsubsection{Anova Table}
\begin{tabular}{c|c|c|c|c|c}
    \textbf{Source} & \textbf{SS} & \textbf{df} & \textbf{MS} & \textbf{F} \\ \hline
    Treatment & $\sum_{i=1}^{t}r_i(\overline{y}_{i+}-\overline{y}_{++})^2$ & $t-1$ & $\dfrac{SS_{treatment}}{t-1}$ & $\dfrac{MS_{treatment}}{MS_{error}}$ \\ 
    Residuals & $\sum_{i=1}^{t}\sum_{j=1}^{r_i}(y_{ij}-\overline{y}_{i+})^2$ & $\sum_{i=1}^{t}(r_i-1)$ & $\dfrac{SS_{error}}{\sum_{i=1}^{t}(r_i-1)}$ & \\ \hline
    Total & $\sum_{i=1}^{t}\sum_{j=1}^{r_i}(y_{ij}-\overline{y}_{++})^2$ & $\sum_{i=1}^{t}r_i-1$ & &
\end{tabular}
$\ $\newline 
As from the ANOVA table, we have 
\[F = \frac{MS_T}{MS_R} = \frac{\sum_i r_i(\overline{Y}_{i+}-\overline{Y}_{++})^2/(t-1)}{\sum_i\sum_j(y_{ij}-\overline{Y}_{i+})^2/\sum_i(r_i-1)}\sim F_{t-1,\sum_i(r_i-1)}\]
\subsubsection{Contrasts}
Each treatment average has the distribution 
\[\overline{Y}_{i+}\sim N(\mu+\tau_i, \sigma^2/r_i)\]
So for a contrast in unequal sample size we have 
\[tilde{\sigma}\sim N(\theta, \sigma^2\sum_{i=1}^{t}a_i^2/r_i)\]
Hence a $100(1-\alpha)\%$ confidence interval for $\theta$ is
\[\hat\theta\pm t*\hat\sigma\sqrt{\sum_{i=1}^{t}a_i^2/r_i}\]
where $t*$ is chosen such that $P(T_{t-1}\leq t*)=1-\alpha/2$
\subsection{Randomized Block Designs}
\subsubsection{Model}
\[Y_{ij}=\mu+\tau_i+\beta_j+R_{ij}\quad i=1,2,\cdots,t\quad j=1,2,\cdots,b\quad R_{ij}\overset{iid}{\sim}N(0,\sigma^2)\]
where
\begin{itemize}
    \item $\sum_{i=1}^{t}\tau_i=\sum_{j=1}^{b}\beta_j=0$
    \item $\mu$: overall mean response
    \item $\tau_i$: treatment effect for treatment $i$
    \item $\beta_j$: block effect for block $j$
\end{itemize}
\subsubsection{Estimation}
Least squares from the model gives us 
\[\hat\mu=\overline{y}_{++}\quad \hat\tau_i=\overline{y}_{i+}-\overline{y}_{++}\quad \hat\beta_j=\overline{y}_{+j}-\overline{y}_{++}\]
where 
\[\overline{y}_{++}=\frac{1}{tb}\sum_{i=1}^{t}\sum_{j=1}^{b}y_{ij}\quad \overline{y}_{i+}=\frac{1}{b}\sum_{j=1}^{b}y_{ij}\quad \overline{y}_{+j}=\frac{1}{t}\sum_{i=1}^{t}y_{ij}\]
The variance estimate is 
\[\hat\sigma^2 = \frac{1}{(t-1)(b-1)}\sum_{i=1}^{t}\sum_{j=1}^{b}(y_{ij}-\hat\mu-\hat\tau_i-\hat\beta_j)^2\]
\subsubsection{ANOVA Table}
\begin{tabular}{c|c|c|c|c}
    \textbf{Source} & \textbf{SS} & \textbf{df} & \textbf{MS} & \textbf{F} \\ \hline
    Treatment & $\sum_{i=1}^{t}b(\overline{y}_{i+}-\overline{y}_{++})^2$ & $t-1$ & $\dfrac{SS_{treatment}}{t-1}$ & $\dfrac{MS_{T}}{MS_{R}}$ \\ 
    Blocks & $\sum_{j=1}^{b}t(\overline{y}_{+j}-\overline{y}_{++})^2$ & $b-1$ & $\dfrac{SS_{blocks}}{b-1}$ & \\ 
    Residuals & $\sum_{i=1}^{t}\sum_{j=1}^{b}(y_{ij}-\overline{y}_{i+}-\overline{y}_{+j}+\overline{y}_{++})^2$ & $(t-1)(b-1)$ & $\dfrac{SS_{error}}{(t-1)(b-1)}$ & \\ \hline
    Total & $\sum_{i=1}^{t}\sum_{j=1}^{b}(y_{ij}-\overline{y}_{++})^2$ & $tb-1$ & &
\end{tabular}
\subsubsection{Contrasts}
\[\tilde{\theta}=\sum_i a_i\overline{Y}_{i+}\sim N(\theta, \sigma^2\sum_i a_i^2/b)\]
Then 
\[(\tilde{\theta}-\theta)/\sqrt{\hat\sigma_r^2\sum_i a_i^2/b}\sim t_{(t-1)(b-1)}\]    
\begin{itemize}
    \item Confidence interval for $\theta$: $\hat\theta\pm t*\hat\sigma\sqrt{\sum_i a_i^2/b}$
    \item Obversed test statistic for $H_0:\theta=\theta_0$: $t_{obs}=(\hat\theta-\theta_0)/\hat\sigma\sqrt{\sum_i a_i^2/b}$
\end{itemize}
\subsubsection{Notes}
Use blocking design makes it easier to detect a treatment effect if there truly is one to detect in the first place

\section{Factorial Treatment Structure and Interaction}
Interaction occurs if the effect of one factor on the response depends on the level of a second factor 
\subsection{Two Factors at Two Levels}
\subsubsection{Model}
To express interaction explicitly we write the model as 
\[Y_{ijk}=\mu+\alpha_i+\lambda_j+\gamma_{ij}+R_{ijk}\quad i=1,2\quad j=1,2\quad k=1,2,\cdots,r\]
where
\begin{itemize}
    \item $\mu$: mean response across all treatments 
    \item $\alpha_i$: effect of factor A at level $i$
    \item $\lambda_j$: effect of factor B at level $j$
    \item $\gamma_{ij}$: interaction effect of factor A at level $i$ and factor B at level $j$
\end{itemize}
\subsubsection*{Constraints}
\begin{itemize}
    \item For the main effects we have $\alpha_1+\alpha_2=0$ and $\lambda_1+\lambda_2=0$
    \item Then for the interaction effects we have 
    \[\sum_i\gamma_{ij}=0\quad \sum_j\gamma_{ij}=0\]
\end{itemize}
\subsection{Two Factors with More Than Two Levels}
\subsubsection{General Interaction Model}
\[Y_{ijkl}=\mu+\alpha_i+\lambda_j+\gamma_{ij}+R_{ijk}\]
where
\begin{itemize}
    \item $\mu$: overall mean response
    \item $\alpha_i$: effect of factor A at level $i$
    \item $\lambda_j$: effect of factor B at level $j$
    \item $\gamma_{ij}$: interaction effect of factor A at level $i$ and factor B at level $j$
\end{itemize}
\subsubsection{Interaction and Blocking Model}
\[Y_{ijkl}=\mu+\alpha_i+\lambda_j+\gamma_{ij}+\beta_k+R_{ijk}\]
where 
\begin{itemize}
    \item $\mu$: overall mean response
    \item $\alpha_i$: effect of factor A at level $i$
    \item $\lambda_j$: effect of factor B at level $j$
    \item $\gamma_{ij}$: interaction effect of factor A at level $i$ and factor B at level $j$
    \item $\beta_k$: effect of block $k$
\end{itemize}
The estimates we got are 
\begin{itemize}
    \item $\hat\mu=\overline{y}_{+++}$
    \item $\hat\alpha_i=\overline{y}_{i++}-\overline{y}_{+++}$
    \item $\hat\lambda_j=\overline{y}_{+j+}-\overline{y}_{+++}$
    \item $\hat\gamma_{ij}=\overline{y}_{ij+}-\overline{y}_{i++}-\overline{y}_{+j+}+\overline{y}_{+++}$
    \item $\hat\beta_k=\overline{y}_{++k}-\overline{y}_{+++}$
\end{itemize}
$\ $\newline 
\subsubsection{ANOVA Table}
\begin{figure*}[h!]
    \centering
    \includegraphics[width=1\textwidth]{anova for two factor block.png}
\end{figure*}
$\ $\newline 
The test statistic is the mean square ration between the test and error 

\end{document}